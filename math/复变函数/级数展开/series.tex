\documentclass[11pt,a4paper]{article}
%\usepackage{fontspec, xunicode, xltxtra}
%\setmainfont{Hiragino Sans GB}
\usepackage{xeCJK}
%\setCJKmainfont[BoldFont=STZhongsong, ItalicFont=STKaiti]{STSong}
%\setCJKsansfont[BoldFont=STHeiti]{STXihei}
%\setCJKmonofont{STFangsong}

%使用Xelatex编译

% 设置页面
%==================================================
\linespread{2} %行距
% \usepackage[top=1in,bottom=1in,left=1.25in,right=1.25in]{geometry}
% \headsep=2cm
% \textwidth=16cm \textheight=24.2cm
%==================================================

% 其它需要使用的宏包
%==================================================
\usepackage[colorlinks,linkcolor=blue,anchorcolor=red,citecolor=green,urlcolor=blue]{hyperref} 
\usepackage{tabularx}
\usepackage{authblk}         % 作者信息
\usepackage{algorithm}     % 算法排版
\usepackage{amsmath}     % 数学符号与公式
\usepackage{amsfonts}     % 数学符号与字体
\usepackage{mathrsfs}      % 花体
\usepackage{amssymb}
\usepackage[framemethod=TikZ]{mdframed}

\usepackage{graphicx} 
\usepackage{graphics}
\usepackage{color}
\usepackage{xcolor}
\usepackage{tcolorbox}
\usepackage{lipsum}
\usepackage{empheq}

\usepackage{fancyhdr}       % 设置页眉页脚
\usepackage{fancyvrb}       % 抄录环境
\usepackage{float}              % 管理浮动体
\usepackage{geometry}     % 定制页面格式
\usepackage{hyperref}       % 为PDF文档创建超链接
\usepackage{lineno}          % 生成行号
\usepackage{listings}        % 插入程序源代码
\usepackage{multicol}       % 多栏排版
%\usepackage{natbib}         % 管理文献引用
\usepackage{rotating}       % 旋转文字,图形,表格
\usepackage{subfigure}    % 排版子图形
\usepackage{titlesec}       % 改变章节标题格式
\usepackage{moresize}   % 更多字体大小
\usepackage{anysize}
\usepackage{indentfirst}  % 首段缩进
\usepackage{booktabs}   % 使用\multicolumn
\usepackage{multirow}    % 使用\multirow

\usepackage{wrapfig}
\usepackage{titlesec}     % 改变标题样式
\usepackage{enumitem}
\usepackage{aas_macros}

\renewcommand{\vec}[1]{\boldsymbol{#1}}
\newcommand{\me}{\mathrm{e}}
\newcommand{\mi}{\mathrm{i}}
\newcommand{\dif}{\mathrm{d}}
\newcommand{\tabincell}[2]{\begin{tabular}{@{}#1@{}}#2\end{tabular}}

\def\kpc{{\rm kpc}}
\def\km{{\rm km}}
\def\cm{{\rm cm}}
\def\TeV{{\rm TeV}}
\def\GeV{{\rm GeV}}
\def\MeV{{\rm MeV}}
\def\GV{{\rm GV}}
\def\MV{{\rm MV}}
\def\yr{{\rm yr}}
\def\s{{\rm s}}
\def\ns{{\rm ns}}
\def\GHz{{\rm GHz}}
\def\muGs{{\rm \mu Gs}}
\def\arcsec{{\rm arcsec}}
\def\K{{\rm K}}
\def\microK{\mu{\rm K}}
\def\sr{{\rm sr}}
\newcolumntype{p}{D{,}{\pm}{-1}}

\renewcommand{\figurename}{Fig.}
\renewcommand{\tablename}{Tab.}

\renewcommand{\arraystretch}{1.5}

\setlength{\parindent}{0pt}  %取消每段开头的空格

\newcounter{theo}[section]\setcounter{theo}{0}
\renewcommand{\thetheo}{\arabic{section}.\arabic{theo}}
\newenvironment{theo}[2][]{%
\refstepcounter{theo}%
\ifstrempty{#1}%
{\mdfsetup{%
frametitle={%
\tikz[baseline=(current bounding box.east),outer sep=0pt]
\node[anchor=east,rectangle,fill=blue!20]
{\strut Theorem~\thetheo};}}
}%
{\mdfsetup{%
frametitle={%
\tikz[baseline=(current bounding box.east),outer sep=0pt]
\node[anchor=east,rectangle,fill=blue!20]
{\strut Theorem~\thetheo:~#1};}}%
}%
\mdfsetup{innertopmargin=10pt,linecolor=blue!20,%
linewidth=2pt,topline=true,%
frametitleaboveskip=\dimexpr-\ht\strutbox\relax
}
\begin{mdframed}[]\relax%
\label{#2}}{\end{mdframed}}

\newcommand*\widefbox[1]{\fbox{\hspace{2em}#1\hspace{2em}}}

\title{级数展开}
\author{}
\date{\today}
\begin{document}

\maketitle
\section{序列与极限}
\subsection{序列的极限}
\begin{tcolorbox}[colback=green!5,colframe=green!40!black,title= 定义]
设$z_1, z_2, \cdots, z_n, \cdots$是复数序列,记作$\{z_n \}$。若任给正数$\varepsilon > 0$,存在自然数$N$,使当$n>N$时,总有
\begin{equation}
|z_n -z_0| < \varepsilon ~,
\end{equation}
成立。则称复数序列$\{z_n \}$收敛于复数$z_0$,或称$\{z_n \}$以$z_0$为极限,
\begin{equation*}
\underset{n\rightarrow +\infty}\lim z_n -z_0 ~\text{或}~ z_n \rightarrow z_0.
\end{equation*}
\end{tcolorbox}

\begin{tcolorbox}[colback=green!5,colframe=green!40!black,title= 定理]
序列$\{z_n \}$以$z_0$为极限的\textcolor{red}{充要条件}是
\begin{equation}
1) \underset{n\rightarrow +\infty}\lim x_n = x_0  ~\text{及}~ \underset{n\rightarrow +\infty}\lim y_n = y_0 
\end{equation}
其中$z_n = x_n +i y_n (n = 1, 2, \cdots), z_0 = x_0 +iy_0$,或 \\
2) 当$z_0 \neq 0$时,
\begin{equation*}
\underset{n\rightarrow +\infty}\lim |z_n| = |z_0|  ~\text{及}~  \underset{n\rightarrow +\infty}\lim {\rm Arg} z_n = {\rm Arg} z_0
\end{equation*}
\end{tcolorbox}

\begin{tcolorbox}[colback=green!5,colframe=green!40!black,title= 定理]
设序列$\{z_n \}$与序列$\{z^\prime_n \}$分别有极限为$z_0$及$z^\prime_0$,即
\begin{align*}
\underset{n\rightarrow +\infty}\lim z_n = z_0  ~, \underset{n\rightarrow +\infty}\lim z^\prime_n = z^\prime_0 ~,
\end{align*}
则
\begin{align*}
& 1) \underset{n\rightarrow +\infty}\lim (z_n \pm z^\prime_n) = z_0 +z^\prime_0 ~, \\
& 2) \underset{n\rightarrow +\infty}\lim z_n z^\prime_n = z_0 z^\prime_0 ~, \\
& 3) \underset{n\rightarrow +\infty}\lim \dfrac{z_n}{z^\prime_n} = \dfrac{z_0}{z^\prime_0} ~, z^\prime \neq 0 ~.
\end{align*}
\end{tcolorbox}

\begin{tcolorbox}[colback=green!5,colframe=green!40!black,title= 定义]
对于一个复数序列$\{z_n \}$,若存在一个正数$M$,使$|z_n| \leqslant M (n = 1, 2, \cdots)$,就称$\{z_n \}$是有界的;否则,就称$\{z_n \}$是无界的。
\end{tcolorbox}

\begin{tcolorbox}[colback=green!5,colframe=green!40!black,title= 定义]
对于一个复数序列$\{z_n \}$,若存在一个正数$M$,使$|z_n| \leqslant M (n = 1, 2, \cdots)$,就称$\{z_n \}$是有界的;否则,就称$\{z_n \}$是无界的。
\end{tcolorbox}


\begin{tcolorbox}[colback=green!5,colframe=green!40!black,title= 定义]
对于一个复数序列$\{z_n \}$,若任给$M > 0$,可以找到自然数$N$,使当$n > N$时,有
\begin{equation*}
|z_n| > M ~,
\end{equation*}
就称$\{z_n \}$趋向于$\infty$,记作$\underset{n\rightarrow +\infty}\lim z_n = \infty$。
\end{tcolorbox}

复数序列$\{z_n \}$收敛的柯西(Cauchy)判别法:
\begin{tcolorbox}[colback=green!5,colframe=green!40!black,title= 定理]
复数序列$\{z_n \}$有极限的\textcolor{red}{充要条件}是:任给$\varepsilon > 0$,可以找到$N$,使当$n > N$时,对于任何自然数$m$,有
\begin{align*}
|z_{n+m} - z_n| < \varepsilon  ~. 
\end{align*}
\end{tcolorbox}


\subsection{矩形套定理 ~ 列紧性定理 ~覆盖定理}
\begin{tcolorbox}[colback=green!5,colframe=green!40!black,title= 矩形套定理]
设有一串矩形$R_n = \{ a_n \leqslant x \leqslant b_n, c_n \leqslant y \leqslant d_n \} (n = 1, 2, \cdots)$,且足标较大的矩形包含在足标较小的矩形中,即$R_1 \supset R_2 \supset \cdots \supset R_n \supset \cdots$。又设矩形$R_n$的对角线$D_n \rightarrow 0$,则有且仅有一个点属于所有的矩形$R_n (n = 1, 2, \cdots)$。
\end{tcolorbox}


\begin{tcolorbox}[colback=green!5,colframe=green!40!black,title= 列紧性定理]
设复数序列$\{z_n \}$是有界的,则必存在一个收敛的子序列$\{z_{n_i} \}$。
\end{tcolorbox}


\begin{tcolorbox}[colback=green!5,colframe=green!40!black,title= 覆盖定理]
设$F$是有界闭集,$K$是一些圆的集合(实际只要开集即可),且$K$覆盖$F$,这表示对于$F$中的任一点$z$,在$K$中一定可以找到一个圆,使点$z$属于这个圆,则在$K$中一定可以找到有限个圆,也覆盖闭集$F$。
\end{tcolorbox}




\subsection{复数项级数}
\begin{tcolorbox}[colback=green!5,colframe=green!40!black,title= 定义]
考虑复数项级数
\begin{equation}
u_1 +u_2 +\cdots +u_n + \cdots
\label{series}
\end{equation}
构成部分和
\begin{equation}
s_n = \sum_{k=1}^n u_k = u_1 +u_2 +\cdots +u_n ~,
\end{equation}
得到序列$\{s_n\}$,若序列$\{s_n\}$有极限为$s$:
\begin{equation*}
\underset{n\rightarrow +\infty}\lim s_n = s ~,
\end{equation*}
则称级数\textcolor{red}{收敛},其和为$s$,记作$\sum\limits_{k=1}^\infty u_k =s$。若序列$\{s_n\}$没有极限,则称级数\textcolor{red}{发散}。
\end{tcolorbox}


\begin{tcolorbox}[colback=green!5,colframe=green!40!black,title= 定理]
设级数(\ref{series})收敛,则
\begin{equation}
\underset{n\rightarrow +\infty} u_n = 0 ~.
\end{equation}
\end{tcolorbox}

\begin{tcolorbox}[colback=green!5,colframe=green!40!black,title= 定理]
级数(\ref{series})收敛的充要条件是:任给$\varepsilon > 0$,存在自然数$N$,使当$n > N$时,对任意自然数$m$,总有
\begin{equation}
|u_{n+1} +u_{n+2} +\cdots +u_{n+m}| <  \varepsilon ~.
\end{equation}
\end{tcolorbox}


\begin{tcolorbox}[colback=green!5,colframe=green!40!black,title= 定义]
若级数(\ref{series})中每一项取模后得到级数
\begin{equation}
|u_{1}| + |u_{2}| +\cdots +|u_{n}| +\cdots
\end{equation}
收敛,则称级数(\ref{series})\textcolor{red}{绝对收敛}。
\end{tcolorbox}
绝对收敛级数一定是收敛的。

\begin{tcolorbox}[colback=green!5,colframe=green!40!black,title= 定理]
若级数(\ref{series})中的每一项$u_n$对充分大的$n$,满足
\begin{equation}
|u_{n}| \leqslant M_n ~,
\end{equation}
且级数$\sum\limits_{k=1}^\infty M_n$收敛,则级数(\ref{series})\textcolor{red}{绝对收敛}。
\end{tcolorbox}

设有两个级数
\begin{align}
u_1 +u_2 +\cdots +u_n \cdots ~, \\
u^\prime_1 +u^\prime_2 +\cdots +u^\prime_n \cdots ~,
\label{series2}
\end{align}
由它们构造另外两个级数
\begin{align}
\label{series3}
& (u_1 \pm u^\prime_1) +(u_2 \pm u^\prime_2) +\cdots +(u_n \pm u^\prime_n) +\cdots ~, \\
\nonumber & u_1u^\prime_1 +(u_1u^\prime_2 +u_2 u^\prime_1) +(u_1u^\prime_3 +u_2 u^\prime_2 +u_3u^\prime_1) +\cdots \\
& +(u_1u^\prime_n +u_2 u^\prime_{n-1} +\cdots +u_{n-1}u^\prime_2 +u_n u^\prime_1) +\cdots ~,
\label{series4}
\end{align}


\begin{tcolorbox}[colback=green!5,colframe=green!40!black,title= 定理]
设级数(\ref{series})与级数(\ref{series2})都绝对收敛,且
\begin{equation}
\sum_{n=1}^\infty u_n = s, ~~ \sum_{n=1}^\infty u^\prime_n = s^\prime ~.
\end{equation}
则级数(\ref{series3})及级数(\ref{series4})也绝对收敛,且
\begin{align}
& \sum_{n=1}^\infty (u_n \pm u^\prime_n) = s \pm s^\prime, \\
& \sum_{n=1}^\infty (u_1u^\prime_n +u_2u^\prime_{n-1} +\cdots +u_{n-1}u^\prime_{2} +u_{n}u^\prime_{1}) = s s^\prime, 
\end{align}
\end{tcolorbox}



































\end{document}