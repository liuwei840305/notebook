\documentclass[12pt,a4paper]{article}
%\usepackage{fontspec, xunicode, xltxtra}
%\setmainfont{Hiragino Sans GB}
\usepackage{xeCJK}
%\setCJKmainfont[BoldFont=STZhongsong, ItalicFont=STKaiti]{STSong}
%\setCJKsansfont[BoldFont=STHeiti]{STXihei}
%\setCJKmonofont{STFangsong}

%使用Xelatex编译

% 设置页面
%==================================================
\linespread{2} %行距
% \usepackage[top=1in,bottom=1in,left=1.25in,right=1.25in]{geometry}
% \headsep=2cm
% \textwidth=16cm \textheight=24.2cm
%==================================================

% 其它需要使用的宏包
%==================================================
\usepackage[colorlinks,linkcolor=blue,anchorcolor=red,citecolor=green,urlcolor=blue]{hyperref}
\usepackage{tabularx}
\usepackage{authblk}         % 作者信息
\usepackage{algorithm}     % 算法排版
\usepackage{amsmath}     % 数学符号与公式
\usepackage{amsfonts}     % 数学符号与字体
\usepackage{mathrsfs}      % 花体
\usepackage{graphics}
\usepackage{color}
\usepackage{fancyhdr}       % 设置页眉页脚
\usepackage{fancyvrb}       % 抄录环境
\usepackage{float}              % 管理浮动体
\usepackage{geometry}     % 定制页面格式
\usepackage{hyperref}       % 为PDF文档创建超链接
\usepackage{lineno}          % 生成行号
\usepackage{listings}        % 插入程序源代码
\usepackage{multicol}       % 多栏排版
\usepackage{natbib}         % 管理文献引用
\usepackage{rotating}       % 旋转文字,图形,表格
\usepackage{subfigure}    % 排版子图形
\usepackage{titlesec}       % 改变章节标题格式
\usepackage{moresize}   % 更多字体大小
\usepackage{anysize}
\usepackage{indentfirst}  % 首段缩进
\usepackage{booktabs}   % 使用\multicolumn
\usepackage{multirow}    % 使用\multirow
\usepackage{graphicx}
\usepackage{wrapfig}
\usepackage{xcolor}
\usepackage{titlesec}     % 改变标题样式
\usepackage{enumitem}

\renewcommand{\vec}[1]{\boldsymbol{#1}}
\newcommand{\me}{\mathrm{e}}
\newcommand{\mi}{\mathrm{i}}
\newcommand{\dif}{\mathrm{d}}
\newcommand{\tabincell}[2]{\begin{tabular}{@{}#1@{}}#2\end{tabular}}

\def\kpc{{\rm kpc}}
\def\km{{\rm km}}
\def\cm{{\rm cm}}
\def\TeV{{\rm TeV}}
\def\GeV{{\rm GeV}}
\def\MeV{{\rm MeV}}
\def\GV{{\rm GV}}
\def\MV{{\rm MV}}
\def\yr{{\rm yr}}
\def\s{{\rm s}}
\def\ns{{\rm ns}}
\def\GHz{{\rm GHz}}
\def\muGs{{\rm \mu Gs}}
\def\arcsec{{\rm arcsec}}
\def\K{{\rm K}}
\def\microK{\mu{\rm K}}
\def\sr{{\rm sr}}
\newcolumntype{p}{D{,}{\pm}{-1}}

\renewcommand{\figurename}{Fig.}
\renewcommand{\tablename}{Tab.}

\renewcommand{\arraystretch}{1.5}

\setlength{\parindent}{0pt}  %取消每段开头的空格

\title{解析函数}
\author{}
\date{\today}
\begin{document}

\maketitle
1. 设$E$为平面点集。若对于$E$内每一个复数$z$,按一定规律,有一个复数$\omega$与之对应,则称$\omega$为$z$的函数(单值函数),记作$\omega = f(z)$,点集$E$称为这个函数的自变量$z$的\textcolor{red}{定义域},$\omega$称为\textcolor{red}{因变量}。

若对于自变量$z \in E$,对应着几个或无穷多个值$\omega$,则称在$E$上确定了一个多值函数$\omega = f(z)$。

若$P$中每一个点$\omega$,通过关系式$\omega = f(z)$只有一个点$z \in E$与之对应,则在$P$上也确定了一个单值函数,记作$z = g(\omega)$,称为函数$\omega = f(z)$的反函数或称变换$f(z)$的逆变换。

若$P$中存在点$\omega$,通过关系式$\omega = f(z)$在$E$中至少有两个点与之相对应,则在$P$上就确定了一个多值函数,记作$z = g(\omega)$,称为变换$f(z)$的逆变换。

\section{极限与连续}
\subsection{极限}
设$E$是复平面上的点集,$z_0$是$E$的一个\textcolor{red}{凝聚点},而函数$\omega = f(z)$定义在$E$上。若存在复数$l$,使得对于任意给定的实数$s > 0$,都存在实数$\delta > 0$,使当$z \in E$及$0 < |z-z_0| < \delta$时,都满足
\begin{equation}
|f(z) -l| < s ~,
\end{equation}
则称函数$f(z)$当$z$在$E$中趋向于$z_0$时有\textcolor{red}{极限}$l$,记作
\begin{equation}
\underset{\mathop{z \rightarrow z_0}\limits_{(z \in E)}}{\lim} f(z) = l ~.
\end{equation}
若$E$包含有$z_0$的邻域$S(z_0)$或包含有$z_0$的邻域除去$z_0$的点集,则以上极限关系简写为
\begin{equation}
\underset{z \rightarrow z_0}{\lim}f(z) = l ~.
\end{equation}

几何意义:以复数$l$为中心,半径$\epsilon > 0$作一个圆$S_{\epsilon}(l)$,则可以找到$z_0$的一个充分小的邻域——它可以是半径为$\delta$、中心为$z_0$的圆$S_{\delta}(z_0)$,当$z\in E$,$z\neq z_0$进入这个邻域中时,对应的值$\omega = f(z)$就位于圆$S_{\epsilon}(l)$中。

\subsection{连续}
设$E$是复平面上的点集,$z_0$是$E$的一个\textcolor{red}{凝聚点},$z_0\in E$,而函数$\omega = f(z)$定义在$E$上。若
\begin{equation}
\underset{\mathop{z \rightarrow z_0}\limits_{(z \in E)}}{\lim} f(z) = f(z_0) ~.
\end{equation}
即任给$\epsilon > 0$,存在数$\delta > 0$,使得当$z \in E$,且满足$|z-z_0| < \delta$时,总有
\begin{equation}
|f(z) -f(z_0)| < \epsilon
\end{equation}
则称函数$\omega = f(z)$沿着集合$E$在点$z=z_0$处\textcolor{red}{连续}。

若复平面的集合$E$上的每一点都是$E$的凝聚点,且函数$\omega = f(z)$在$E$上每一点都连续,则称函数$f(z)$在$E$上连续。




\section{复变函数的导数}
设函数$\omega = f(z)$在$z=z_0$的邻域$S(z_0)$上有定义,比值
\begin{equation}
\frac{\Delta \omega}{\Delta z} = \frac{f(z) -f(z_0)}{z -z_0}
\end{equation}
若$z$不论以什么方式趋向于$z_0$时,上式都存在极限,则称这个极限值为函数$f(z)$在$z=z_0$处的\textcolor{red}{导数},记作$f^{\prime}(z_0)$,并说函数$f(z)$在$z=z_0$处\textcolor{red}{可导},即
\begin{equation}
\underset{z \rightarrow z_0}{\lim} \frac{f(z) -f(z_0)}{z-z_0} = f^{\prime} (z_0) ~~\text{或} ~~ [f(z_0)]^{\prime}
\end{equation}


\subsection{柯西-黎曼方程}


设函数$f(z) = u(x, y) +iv(x, y)$在区域$D$内有定义,则它在$D$内\textcolor{red}{解析}的\textcolor{red}{充分必要条件}是:

1. $u(x, y)$与$v(x, y)$在$D$内处处可微;

2. $u(x, y)$与$v(x, y)$在$D$内处处满足一阶偏微分方程组
\begin{eqnarray}
\frac{\partial u}{\partial x} &=& \frac{\partial v}{\partial y} \\
\frac{\partial u}{\partial y} &=& -\frac{\partial v}{\partial x}
\end{eqnarray}




\section{初等解析函数}







\section{调和函数}










\end{document}