\documentclass[12pt,a4paper]{article}
%\usepackage{fontspec, xunicode, xltxtra}  
%\setmainfont{Hiragino Sans GB}  
\usepackage{xeCJK}
\setCJKmainfont[BoldFont=STZhongsong, ItalicFont=STKaiti]{STSong}
\setCJKsansfont[BoldFont=STHeiti]{STXihei}
\setCJKmonofont{STFangsong}

%使用Xelatex编译

% 设置页面
%==================================================
\linespread{2} %行距
% \usepackage[top=1in,bottom=1in,left=1.25in,right=1.25in]{geometry}
% \headsep=2cm
% \textwidth=16cm \textheight=24.2cm
%==================================================

% 其它需要使用的宏包
%==================================================
\usepackage[colorlinks,linkcolor=blue,anchorcolor=red,citecolor=green,urlcolor=blue]{hyperref} 
\usepackage{tabularx}
\usepackage{authblk}         % 作者信息
\usepackage{algorithm}     % 算法排版
\usepackage{amsmath}     % 数学符号与公式
\usepackage{amsfonts}     % 数学符号与字体
\usepackage{mathrsfs}      % 花体
\usepackage{graphics}
\usepackage{color}
\usepackage{fancyhdr}       % 设置页眉页脚
\usepackage{fancyvrb}       % 抄录环境
\usepackage{float}              % 管理浮动体
\usepackage{geometry}     % 定制页面格式
\usepackage{hyperref}       % 为PDF文档创建超链接
\usepackage{lineno}          % 生成行号
\usepackage{listings}        % 插入程序源代码
\usepackage{multicol}       % 多栏排版
\usepackage{natbib}         % 管理文献引用
\usepackage{rotating}       % 旋转文字,图形,表格
\usepackage{subfigure}    % 排版子图形
\usepackage{titlesec}       % 改变章节标题格式
\usepackage{moresize}   % 更多字体大小
\usepackage{anysize}
\usepackage{indentfirst}  % 首段缩进
\usepackage{booktabs}   % 使用\multicolumn
\usepackage{multirow}    % 使用\multirow
\usepackage{graphicx} 
\usepackage{wrapfig}
\usepackage{xcolor}
\usepackage{titlesec}     % 改变标题样式
\usepackage{enumitem}
\usepackage{harpoon}   %矢量符号

\newcommand{\myvec}[1]%
   {\stackrel{\raisebox{-2pt}[0pt][0pt]{\small$\rightharpoonup$}}{#1}}  %矢量符号
\renewcommand{\vec}[1]{\boldsymbol{#1}}
\newcommand{\me}{\mathrm{e}}
\newcommand{\mi}{\mathrm{i}}
\newcommand{\dif}{\mathrm{d}}
\newcommand{\tabincell}[2]{\begin{tabular}{@{}#1@{}}#2\end{tabular}}

\def\kpc{{\rm kpc}}
\def\km{{\rm km}}
\def\cm{{\rm cm}}
\def\TeV{{\rm TeV}}
\def\GeV{{\rm GeV}}
\def\MeV{{\rm MeV}}
\def\GV{{\rm GV}}
\def\MV{{\rm MV}}
\def\yr{{\rm yr}}
\def\s{{\rm s}}
\def\ns{{\rm ns}}
\def\GHz{{\rm GHz}}
\def\muGs{{\rm \mu Gs}}
\def\arcsec{{\rm arcsec}}
\def\K{{\rm K}}
\def\microK{\mu{\rm K}}
\def\sr{{\rm sr}}
\newcolumntype{p}{D{,}{\pm}{-1}}

\renewcommand{\figurename}{Fig.}
\renewcommand{\tablename}{Tab.}

\renewcommand{\arraystretch}{1.5}

\setlength{\parindent}{0pt}  %取消每段开头的空格

\title{随机变量的数字特征}
\author{}
\date{\today}
\begin{document}

\maketitle

\section{数学期望}
又称\textcolor{red}{期望},\textcolor{red}{均值} 

设离散型随机变量$X$的分布律
\begin{equation}
P\{X= x_k\} = p_k, ~~k = 1,2, \cdots
\end{equation}
若级数
\begin{equation}
\sum_{k=1}^{\infty} x_k p_k
\end{equation}
绝对收敛,则称级数$\sum_{k=1}^{\infty} x_k p_k$的和为随机变量$X$的\textcolor{red}{数学期望},记为$E(X)$,
\begin{equation}
E(X) = \sum_{k=1}^{\infty} x_k p_k .
\end{equation}
设连续型随机变量$X$的概率密度为$f(x)$,若积分
\begin{equation}
\int_{-\infty}^{\infty} xf(x) \dif x
\end{equation}
绝对收敛,则称积分$\int_{-\infty}^{\infty} xf(x) \dif x$的值为随机变量$X$的\textcolor{red}{数学期望},记为$E(X)$,
\begin{equation}
E(X) = \int_{-\infty}^{\infty} xf(x) \dif x .
\end{equation}

设Y是随机变量$X$的函数:
\begin{equation}
Y = g(X),
\end{equation}
($g$是连续函数)。\\
i) $X$是离散型随机变量,它的分布律为
\begin{equation}
P\{X= x_k\} = p_k, ~~k = 1,2, \cdots
\end{equation}
若
\begin{equation}
\sum_{k=1}^{\infty} g(x_k) p_k
\end{equation}
绝对收敛,则有
\begin{equation}
E(Y) = E[g(X)] = \sum_{k=1}^{\infty} g(x_k) p_k 
\end{equation}
ii) $X$是连续型随机变量,它的概率密度为$f(x)$。若
\begin{equation}
\int_{-\infty}^{\infty}  g(x) f(x) \dif x 
\end{equation}
绝对收敛,则有
\begin{equation}
E(Y) = E[g(X)] = \int_{-\infty}^{\infty} g(x) f(x) \dif x
\end{equation}

数学性质

设随机变量的数学期望存在,

1) 设$C$是常数,则有
\begin{equation}
E(C) = C
\end{equation}
2) 设$X$是一随机变量,$C$是常数,则有
\begin{equation}
E(CX) = C E(X)
\end{equation}
3) 设$X$,$Y$是两个随机变量,则有
\begin{equation}
E(X+Y) = E(X) + E(Y)
\end{equation}
4) 设$X$,$Y$是相互独立的随机变量,则有
\begin{equation}
E(XY) = E(X)E(Y)
\end{equation}

\section{方差}
设$X$是一个随机变量,若$E\{[X -E(X)]^2\}$存在,则称
\begin{equation*}
E\{[X -E(X)]^2\}
\end{equation*}
为$X$的\textcolor{red}{方差},记为$D(X)$,或${\rm Var}(X)$
\begin{equation}
D(X) = {\rm Var}(X) = E\{[X -E(X)]^2\} = E(X^2) -[E(X)]^2
\end{equation}

\textcolor{red}{标准差}、\textcolor{red}{均方差}
\begin{equation}
\sigma(X) = \sqrt{D(X)}
\end{equation}

数学性质

1) 设$C$是常数,则有
\begin{equation}
D(C) = 0,
\end{equation}
2) 设$X$是随机变量,$C$是常数,则有
\begin{equation}
D(CX) = C^2 D(X),
\end{equation}
3) 设$X$,$Y$是两个随机变量,则有
\begin{eqnarray}
\nonumber D(X+Y) &=& D(X) + D(Y) +2E\{[X-E(X)][Y-E(Y)]\} \\
\nonumber &=& D(X) + D(Y) +2\{E(XY) -E(X)E(Y) \}, \\
 &=& D(X) + D(Y) +2{\rm Cov}(X, Y)
\end{eqnarray}
若$X$,$Y$相互独立,则有
\begin{equation}
D(X+Y) = D(X) + D(Y),
\end{equation}
4) $D(X)=0$的充要条件是$X$以概率$1$取常数$C$,即
\begin{equation}
P\{X =C\} =1
\end{equation}

若$X_i \sim N(\mu_i, \sigma^2_i), i = 1,2,\cdots, n$,且它们相互独立,则它们的线性组合:
\begin{equation}
C_1 X_1 +C_2 X_2 +\cdots +C_n X_n,
\end{equation}
($C_1, C_2, \cdots, C_n$是不全为$0$的常数),仍然服从正态分布,
\begin{equation}
C_1 X_1 +C_2 X_2 +\cdots +C_n X_n \sim N\left(\sum_{i=1}^{n}C_i \mu_i, \sum_{i=1}^{n} C^2_i \sigma^2_i \right )
\end{equation}

\textcolor{red}{切比雪夫不等式}

设随机变量$X$具有数学期望
\begin{equation}
E(X) = \mu,
\end{equation}
方差
\begin{equation}
D(X) = \sigma^2,
\end{equation}
则对于任意正数$\varepsilon$,不等式
\begin{equation}
P\{|X-\mu| \geq \varepsilon\} \leq \frac{\sigma^2}{\varepsilon^2}
\end{equation}
成立。


\section{协方差、相关系数}

\textcolor{red}{协方差}
\begin{equation}
{\rm Cov}(X, Y) = E\{[X-E(X)][Y-E(Y)]\}
\end{equation}

\textcolor{red}{相关系数}
\begin{equation}
\rho_{XY} = \frac{{\rm Cov}(X, Y)}{\sqrt{D(X)}\sqrt{D(Y)}}
\end{equation}

$X$和$Y$\textcolor{red}{不相关}:
\begin{equation}
\rho_{XY} = 0
\end{equation}



\section{矩、协方差矩阵}
设$X$,$Y$是两个随机变量,

若
\begin{equation}
E(X^k), k = 1,2,\cdots
\end{equation}
存在,称它为$X$的\textcolor{red}{$k$阶原点矩},简称\textcolor{red}{$k$阶矩};

若
\begin{equation}
E\{[X-E(X)]^k\}, k = 1,2,3,\cdots
\end{equation}
存在,称它为$X$的\textcolor{red}{$k$阶中心矩};

若
\begin{equation}
E(X^kY^l), k, l = 1,2,\cdots
\end{equation}
存在,称它为$X$和$Y$的\textcolor{red}{$k+l$阶混合矩};

若
\begin{equation}
E\{[X-E(X)]^k [Y-E(Y)]^l\}, k = 1,2,3,\cdots
\end{equation}
存在,称它为$X$和$Y$的\textcolor{red}{$k+l$阶混合中心矩};\\

设$n$维随机变量$(X_1, X_2, \cdots, X_n)$的二阶混合中心矩
\begin{equation}
c_{ij} = {\rm Cov}(X_i, X_j) = E\{[X_i-E(X_i)] [X_j-E(X_j)]\}, i,j = 1,2,\cdots, n
\end{equation}
都存在,则称矩阵
\begin{equation}
C = \begin{pmatrix}
c_{11} & c_{12} & \dots & c_{1n} \\
c_{21} & c_{22} & \dots & c_{2n} \\
\vdots  & \vdots &            & \vdots \\
c_{n1} & c_{n2} & \dots & c_{nn} \\
\end{pmatrix}
\end{equation}
为$n$维随机变量$(X_1, X_2, \cdots, X_n)$的\textcolor{red}{协方差矩阵};对称矩阵:$c_{ij} = c_{ji}$


































\end{document}