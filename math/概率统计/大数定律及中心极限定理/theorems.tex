\documentclass[12pt,a4paper]{article}
%\usepackage{fontspec, xunicode, xltxtra}  
%\setmainfont{Hiragino Sans GB}  
\usepackage{xeCJK}
%\setCJKmainfont[BoldFont=STZhongsong, ItalicFont=STKaiti]{STSong}
%\setCJKsansfont[BoldFont=STHeiti]{STXihei}
%\setCJKmonofont{STFangsong}

%使用Xelatex编译

% 设置页面
%==================================================
\linespread{2} %行距
% \usepackage[top=1in,bottom=1in,left=1.25in,right=1.25in]{geometry}
% \headsep=2cm
% \textwidth=16cm \textheight=24.2cm
%==================================================

% 其它需要使用的宏包
%==================================================
\usepackage[colorlinks,linkcolor=blue,anchorcolor=red,citecolor=green,urlcolor=blue]{hyperref} 
\usepackage{tabularx}
\usepackage{authblk}         % 作者信息
\usepackage{algorithm}     % 算法排版
\usepackage{amsmath}     % 数学符号与公式
\usepackage{amssymb}
\usepackage{amsfonts}     % 数学符号与字体
\usepackage{mathrsfs}      % 花体
\usepackage{graphics}
\usepackage{color}
\usepackage{fancyhdr}       % 设置页眉页脚
\usepackage{fancyvrb}       % 抄录环境
\usepackage{float}              % 管理浮动体
\usepackage{geometry}     % 定制页面格式
\usepackage{hyperref}       % 为PDF文档创建超链接
\usepackage{lineno}          % 生成行号
\usepackage{listings}        % 插入程序源代码
\usepackage{multicol}       % 多栏排版
%\usepackage{natbib}         % 管理文献引用
\usepackage{rotating}       % 旋转文字,图形,表格
\usepackage{subfigure}    % 排版子图形
\usepackage{titlesec}        % 改变章节标题格式
\usepackage{moresize}    % 更多字体大小
\usepackage{anysize}
\usepackage{indentfirst}   % 首段缩进
\usepackage{booktabs}    % 使用\multicolumn
\usepackage{multirow}     % 使用\multirow
\usepackage{graphicx} 
\usepackage{wrapfig}
\usepackage{xcolor}
\usepackage{titlesec}       % 改变标题样式
\usepackage{enumitem}
\usepackage{harpoon}    %矢量符号
\usepackage{xcolor}        % 高亮




\newcommand{\myvec}[1]%
   {\stackrel{\raisebox{-2pt}[0pt][0pt]{\small$\rightharpoonup$}}{#1}}  %矢量符号
\renewcommand{\vec}[1]{\boldsymbol{#1}}
\newcommand{\me}{\mathrm{e}}
\newcommand{\mi}{\mathrm{i}}
\newcommand{\dif}{\mathrm{d}}
\newcommand{\tabincell}[2]{\begin{tabular}{@{}#1@{}}#2\end{tabular}}

\def\kpc{{\rm kpc}}
\def\km{{\rm km}}
\def\cm{{\rm cm}}
\def\TeV{{\rm TeV}}
\def\GeV{{\rm GeV}}
\def\MeV{{\rm MeV}}
\def\GV{{\rm GV}}
\def\MV{{\rm MV}}
\def\yr{{\rm yr}}
\def\s{{\rm s}}
\def\ns{{\rm ns}}
\def\GHz{{\rm GHz}}
\def\muGs{{\rm \mu Gs}}
\def\arcsec{{\rm arcsec}}
\def\K{{\rm K}}
\def\microK{\mu{\rm K}}
\def\sr{{\rm sr}}
\newcolumntype{p}{D{,}{\pm}{-1}}

\renewcommand{\figurename}{Fig.}
\renewcommand{\tablename}{Tab.}

\renewcommand{\arraystretch}{1.5}

\setlength{\parindent}{0pt}  %取消每段开头的空格



\title{大数定律及中心极限定理}
\author{}
\date{\today}
\begin{document}

\maketitle

\section{大数定律}
\subsection{契比雪夫定理的特殊情况}

设随机变量$X_1, X_2, \cdots, X_n, \cdots$相互独立,且具有相同的数学期望和方差:$E(X_k) = \mu, D(X_k) = \sigma^2 (k = 1, 2, \cdots)$。作前$n$个随机变量的算术平均
\begin{equation}
\bar{X} = \frac{1}{n} \sum_{k=1}^n X_k ,
\end{equation}
则对于任意正数$\varepsilon$,有
\begin{equation}
\lim_{n\rightarrow \infty} P\{|\bar{X} -\mu| < \varepsilon \} = \lim_{n\rightarrow \infty} P\left \{\bigg|\frac{1}{n} \sum_{k=1}^n X_k -\mu \bigg| < \varepsilon \right \} = 1
\end{equation}


设$Y_1, Y_2, \cdots, Y_n, \cdots$是一个随机变量序列,$a$是一个常数。若对任意正数$\varepsilon$,有
\begin{equation}
\lim_{n\rightarrow \infty} P\{|Y_n -a| < \varepsilon \}  = 1,
\end{equation}
称序列$Y_1, Y_2, \cdots, Y_n, \cdots$\textcolor{red}{依概率收敛于}$a$。记为
\begin{equation}
Y_n \stackrel{P}{\longrightarrow} a.
\end{equation}

设$X_n \stackrel{P}{\longrightarrow} a, Y_n \stackrel{P}{\longrightarrow} b$,又设函数$g(x,y)$在点$(a,b)$连续,则
\begin{equation}
g(X_n, Y_n) \stackrel{P}{\longrightarrow} g(a, b).
\end{equation}

定理一:

设随机变量$X_1, X_2, \cdots, X_n, \cdots$相互独立,且具有相同的数学期望和方差:$E(X_k) = \mu, D(X_k) = \sigma^2 (k = 1, 2, \cdots)$。则序列
\begin{equation}
\bar{X} = \frac{1}{n} \sum_{k=1}^n X_k
\end{equation}
依概率收敛于$\mu$,即$\bar{X} \stackrel{P}{\longrightarrow} \mu$。


\subsection{伯努利大数定理}

设$n_A$是$n$次独立重复试验中事件$A$发生的次数,$p$是事件$A$在每次试验 中发生的概率,则对于任意正数$\varepsilon > 0$,有
\begin{equation}
\lim_{n\rightarrow \infty} P\left \{\bigg|\frac{n_A}{n}  -p \bigg| < \varepsilon \right \} = 1
\end{equation}
或
\begin{equation}
\lim_{n\rightarrow \infty} P\left \{\bigg|\frac{n_A}{n}  -p \bigg| \geqslant \varepsilon \right \} = 0
\end{equation}
伯努利大数定理表明事件发生的频率$n_A/n$依概率收敛于事件的概率$p$。频率的稳定性;当试验次数很大时,可以用事件发生的频率来代替事件的概率;

\subsection{辛钦定理}
设随机变量$X_1, X_2, \cdots, X_n, \cdots$相互独立,服从同一分布,且具有数学期望$E(X_k) = \mu (k = 1, 2, \cdots)$,则对于任意正数$\varepsilon$,有
\begin{equation}
\lim_{n\rightarrow \infty} P\left \{\bigg|\frac{1}{n} \sum_{k=1}^n X_k -\mu \bigg| < \varepsilon \right \} = 1
\end{equation}



\section{中心极限定理}
\subsection{独立同分布的中心极限定理}
设随机变量$X_1, X_2, \cdots, X_n, \cdots$相互独立,服从同一分布,且具有数学期望和方差:$E(X_k) = \mu, D(X_k) = \sigma^2 > 0 (k = 1, 2, \cdots)$,则随机变量之和$\sum\limits_{k=1}^n X_k$的标准化变量:
\begin{equation}
Y_n = \frac{\sum\limits_{k=1}^n X_k -E\left(\sum\limits_{k=1}^n X_k \right)}{\sqrt{D\left(\sum\limits_{k=1}^n X_k \right)}} = \frac{\sum\limits_{k=1}^n X_k - n\mu}{\sqrt{n} \sigma}
\end{equation}
的分布函数$F_n(x)$对于任意$x$满足
\begin{eqnarray}
\nonumber \lim_{n\rightarrow \infty} F_n(x) &=& \lim_{n\rightarrow \infty} P\left\{\frac{\sum\limits_{k=1}^n X_k - n\mu}{\sqrt{n} \sigma} \leqslant x \right\} \\
&=& \int_{-\infty}^x \frac{1}{\sqrt{2\pi}} e^{-t^2/2} \dif t = \Phi(x) 
\end{eqnarray}


\subsection{李雅普诺夫(Liapunov)定理}
设随机变量$X_1, X_2, \cdots, X_n, \cdots$相互独立,它们具有数学期望和方差:
\begin{eqnarray}
\nonumber E(X_k) &=& \mu_k, D(X_k) = \sigma_k^2 > 0 ~k = 1, 2, \cdots
\end{eqnarray}
记
\begin{equation}
\nonumber B_n^2 = \sum_{k=1}^n \sigma_k^2 
\end{equation}
若存在正数$\delta$,使得当$n\rightarrow\infty$时,
\begin{equation}
\frac{1}{B_n^{2+\delta}} \sum_{k=1}^n E\{|X_k -\mu_k|^{2+\delta}  \} \rightarrow 0,
\end{equation}
则随机变量之和$\sum\limits_{k=1}^n X_k$的标准化变量:
\begin{equation}
Z_n = \frac{\sum\limits_{k=1}^n X_k -E\left(\sum\limits_{k=1}^n X_k \right)}{\sqrt{D\left(\sum\limits_{k=1}^n X_k \right)}} = \frac{\sum\limits_{k=1}^n X_k - \sum\limits_{k=1}^n \mu_k}{B_n}
\end{equation}
的分布函数$F_n(x)$对于任意$x$,满足
\begin{eqnarray}
\nonumber \lim_{n\rightarrow \infty} F_n(x) &=& \lim_{n\rightarrow \infty} P\left\{\frac{\sum\limits_{k=1}^n X_k - \sum\limits_{k=1}^n \mu_k}{B_n} \leqslant x \right\} \\
&=& \int_{-\infty}^x \frac{1}{\sqrt{2\pi}} e^{-t^2/2} \dif t = \Phi(x)
\end{eqnarray}

在定理的条件下,随机变量
\begin{equation}
Z_n = \frac{\sum\limits_{k=1}^n X_k - \sum\limits_{k=1}^n \mu_k}{B_n}
\end{equation}
当$n$很大时,近似地服从正态分布$N(0, 1)$。由此,当$n$很大时,$\sum\limits_{k=1}^n X_k = B_n Z_n +\sum\limits_{k=1}^n \mu_k$近似地服从正态分布$N\left(\sum\limits_{k=1}^n \mu_k, B_n^2 \right)$。


无论各个随机变量$X_k (k =1, 2, \cdots)$服从什么分布,只有满足定理的条件,那么它们的和$\sum\limits_{k=1}^n X_k$当$n$很大时,就近似地服从正态分布。


\subsection{棣莫弗—拉普拉斯(De Moivre—Laplace)定理}
设随机变量$\eta_n (n=1, 2, \cdots)$服从参数为$n, p (0 < p < 1)$的二项分布,则对于任意$x$,有
\begin{equation}
\lim_{n\rightarrow \infty} P\left\{\frac{\eta_n - n p}{\sqrt{np(1-p)}} \leqslant x \right\} =  \int_{-\infty}^x \frac{1}{\sqrt{2\pi}} e^{-t^2/2} \dif t = \Phi(x)
\end{equation}






































\end{document}