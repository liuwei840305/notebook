\documentclass[12pt,a4paper]{article}
%\usepackage{fontspec, xunicode, xltxtra}  
%\setmainfont{Hiragino Sans GB}  
\usepackage{xeCJK}
%\setCJKmainfont[BoldFont=STZhongsong, ItalicFont=STKaiti]{STSong}
%\setCJKsansfont[BoldFont=STHeiti]{STXihei}
%\setCJKmonofont{STFangsong}

%使用Xelatex编译

% 设置页面
%==================================================
\linespread{2} %行距
% \usepackage[top=1in,bottom=1in,left=1.25in,right=1.25in]{geometry}
% \headsep=2cm
% \textwidth=16cm \textheight=24.2cm
%==================================================

% 其它需要使用的宏包
%==================================================
\usepackage[colorlinks,linkcolor=blue,anchorcolor=red,citecolor=green,urlcolor=blue]{hyperref} 
\usepackage{tabularx}
\usepackage{authblk}         % 作者信息
\usepackage{algorithm}     % 算法排版
\usepackage{amsmath}     % 数学符号与公式
\usepackage{amsfonts}     % 数学符号与字体
\usepackage{mathrsfs}      % 花体
\usepackage[framemethod=TikZ]{mdframed}

\usepackage{graphicx} 
\usepackage{graphics}
\usepackage{color}
\usepackage{xcolor}
\usepackage{tcolorbox}
\usepackage{lipsum}
\usepackage{empheq}

\usepackage{fancyhdr}       % 设置页眉页脚
\usepackage{fancyvrb}       % 抄录环境
\usepackage{float}              % 管理浮动体
\usepackage{geometry}     % 定制页面格式
\usepackage{hyperref}       % 为PDF文档创建超链接
\usepackage{lineno}          % 生成行号
\usepackage{listings}        % 插入程序源代码
\usepackage{multicol}       % 多栏排版
%\usepackage{natbib}         % 管理文献引用
\usepackage{rotating}       % 旋转文字,图形,表格
\usepackage{subfigure}    % 排版子图形
\usepackage{titlesec}       % 改变章节标题格式
\usepackage{moresize}   % 更多字体大小
\usepackage{anysize}
\usepackage{indentfirst}  % 首段缩进
\usepackage{booktabs}   % 使用\multicolumn
\usepackage{multirow}    % 使用\multirow

\usepackage{wrapfig}
\usepackage{titlesec}     % 改变标题样式
\usepackage{enumitem}
\usepackage{aas_macros}

\usepackage{enumitem}
\usepackage{harpoon}   %矢量符号
\usepackage{extpfeil}

\newcommand{\myvec}[1]%
   {\stackrel{\raisebox{-2pt}[0pt][0pt]{\small$\rightharpoonup$}}{#1}}  %矢量符号
   
\renewcommand{\vec}[1]{\boldsymbol{#1}}
\newcommand{\me}{\mathrm{e}}
\newcommand{\mi}{\mathrm{i}}
\newcommand{\dif}{\mathrm{d}}
\newcommand{\tabincell}[2]{\begin{tabular}{@{}#1@{}}#2\end{tabular}}

\def\kpc{{\rm kpc}}
\def\km{{\rm km}}
\def\cm{{\rm cm}}
\def\TeV{{\rm TeV}}
\def\GeV{{\rm GeV}}
\def\MeV{{\rm MeV}}
\def\GV{{\rm GV}}
\def\MV{{\rm MV}}
\def\yr{{\rm yr}}
\def\s{{\rm s}}
\def\ns{{\rm ns}}
\def\GHz{{\rm GHz}}
\def\muGs{{\rm \mu Gs}}
\def\arcsec{{\rm arcsec}}
\def\K{{\rm K}}
\def\microK{\mu{\rm K}}
\def\sr{{\rm sr}}
\newcolumntype{p}{D{,}{\pm}{-1}}

\renewcommand{\figurename}{Fig.}
\renewcommand{\tablename}{Tab.}

\renewcommand{\arraystretch}{1.5}

\setlength{\parindent}{0pt}  %取消每段开头的空格

\newcounter{theo}[section]\setcounter{theo}{0}
\renewcommand{\thetheo}{\arabic{section}.\arabic{theo}}
\newenvironment{theo}[2][]{%
\refstepcounter{theo}%
\ifstrempty{#1}%
{\mdfsetup{%
frametitle={%
\tikz[baseline=(current bounding box.east),outer sep=0pt]
\node[anchor=east,rectangle,fill=blue!20]
{\strut Theorem~\thetheo};}}
}%
{\mdfsetup{%
frametitle={%
\tikz[baseline=(current bounding box.east),outer sep=0pt]
\node[anchor=east,rectangle,fill=blue!20]
{\strut Theorem~\thetheo:~#1};}}%
}%
\mdfsetup{innertopmargin=10pt,linecolor=blue!20,%
linewidth=2pt,topline=true,%
frametitleaboveskip=\dimexpr-\ht\strutbox\relax
}
\begin{mdframed}[]\relax%
\label{#2}}{\end{mdframed}}



\newcommand*\widefbox[1]{\fbox{\hspace{2em}#1\hspace{2em}}}

\title{多元函数微分法}
\author{}
\date{\today}
\begin{document}

\maketitle

\section{基本概念}

二元有序实数组$(x, y)$的全体,即$\vec{R}^2 = \vec{R} \times \vec{R} = \{(x, y) | x, y \in \vec{R} \}$就表示\textcolor{red}{坐标平面}。坐标平面上具有某种性质$P$的点的集合,称为\textcolor{red}{平面点集},记作
\begin{equation}
E = \{(x, y) | (x, y) \text{具有性质} P \}
\end{equation}

设$P_0(x_0, y_0)$是$xOy$平面上的一个点,$\delta$是某一正数。与点$P_0(x_0, y_0)$距离小于$\delta$的点$P(x, y)$的全体,称为点$P_0$的\textcolor{red}{$\delta$邻域},记作$U(P_0, \delta)$,即
\begin{equation*}
U(P_0, \delta) = \{P |~ |P P_0| < \delta \} ~,
\end{equation*}
即
\begin{equation*}
U(P_0, \delta) = \{(x, y) | \sqrt{(x-x_0)^2 +(y-y_0)^2 }  < \delta \}
\end{equation*}
点$P_0$的\textcolor{red}{去心$\delta$邻域},记作$\mathring{U}(P_0, \delta)$,即
\begin{equation*}
\mathring U(P_0, \delta) = \{P |~ 0 < |P P_0| < \delta \} ~.
\end{equation*}
任意一点$P\in \vec{R}^2$与任意一个点集$E \subset \vec{R}^2$之间的有三种关系:\\
\textcolor{red}{内点}:若存在点$P$的某个邻域$U(P)$,使得$U(P) \subset E$,则称$P$为$E$的内点;\\
\textcolor{red}{外点}:若存在点$P$的某个邻域$U(P)$,使得$U(P) \bigcap E = \emptyset$,则称$P$为$E$的外点;\\
\textcolor{red}{边界点}:若点$P$的任一邻域内既含有属于$E$点,又含有不属于$E$的点,则称$P$为$E$的边界点。$E$的边界点的全体,称为$E$的边界,记作$\partial E$。

$E$的内点必属于$E$,$E$的外点必不属于$E$,$E$的边界点可能属于$E$,也可能不属于$E$。

\textcolor{red}{聚点}:若对于任意给定的$\delta > 0$,点$P$的去心邻域$\mathring U(P_0, \delta)$内总有$E$中的点,则称$P$是$E$的聚点。点集$E$的聚点$P$本身,可以属于$E$,也可以不属于$E$。


\textcolor{red}{开集}:若点集$E$的点都是$E$的内点,则称$E$为开集。

\textcolor{red}{闭集}:若点集$E$的边界$\partial E \subset E$,则称$E$为闭集。

\textcolor{red}{连通集}:若点集$E$内任何两点,都可用折线联结起来,且该折线上的点都属于$E$,则称$E$为连通集。


\textcolor{red}{区域}(开区域):连通的开集称为区域或开区域。

\textcolor{red}{闭区域}:开区域连同它的边界一起构成的点集称为闭区域。

\textcolor{red}{有界集}:对于平面点集$E$,若存在某一正数$r$,使得
\begin{equation*}
E \subset U(O, r) ~,
\end{equation*}
其中$O$是坐标原点,则称$E$为有界集。

\textcolor{red}{无界集}:一个集合如果不是有界集,就称这集合为无界集。

设$n$为取定的一个正整数,用$\vec{R}^n$表示$n$元有序实数组$(x_1, x_2, \cdots, x_n)$的全体所构成的集合,即
\begin{equation*}
\vec{R}^n = \vec{R} \times \vec{R} \times \cdots \times \vec{R} = \{(x_1, x_2, \cdots, x_n) | x_i \in \vec{R}, i = 1, 2, \cdots, n \} ~.
\end{equation*}
$\vec{R}^n$中的元素$(x_1, x_2, \cdots, x_n)$也可以用单个字母$\vec{x}$表示,即$\vec{x} = (x_1, x_2, \cdots, x_n)$。当所有的$x_i(i =1, 2, \cdots, n)$都为零时,称这样的元素为$\vec{R}^n$中的零元,记为$\vec{0}$或$O$。

设$\vec{x} = (x_1, x_2, \cdots, x_n), \vec{y} = (y_1, y_2, \cdots, y_n)$为$\vec{R}^n$中任意两个元素,$\lambda \in \vec{R}$,规定
\begin{align*}
& \vec{x} + \vec{y} = (x_1 +y_1, x_2 +y_2, \cdots, x_n +y_n) ~, \\
& \lambda \vec{x} =  (\lambda x_1, \lambda  x_2, \cdots, \lambda x_n) ~.
\end{align*}
定义了线性运算的集合$\vec{R}^n$称为\textcolor{red}{$n$维空间}。

\begin{tcolorbox}[colback=green!15,colframe=green!40!black,title= 定义]
设$D$是$\vec{R}^2$的一个非空子集,称映射$f : D \rightarrow \vec{R}$为定义在$D$上的二元函数,记为
\begin{equation*}
z = f(x, y) ~, (x, y) \in D
\end{equation*}
或
\begin{equation*}
z = f(P) ~, P \in D
\end{equation*}
其中点集$D$称为该函数的\textcolor{red}{定义域},$x, y$称为\textcolor{red}{自变量},$z$称为\textcolor{red}{因变量}。
\end{tcolorbox}






\begin{tcolorbox}[colback=green!15,colframe=green!40!black,title= 定义]
设二元函数$f(P) = f(x, y)$的定义域为$D$,$P_0(x_0, y_0)$是$D$的聚点。若存在常数$A$,对于任意给定的正数$\varepsilon$,总存在正数$\delta$,使得当点$P(x, y) \in D \bigcap \mathring{U}(P_0, \delta)$时,都有
\begin{equation*}
|f(P) - A| = |f(x, y) - A | < \varepsilon ~,
\end{equation*}
成立,那么就称常数$A$为函数$f(x, y)$当$(x, y) \rightarrow (x_0, y_0)$时的极限,记作
\begin{equation*}
\lim_{(x,y) \rightarrow (x_0, y_0)} f(x, y) = A  ~~\text{或者}  ~~ f(x, y) \rightarrow A( (x, y) \rightarrow (x_0, y_0) ) ~,
\end{equation*}
也记作
\begin{equation*}
\lim_{P \rightarrow P_0} f(P) = A  ~~\text{或者}  ~~ f(P) \rightarrow A( P \rightarrow P_0 ) ~.
\end{equation*}
\end{tcolorbox}
二元函数的极限也叫做二重极限。二重极限的存在,是指$P(x, y)$以任何方式趋于$P_0(x_0, y_0)$时,$f(x, y)$都无限接近于$A$。若$P(x, y)$以某一特殊方式,如沿着一条定直线或定曲线趋于$P_0(x_0, y_0)$时,即使$f(x, y)$无限接近于某一确定值,还是不能由此断定函数极限的存在。但反过来,若当$P(x, y)$以不同方式趋于$P_0(x_0, y_0)$时,$f(x, y)$趋于不同的值,可以断定函数的极限不存在。



\section{偏导数}
\begin{tcolorbox}[colback=green!15,colframe=green!40!black,title= 定义]
设函数$z = f(x, y)$在点$(x_0, y_0)$的某一领域内有定义,当$y$固定在$y_0$而$x$在$x_0$处有增量$\delta x$时,相应的函数有增量
\begin{equation*}
f(x_0 +\Delta x, y_0) - f(x_0, y_0) ~,
\end{equation*}
若
\begin{equation*}
\lim_{\Delta x \rightarrow 0} \dfrac{f(x_0 +\Delta x, y_0) - f(x_0, y_0)}{\Delta x} ~,
\end{equation*}
存在,则称此极限为函数$z=f(x,y)$在点$(x_0, y_0)$处对$x$的偏导数,记作
\begin{equation*}
\dfrac{\partial z}{\partial x} \Bigg|_{x=x_0, y=y_0} ~, \dfrac{\partial f}{\partial x} \Bigg|_{x=x_0, y=y_0} ~, z_x \Big|_{x=x_0, y=y_0} ~, f_x (x=x_0, y=y_0) ~.
\end{equation*}
函数$z = f(x, y)$在点$(x_0, y_0)$处对$y$的偏导数定义为
\begin{equation}
\lim_{\Delta y \rightarrow 0} \dfrac{f(x_0, y_0 +\Delta y) - f(x_0, y_0)}{\Delta y} ~,
\end{equation}
记作
\begin{equation*}
\dfrac{\partial z}{\partial y} \Bigg|_{x=x_0, y=y_0} ~, \dfrac{\partial f}{\partial y} \Bigg|_{x=x_0, y=y_0} ~, z_y \Big|_{x=x_0, y=y_0} ~, f_y (x=x_0, y=y_0) ~.
\end{equation*}
\end{tcolorbox}



\begin{tcolorbox}[colback=green!15,colframe=green!40!black,title= 定理]
若函数$z = f(x, y)$的两个二阶混合偏导数$\dfrac{\partial^2 z}{\partial y \partial x}$以及$\dfrac{\partial^2 z}{\partial x \partial y}$在区域$D$内连续,那么该区域内这两个二阶混合偏导数必相等。  
\end{tcolorbox}
二阶混合偏导数在连续的条件下与求导的次序无关。























\section{全微分}

















\section{多元复合函数的求导}












\section{隐函数的求导}










\section{多元函数微分学的几何应用}











\section{方向导数与梯度}
偏导数反映的是函数沿坐标轴方向的变化率。

设$l$是$xOy$平面上以$P_0(x_0, y_0)$为始点的一条射线,$\vec{e}_l = (\cos \alpha, \cos \beta)$是与$l$同方向的单位向量。射线$l$的参数方程为
\begin{align*}
& x = x_0 + t \cos \alpha ~, (t \geqslant 0 )\\
& y = y_0 +t \cos \beta ~, 
\end{align*}
设函数$z = f(x, y)$在点$P_0(x_0, y_0)$的某个邻域$U(P_0)$内有定义,$P(x_0 +t \cos \alpha, y_0 +t \cos \beta)$为$l$上另一点,且$P \in U(P_0)$。若\textcolor{blue}{函数增量}$f(x_0 +t\cos \alpha, y_0 +t \cos \beta) - f(x_0, y_0)$与\textcolor{blue}{$P$到$P_0$的距离}$|PP_0| = t$的比值
\begin{equation*}
\dfrac{f(x_0 +t\cos \alpha, y_0 +t \cos \beta) - f(x_0, y_0)}{t}
\end{equation*}
当$P$沿着$l$趋于$P_0$(即$t \rightarrow 0^+$)时的极限存在,称此极限为函数$f(x, y)$在点$P_0$沿方向$l$的\textcolor{red}{方向导数},记作$\dfrac{\partial f}{\partial l} \Bigg|_{(x_0, y_0)}$,即
\begin{equation}
\dfrac{\partial f}{\partial l} \Bigg|_{x_0, y_0} = \lim_{t \rightarrow 0^+} \dfrac{f(x_0 +t\cos \alpha, y_0 +t \cos \beta) - f(x_0, y_0)}{t}
\end{equation}
\textcolor{orange}{方向导数$\dfrac{\partial f}{\partial l} \Bigg|_{(x_0, y_0)}$就是函数$f(x, y)$在点$P_0(x_0, y_0)$处沿方向$l$的变化率}。若函数$f(x, y)$在点$P_0(x_0, y_0)$的偏导数存在,$\vec{e}_l = \vec{i} = (1, 0)$,则
\begin{equation*}
\dfrac{\partial f}{\partial l} \Bigg|_{(x_0, y_0)} = \lim_{t \rightarrow 0^+} \dfrac{f(x_0 +t, y_0) - f(x_0, y_0)}{t} = f_x(x_0, y_0) ~,
\end{equation*}
若$\vec{e}_l = \vec{j} = (0, 1)$,则
\begin{equation*}
\dfrac{\partial f}{\partial l} \Bigg|_{(x_0, y_0)} = \lim_{t \rightarrow 0^+} \dfrac{f(x_0, y_0 +t) - f(x_0, y_0)}{t} = f_y(x_0, y_0) ~,
\end{equation*}
反之,\textcolor{yellow}{若$\vec{e}_l = \vec{i}, \dfrac{\partial z}{\partial l} \Bigg|_{(x_0, y_0)}$存在,则$\dfrac{\partial z}{\partial x} \Bigg|_{(x_0, y_0)}$未必存在。}


\begin{tcolorbox}[colback=green!15,colframe=green!40!black,title= 定理]
如果函数$f(x,y)$在点$P_0(x_0, y_0)$可微分,那么函数在该点沿任一方向$l$的方向导数存在,且有
\begin{equation}
\dfrac{\partial f}{\partial l} \Bigg|_{(x_0, y_0)} = f_x (x_0, y_0) \cos \alpha +f_y (x_0, y_0) \cos \beta ~,
\end{equation}
其中$\cos \alpha, \cos \beta$是方向$l$的方向余弦。
\end{tcolorbox}



设函数$f(x, y)$在平面区域$D$内具有一阶连续偏导数,则对于每一点$P_0(x_0, y_0) \in D$,都可定出一个向量
\begin{equation}
f_x(x_0, y_0) \vec{i} +f_y(x_0, y_0) \vec{j} ~,
\end{equation}
这向量称为函数$f(x, y)$在点$P_0(x_0, y_0)$的\textcolor{red}{梯度},记作$\textbf{grad} f(x_0, y_0)$或$\nabla f(x_0, y_0)$,即
\begin{equation}
\textbf{grad} f(x_0, y_0) = \nabla f(x_0, y_0) = f_x(x_0, y_0) \vec{i} +f_y(x_0, y_0) \vec{j} ~.
\end{equation}
其中$\nabla = \dfrac{\partial }{\partial x} \vec{i} +\dfrac{\partial }{\partial y} \vec{j} $称为(二维的)向量微分算子或Nabla算子,$\nabla f = \dfrac{\partial f}{\partial x} \vec{i} +\dfrac{\partial f}{\partial y} \vec{j}$。


若函数$f(x, y)$在点$P_0(x_0, y_0)$可微分,$\vec{e}_l = (\cos \alpha, \cos \beta)$是与方向$l$同向的单位向量,则
\begin{align}
\nonumber \color{red} \dfrac{\partial f}{\partial l} \Bigg|_{(x_0, y_0)} &= \color{red} f_x (x_0, y_0) \cos \alpha +f_y (x_0, y_0) \cos \beta \\
&= \color{red} \textbf{grad} f(x_0, y_0) \cdot \vec{e}_l = |\textbf{grad} f(x_0, y_0) | \cos \theta ~,
\end{align}
其中$\theta = (\textbf{grad} f(x_0, y_0), \vec{e}_l)$。

当$\theta = 0$,即方向$\vec{e}_l$与梯度$\textbf{grad} f(x_0, y_0)$的方向相同时,函数$f(x, y)$增加最快。函数在这个方向的方向导数达到最大值,即为梯度$\textbf{grad} f(x_0, y_0)$的模,即
\begin{equation}
\dfrac{\partial f}{\partial l} \Bigg|_{(x_0, y_0)} = |\textbf{grad} f(x_0, y_0) | ~.
\end{equation}

当$\theta =\pi$,即方向$\vec{e}_l$与梯度$\textbf{grad} f(x_0, y_0)$的方向相反时,函数$f(x, y)$减少最快。函数在这个方向的方向导数达到最小值,即
\begin{equation}
\dfrac{\partial f}{\partial l} \Bigg|_{(x_0, y_0)} = -|\textbf{grad} f(x_0, y_0) | ~.
\end{equation}


当$\theta = \dfrac{\pi}{2}$,即方向$\vec{e}_l$与梯度$\textbf{grad} f(x_0, y_0)$的方向正交时,函数的变化率为$0$,即
\begin{equation}
\dfrac{\partial f}{\partial l} \Bigg|_{(x_0, y_0)} = 0 ~.
\end{equation}




\subsection{Directional Derivatives and the Gradient Vector}
\cite{stewart2015calculus}  If $z = f(x, y)$, then the partial derivatives $f_x$ and $f_y$ are defined as
\begin{align}
& f_x(x_0, y_0) = \underset{h \rightarrow 0}\lim \dfrac{f(x_0 +h, y_0) -f(x_0, y_0)}{h} \\
& f_y(x_0, y_0) = \underset{h \rightarrow 0}\lim \dfrac{f(x_0, y_0+h) -f(x_0, y_0)}{h} 
\end{align}
and represent the rates of change of $z$ in the $x$- and $y$-directions, that is, in the directions of the unit vectors $\vec{i}$ and $\vec{j}$.

Suppose that we now wish to find the rate of change of $z$ at $(x_0, y_0$ in the direction of an arbitrary unit vector $\vec{u} = \langle a, b\rangle$. Consider the surface $S$ with the equation $z = f(x, y)$ (the graph of $f$ ) and we let $z_0 = f(x_0, y_0)$. Then the point $P(x_0, y_0, z_0$ lies on $S$. The vertical plane that passes through $P$ in the direction of $\vec{u}$ intersects $S$ in a curve $C$. The slope of the tangent line $T$ to $C$ at the point $P$ is the rate of change of $z$ in the direction of $\vec{u}$.

If $Q(x, y, z)$ is another point on $C$ and $P^\prime$, $Q^\prime$ are the projections of $P, Q$ onto the $xy$-plane, then the vector $\myvec{P^{\prime}Q^{\prime}}$ is parallel to $\vec{u}$ and so
\begin{equation*}
\myvec{P^{\prime}Q^{\prime}} = h\vec{u} = \langle ha, hb\rangle
\end{equation*}
for some scalar $h$. Therefore $x-x_0 = ha$, $y-y_0 = hb$, so $x = x_0 +ha$, $y =y_0 +hb$, and 
\begin{equation*}
\dfrac{\Delta z}{h} = \dfrac{z -z_0}{h} = \dfrac{f(x_0 +h a, y_0 +h b) -f(x_0, y_0)}{h}
\end{equation*}
If we take the limit as $h \rightarrow 0$, we obtain the rate of change of $z$ (with respect to distance) in the direction of $\vec{u}$, which is called the directional derivative of $f$ in the direction of $\vec{u}$.


\begin{tcolorbox}[colback=green!5,colframe=green!40!black,title= Definition]
The \textcolor{red}{directional derivative} of $f$ at $(x_0, y_0)$ in the direction of a unit vector $\vec{u} = \langle a, b\rangle$ is
\begin{equation*}
D_{\vec{u}} f(x_0, y_0) = \underset{h \rightarrow 0}\lim \dfrac{f(x_0 +h a, y_0 +h b) -f(x_0, y_0)}{h}
\end{equation*}
if this limit exists.
\end{tcolorbox}


\begin{tcolorbox}[colback=green!5,colframe=green!40!black,title= Theorem]
If $f$ is a differentiable function of $x$ and $y$, then $f$ has a directional derivative in the direction of any unit vector $\vec{u} = \langle a, b\rangle$ and
\begin{equation*}
D_{\vec{u}} f(x, y) = f_x(x, y) a +f_y(x, y) b
\end{equation*}
if this limit exists.
\end{tcolorbox}
If the unit vector $\vec{u}$ makes an angle $\theta$ with the positive $x$-axis, then we can write $\vec{u} = \langle \cos \theta, \sin \theta$ and it becomes
\begin{equation}
D_{\vec{u}} f(x, y) = f_x(x, y) \cos \theta +f_y(x, y) \sin \theta ~.
\end{equation}


The directional derivative of a differentiable function can be written as the dot product of two vectors:
\begin{align*}
D_{\vec{u}} f(x, y) &= f_x(x, y) a +f_y(x, y) b \\
&= \langle f_x(x,y), f_y(x,y)\rangle \cdot \langle a, b\rangle \\
&=  \langle f_x(x,y), f_y(x,y)\rangle \cdot \vec{u} ~.
\end{align*}

\begin{tcolorbox}[colback=green!5,colframe=green!40!black,title= Definition]
If $f$ is a function of two variables $x$ and $y$, then the \textcolor{red}{gradient} of $f$ is the vector function $\nabla f$ defined by
\begin{equation*}
\nabla f = \langle f_x(x,y), f_y(x,y)\rangle = \dfrac{\partial f}{\partial x} \vec{i} +\dfrac{\partial f}{\partial y} \vec{j} ~.
\end{equation*}
\end{tcolorbox}

\begin{equation}
D_{\vec{u}} f(x, y) = \nabla f (x,y) \cdot \vec{u} ~.
\end{equation}
This expresses the directional derivative in the direction of a unit vector $\vec{u}$ as the scalar projection of the gradient vector onto $\vec{u}$.


\begin{tcolorbox}[colback=green!5,colframe=green!40!black,title= Definition]
The directional derivative of $f$ at $(x_0, y_0, z_0)$ in the direction of a unit vector $\vec{u} = \langle a, b, c\rangle$ is
\begin{equation*}
D_{\vec{u}} f(x_0, y_0, z_0) =  \underset{h \rightarrow 0}\lim \dfrac{f(x_0 +h a, y_0 +h b, z_0 +hc) -f(x_0, y_0, z_0)}{h} ~.
\end{equation*}
if this limit exists.
\end{tcolorbox}

\begin{equation}
D_{\vec{u}} f(\vec{x}_0) = \underset{h \rightarrow 0}\lim \dfrac{f(\vec{x}_0 +h \vec{u}) -f(\vec{x}_0)}{h} ~.
\end{equation}


\begin{tcolorbox}[colback=green!5,colframe=green!40!black,title= Theorem]
Suppose $f$ is a differentiable function of two or three variables. The maximum value of the directional derivative $D_{\vec{u}} f(\vec{x})$ is $|\nabla f (\vec{x})|$ and it occurs when $\vec{u}$ has the same direction as the gradient vector $\nabla f(\vec{x})$.
\end{tcolorbox}

\section{多元函数的极值}

\subsection{多元函数的极值及最大值、最小值}







\subsection{条件极值 ~~ 拉格朗日乘数法}
对于函数的自变量,除了限制在函数的定义域内外,无其他条件,称为\textcolor{blue}{无条件极值}。

对自变量有附加条件的极值称为\textcolor{blue}{条件极值}。
















\section{二元函数的泰勒公式}
















\section{最小二乘法}



















































%%%%%%%%%%%%%%%%%%%%%%%%%%%%%%%%%%%%%%%%%%%%%%%%%%%%%%%%%%%%%%%%%%%%%%
\bibliographystyle{unsrt_update}
\bibliography{ref}
%%%%%%%%%%%%%%%%%%%%%%%%%%%%%%%%%%%%%%%%%%%%%%%%%%%%%%%%%%%%%%%%%%%%%%

\end{document}