\documentclass[12pt,a4paper]{article}
%\usepackage{fontspec, xunicode, xltxtra}  
%\setmainfont{Hiragino Sans GB}  
\usepackage{xeCJK}
%\setCJKmainfont[BoldFont=STZhongsong, ItalicFont=STKaiti]{STSong}
%\setCJKsansfont[BoldFont=STHeiti]{STXihei}
%\setCJKmonofont{STFangsong}

%使用Xelatex编译

% 设置页面
%==================================================
\linespread{2} %行距
% \usepackage[top=1in,bottom=1in,left=1.25in,right=1.25in]{geometry}
% \headsep=2cm
% \textwidth=16cm \textheight=24.2cm
%==================================================

% 其它需要使用的宏包
%==================================================
\usepackage[colorlinks,linkcolor=blue,anchorcolor=red,citecolor=green,urlcolor=blue]{hyperref} 
\usepackage{tabularx}
\usepackage{authblk}         % 作者信息
\usepackage{algorithm}     % 算法排版
\usepackage{amsmath}     % 数学符号与公式
\usepackage{amsfonts}     % 数学符号与字体
\usepackage{amssymb}
\usepackage{amsthm}
\usepackage[framemethod=TikZ]{mdframed}
\usepackage{mathrsfs}      % 花体

\usepackage{graphics}
\usepackage{graphicx} 
\usepackage{color}
\usepackage{xcolor}
\usepackage{fancyhdr}       % 设置页眉页脚
\usepackage{fancyvrb}       % 抄录环境
\usepackage{float}              % 管理浮动体
\usepackage{geometry}     % 定制页面格式
\usepackage{hyperref}       % 为PDF文档创建超链接
\usepackage{lineno}          % 生成行号
\usepackage{listings}        % 插入程序源代码
\usepackage{multicol}       % 多栏排版
\usepackage{natbib}         % 管理文献引用
\usepackage{rotating}       % 旋转文字,图形,表格
\usepackage{subfigure}    % 排版子图形
\usepackage{titlesec}       % 改变章节标题格式
\usepackage{moresize}   % 更多字体大小
\usepackage{anysize}
\usepackage{indentfirst}  % 首段缩进
\usepackage{booktabs}   % 使用\multicolumn
\usepackage{multirow}    % 使用\multirow
\usepackage{wrapfig}
\usepackage{titlesec}     % 改变标题样式
\usepackage{enumitem}
\usepackage{harpoon}   %矢量符号
\usepackage{tcolorbox}

\newcommand{\myvec}[1]%
   {\stackrel{\raisebox{-2pt}[0pt][0pt]{\small$\rightharpoonup$}}{#1}}  %矢量符号
\renewcommand{\vec}[1]{\boldsymbol{#1}}
\newcommand{\me}{\mathrm{e}}
\newcommand{\mi}{\mathrm{i}}
\newcommand{\dif}{\mathrm{d}}
\newcommand{\tabincell}[2]{\begin{tabular}{@{}#1@{}}#2\end{tabular}}

\def\kpc{{\rm kpc}}
\def\km{{\rm km}}
\def\cm{{\rm cm}}
\def\TeV{{\rm TeV}}
\def\GeV{{\rm GeV}}
\def\MeV{{\rm MeV}}
\def\GV{{\rm GV}}
\def\MV{{\rm MV}}
\def\yr{{\rm yr}}
\def\s{{\rm s}}
\def\ns{{\rm ns}}
\def\GHz{{\rm GHz}}
\def\muGs{{\rm \mu Gs}}
\def\arcsec{{\rm arcsec}}
\def\K{{\rm K}}
\def\microK{\mu{\rm K}}
\def\sr{{\rm sr}}
\newcolumntype{p}{D{,}{\pm}{-1}}

\renewcommand{\figurename}{Fig.}
\renewcommand{\tablename}{Tab.}

\renewcommand{\arraystretch}{1.5}

\setlength{\parindent}{0pt}  %取消每段开头的空格

\newtheorem*{thm}{The Theorem}

\newcounter{theo}[section]\setcounter{theo}{0}
\renewcommand{\thetheo}{\arabic{section}.\arabic{theo}}
\newenvironment{theo}[2][]{%
\refstepcounter{theo}%
\ifstrempty{#1}%
{\mdfsetup{%
frametitle={%
\tikz[baseline=(current bounding box.east),outer sep=0pt]
\node[anchor=east,rectangle,fill=blue!20]
{\strut Theorem~\thetheo};}}
}%
{\mdfsetup{%
frametitle={%
\tikz[baseline=(current bounding box.east),outer sep=0pt]
\node[anchor=east,rectangle,fill=blue!20]
{\strut Theorem~\thetheo:~#1};}}%
}%
\mdfsetup{innertopmargin=10pt,linecolor=blue!20,%
linewidth=2pt,topline=true,%
frametitleaboveskip=\dimexpr-\ht\strutbox\relax
}
\begin{mdframed}[]\relax%
\label{#2}}{\end{mdframed}}


\title{极限}
\author{}
\date{\today}
\begin{document}

\maketitle


\section{数列}
数列最一般的表示
\begin{equation}
a_1, a_2, \cdots, a_n, \cdots
\end{equation}
$a_n$是数列的第$n$项,称为数列的\textcolor{red}{通项}。数列常记为$\{a_n \}$。

收敛数列:当$n$变得越来越大时,项$a_n$就越来越接近某一个常数$a$。

设$\{a_n \}$是一个数列,$a$是一个实数。若对于任意给定的$\epsilon > 0$,存在一个$N \in \mathcal{N}^*$,使得凡是$n > N$时,都有
\begin{equation}
|a_n - a| < \epsilon
\end{equation}
就说数列$\{a_n \}$当$n$趋向无穷大时,以$a$为\textcolor{red}{极限},记成
\begin{equation}
\underset{n \rightarrow \infty}\lim a_n = a ~,
\end{equation}
简记为$a_n \rightarrow a(n \rightarrow \infty)$。\textcolor{red}{数列$\{a_n \}$收敛于$a$}。存在极限的数列称为\textcolor{red}{收敛数列}。不收敛的数列称为\textcolor{red}{发散数列}。





\section{收敛数列的性质}
关于$a$对称的开区间$(a-\epsilon, a+\epsilon)$为$a$的\textcolor{red}{$\epsilon-$领域}。


\textcolor{red}{数列$\{a_n \}$当$n\rightarrow \infty$时收敛于实数$a$}指:对任意的$\epsilon > 0$,总存在$N \in \mathcal{N}^*$,使得数列中除有限多项$a_1, a_2, \cdots, a_N$可能是例外,其他的项均落在$a$的$\epsilon-$领域中。

\begin{theo}[]{}
若数列$a_n$收敛,则它只有一个极限。即收敛数列的极限是唯一的。
\end{theo}

设$\{a_n \}$是一个数列。若存在一个实数$A$,使得$a_n \leqslant A$对一切$n \in \mathcal{N}^*$成立,则称$\{a_n \}$是\textcolor{red}{有上界}的,$A$是这数列的一个上界。

有下界的数列

若数列$\{a_n \}$既有上界,又有下界,则称它是一个\textcolor{red}{有界数列}。

\begin{theo}[]{}
收敛数列必是有界的。
\end{theo}

设$\{a_n \}$是一个数列,$k_i \in \mathcal{N}^*(i = 1, 2, 3 \cdots)$且满足$k_1 < k_2 < k_3 \cdots$,那么数列$\{a_{k_n} \}$叫做$\{a_n \}$的一个\textcolor{red}{子列}。

\begin{theo}[]{}
设收敛数列$\{a_n \}$的极限是$a$,那么$\{a_n \}$的任何子列都收敛到$a$。
\end{theo}

\begin{theo}[极限的四则运算]{}
设$\{a_n \}$和$\{b_n \}$都是收敛数列,则$\{a_n \pm b_n \}$,$\{a_nb_n\}$也是收敛数列。若
$\underset{n \rightarrow \infty}\lim b_n \neq 0$,
则$\left\{\dfrac{a_n}{b_n} \right\}$也收敛,且 \\
1. $\underset{n \rightarrow \infty}\lim (a_n \pm b_n) = \underset{n \rightarrow \infty}\lim a_n \pm \underset{n \rightarrow \infty}\lim b_n$;\\
2. $\underset{n \rightarrow \infty}\lim a_n b_n = \underset{n \rightarrow \infty}\lim a_n \cdot \underset{n \rightarrow \infty}\lim b_n$;\\
特别若$c$是常数,则$\underset{n \rightarrow \infty}\lim c a_n = c \underset{n \rightarrow \infty}\lim a_n$;\\
3. $\underset{n \rightarrow \infty}\lim \dfrac{a_n}{b_n} = \dfrac{\underset{n \rightarrow \infty}\lim a_n}{\underset{n \rightarrow \infty}\lim b_n}$ (其中$\underset{n \rightarrow \infty}\lim b_n \neq 0$)。
\end{theo}

若收敛数列$\{a_n \}$的极限等于$0$,则这个数列称为\textcolor{red}{无穷小数列},简称无穷小。

\begin{theo}[]{}
1. $\{a_n \}$为无穷小的充分必要条件是$\{|a_n| \}$是无穷小;\\
2. 两个无穷小之和(或差)仍是无穷小;\\
3. 设$\{a_n \}$为无穷小,$\{c_n \}$为有界数列,那么$\{c_n a_n \}$也是无穷小;\\
4. 设$0 \leqslant a_n \leqslant b_n$,$n \in \mathcal{N}^*$,若$\{b_n\}$为无穷小,那么$\{a_n\}$也是无穷小;\\
5. $\underset{n \rightarrow \infty}\lim a_n = a$的充分必要条件是$|a_n -a|$是无穷小。
\end{theo}

\begin{theo}[夹逼原理]{}
设$a_n \leqslant b_n \leqslant c_n ~, ~n \in \mathcal{N}^* ~,$且$\underset{n \rightarrow \infty}\lim a_n = \underset{n \rightarrow \infty}\lim c_n = a ~,$那么$\underset{n \rightarrow \infty}\lim b_n = a$。
\end{theo}

\begin{theo}[]{}
1. 设$\underset{n \rightarrow \infty}\lim a_n = a$,$\alpha, \beta$满足$\alpha < a < \beta$,当$n$充分大时有$a_n > \alpha$;同样,当$n$充分大时有$a_n < \beta$;\\
2. 设$\underset{n \rightarrow \infty}\lim a_n = a$,$\underset{n \rightarrow \infty}\lim b_n = b$,且$a < b$,当$n$充分大时一定有$a_n < b_n$;\\
3. 设$\underset{n \rightarrow \infty}\lim a_n = a$,$\underset{n \rightarrow \infty}\lim b_n = b$,且当$n$充分大时$a_n \leqslant b_n$,有$a \leqslant b$。
\end{theo}


\section{数列极限概念的推广}
若数列$\{a_n \}$适合条件:对任何正数$A$,都存在$N \in \mathcal{N}^*$,使得凡是$n > N$时,都有$a_n > A$,称数列$\{a_n \}$趋向于$+\infty$。记作
\begin{equation}
\underset{n \rightarrow \infty}\lim a_n = +\infty
\end{equation}
若对于任何正数$A$,都存在$N \in \mathcal{N}^*$使得凡是$n > N$时有$a_n < -A$,称数列$\{a_n \}$趋向于$-\infty$,记作
\begin{equation}
\underset{n \rightarrow \infty}\lim a_n = -\infty
\end{equation}






\section{单调数列}


\section{自然对数底\text{e}}


\section{基本列和收敛定理}





\section{上确界和下确界}



\section{有限覆盖定理}



\section{上极限和下极限}


\section{Stolz定理}




\section{数列极限的应用}



























































\end{document}