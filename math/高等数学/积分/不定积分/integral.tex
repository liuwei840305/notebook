\documentclass[12pt,a4paper]{article}
%\usepackage{fontspec, xunicode, xltxtra}  
%\setmainfont{Hiragino Sans GB}  
\usepackage{xeCJK}
%\setCJKmainfont[BoldFont=STZhongsong, ItalicFont=STKaiti]{STSong}
%\setCJKsansfont[BoldFont=STHeiti]{STXihei}
%\setCJKmonofont{STFangsong}

%使用Xelatex编译

% 设置页面
%==================================================
\linespread{2} %行距
% \usepackage[top=1in,bottom=1in,left=1.25in,right=1.25in]{geometry}
% \headsep=2cm
% \textwidth=16cm \textheight=24.2cm
%==================================================

% 其它需要使用的宏包
%==================================================
\usepackage[colorlinks,linkcolor=blue,anchorcolor=red,citecolor=green,urlcolor=blue]{hyperref} 
\usepackage{tabularx}
\usepackage{authblk}         % 作者信息
\usepackage{algorithm}     % 算法排版
\usepackage{amsmath}     % 数学符号与公式
\usepackage{amsfonts}     % 数学符号与字体
\usepackage{mathrsfs}      % 花体
\usepackage{graphics}
\usepackage{color}
\usepackage{fancyhdr}       % 设置页眉页脚
\usepackage{fancyvrb}       % 抄录环境
\usepackage{float}              % 管理浮动体
\usepackage{geometry}     % 定制页面格式
\usepackage{hyperref}       % 为PDF文档创建超链接
\usepackage{lineno}          % 生成行号
\usepackage{listings}        % 插入程序源代码
\usepackage{multicol}       % 多栏排版
\usepackage{natbib}         % 管理文献引用
\usepackage{rotating}       % 旋转文字,图形,表格
\usepackage{subfigure}    % 排版子图形
\usepackage{titlesec}       % 改变章节标题格式
\usepackage{moresize}   % 更多字体大小
\usepackage{anysize}
\usepackage{indentfirst}  % 首段缩进
\usepackage{booktabs}   % 使用\multicolumn
\usepackage{multirow}    % 使用\multirow
\usepackage{graphicx} 
\usepackage{wrapfig}
\usepackage{xcolor}
\usepackage{titlesec}     % 改变标题样式
\usepackage{enumitem}
\usepackage{harpoon}   %矢量符号

\newcommand{\myvec}[1]%
   {\stackrel{\raisebox{-2pt}[0pt][0pt]{\small$\rightharpoonup$}}{#1}}  %矢量符号
\renewcommand{\vec}[1]{\boldsymbol{#1}}
\newcommand{\me}{\mathrm{e}}
\newcommand{\mi}{\mathrm{i}}
\newcommand{\dif}{\mathrm{d}}
\newcommand{\tabincell}[2]{\begin{tabular}{@{}#1@{}}#2\end{tabular}}

\def\kpc{{\rm kpc}}
\def\km{{\rm km}}
\def\cm{{\rm cm}}
\def\TeV{{\rm TeV}}
\def\GeV{{\rm GeV}}
\def\MeV{{\rm MeV}}
\def\GV{{\rm GV}}
\def\MV{{\rm MV}}
\def\yr{{\rm yr}}
\def\s{{\rm s}}
\def\ns{{\rm ns}}
\def\GHz{{\rm GHz}}
\def\muGs{{\rm \mu Gs}}
\def\arcsec{{\rm arcsec}}
\def\K{{\rm K}}
\def\microK{\mu{\rm K}}
\def\sr{{\rm sr}}
\newcolumntype{p}{D{,}{\pm}{-1}}

\renewcommand{\figurename}{Fig.}
\renewcommand{\tablename}{Tab.}

\renewcommand{\arraystretch}{1.5}

\setlength{\parindent}{0pt}  %取消每段开头的空格

\title{不定积分}
\author{}
\date{\today}
\begin{document}

\maketitle

\section{原函数(不定积分)的概念}

若在给定的整个区间上,$f(x)$是函数$F(x)$的导数,或$f(x)\dif x$是$F(x)$的微分
\begin{equation}
F^{\prime}(x) = f(x), ~\dif F(x) = f(x) \dif x
\end{equation}
那么,在所给定的区间上,函数$F(x)$叫做$f(x)$的\textcolor{red}{原函数}或$f(x)$的积分。

\section{有理式的积分}

\subsection{某些含有根式的函数的积分}

寻求替换$t = \omega(x)$(其中$\omega$本身能用初等函数表示出来),这替换会把被积表达式化成有理函数的形状,这个方法叫做\textcolor{red}{有理化被积表达式法}。

\subsubsection{形状为$R\left(x, \sqrt[\uproot{10}\leftroot{-2}m]{\dfrac{\alpha x+\beta}{\gamma x+\delta}}  \right)$的积分}
考虑形如
\begin{equation}
\int R\left(x, \sqrt[\uproot{16}\leftroot{-2}m]{\frac{\alpha x+\beta}{\gamma x+\delta}}  \right)
\end{equation}
的积分,其中$R$表示两个自变量的有理函数,$m$是\textcolor{red}{自然数},而$\alpha, \beta, \gamma, \delta$是常数。

令
\begin{equation}
t = \omega(x) = \sqrt[\uproot{16}\leftroot{-2}m]{\frac{\alpha x+\beta}{\gamma x+\delta}} , ~t^m = \frac{\alpha x+\beta}{\gamma x+\delta} , ~x = \varphi(t) = \frac{\delta t^m -\beta}{\alpha -\gamma t^m}
\end{equation}
则积分变成
\begin{equation}
\int R(\varphi(t), t) \varphi^{\prime}(t) \dif t
\end{equation}

形如
\begin{equation}
\int R\left(x, \left(\frac{\alpha x+\beta}{\gamma x+\delta} \right)^r, \left(\frac{\alpha x+\beta}{\gamma x+\delta} \right)^s, \cdots \right) \dif x
\end{equation}
其中指数$r, s, \cdots$都是有理数。


\subsubsection{二项式微分的积分}
形如
\begin{equation}
x^m (a+bx^n)^p \dif x
\end{equation}
的微分式叫做\textcolor{red}{二项式微分},其中$a, b$是任何常数,而指数$m, n, p$是\textcolor{red}{有理数}。

若用$\lambda$表示分数$m$及$n$的分母的最小公倍数,就有形如$R(\sqrt[\uproot{6}\leftroot{-2}\lambda]{x})\dif x$的表达式。作替换$t = \sqrt[\uproot{6}\leftroot{-2}\lambda]{x}$;

令$z = x^n$,
\begin{equation}
x^m (a+bx^n)^p \dif x = \frac{1}{n} (a+bz)^p z^{\frac{m+1}{n} -1} \dif z ,
\end{equation}
令
\begin{equation}
\frac{m+1}{n} -1 = q ,
\end{equation}
即有
\begin{equation}
\int x^m (a+bx^n)^p \dif x = \frac{1}{n} \int (a+bz)^p z^{q} \dif z ,
\label{formu_1}
\end{equation}


\begin{equation}
t = \sqrt[\uproot{16}\leftroot{-2}\nu]{\frac{a+bz}{z}}
\end{equation}

若$p, q, p+q$中有一个是整数,或者$p, (m+1)/n, (m+1)/n +p$中有一个是整数,等式(\ref{formu_1})的两个积分都可按有限形状表示出来。

在有限形状中二项微分没有其他可积的情形。

























































































































\end{document}