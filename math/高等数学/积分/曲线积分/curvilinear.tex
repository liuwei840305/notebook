\documentclass[12pt,a4paper]{article}
%\usepackage{fontspec, xunicode, xltxtra}  
%\setmainfont{Hiragino Sans GB}  
\usepackage{xeCJK}
%\setCJKmainfont[BoldFont=STZhongsong, ItalicFont=STKaiti]{STSong}
%\setCJKsansfont[BoldFont=STHeiti]{STXihei}
%\setCJKmonofont{STFangsong}

%使用Xelatex编译

% 设置页面
%==================================================
\linespread{2} %行距
% \usepackage[top=1in,bottom=1in,left=1.25in,right=1.25in]{geometry}
% \headsep=2cm
% \textwidth=16cm \textheight=24.2cm
%==================================================

% 其它需要使用的宏包
%==================================================
\usepackage[colorlinks,linkcolor=blue,anchorcolor=red,citecolor=green,urlcolor=blue]{hyperref} 
\usepackage{tabularx}
\usepackage{authblk}         % 作者信息
\usepackage{algorithm}     % 算法排版
\usepackage{amsmath}     % 数学符号与公式
\usepackage{amsfonts}     % 数学符号与字体
\usepackage{mathrsfs}      % 花体
\usepackage[framemethod=TikZ]{mdframed}
\usepackage{amssymb}
\usepackage{yhmath}


\usepackage{graphicx} 
\usepackage{graphics}
\usepackage{color}
\usepackage{xcolor}
\usepackage{tcolorbox}
\usepackage{lipsum}
\usepackage{empheq}

\usepackage{fancyhdr}       % 设置页眉页脚
\usepackage{fancyvrb}       % 抄录环境
\usepackage{float}              % 管理浮动体
\usepackage{geometry}     % 定制页面格式
\usepackage{hyperref}       % 为PDF文档创建超链接
\usepackage{lineno}          % 生成行号
\usepackage{listings}        % 插入程序源代码
\usepackage{multicol}       % 多栏排版
%\usepackage{natbib}         % 管理文献引用
\usepackage{rotating}       % 旋转文字,图形,表格
\usepackage{subfigure}    % 排版子图形
\usepackage{titlesec}       % 改变章节标题格式
\usepackage{moresize}   % 更多字体大小
\usepackage{anysize}
\usepackage{indentfirst}  % 首段缩进
\usepackage{booktabs}   % 使用\multicolumn
\usepackage{multirow}    % 使用\multirow

\usepackage{wrapfig}
\usepackage{titlesec}     % 改变标题样式
\usepackage{enumitem}
\usepackage{aas_macros}
\usepackage{harpoon}   %矢量符号
\usepackage{tcolorbox}


\newcommand{\myvec}[1]%
   {\stackrel{\raisebox{-2pt}[0pt][0pt]{\small$\rightharpoonup$}}{#1}}  %矢量符号
\renewcommand{\vec}[1]{\boldsymbol{#1}}
\newcommand{\me}{\mathrm{e}}
\newcommand{\mi}{\mathrm{i}}
\newcommand{\dif}{\mathrm{d}}
\newcommand{\tabincell}[2]{\begin{tabular}{@{}#1@{}}#2\end{tabular}}

\def\kpc{{\rm kpc}}
\def\km{{\rm km}}
\def\cm{{\rm cm}}
\def\TeV{{\rm TeV}}
\def\GeV{{\rm GeV}}
\def\MeV{{\rm MeV}}
\def\GV{{\rm GV}}
\def\MV{{\rm MV}}
\def\yr{{\rm yr}}
\def\s{{\rm s}}
\def\ns{{\rm ns}}
\def\GHz{{\rm GHz}}
\def\muGs{{\rm \mu Gs}}
\def\arcsec{{\rm arcsec}}
\def\K{{\rm K}}
\def\microK{\mu{\rm K}}
\def\sr{{\rm sr}}
\newcolumntype{p}{D{,}{\pm}{-1}}

\renewcommand{\figurename}{Fig.}
\renewcommand{\tablename}{Tab.}

\renewcommand{\arraystretch}{1.5}

\setlength{\parindent}{0pt}  %取消每段开头的空格


\newcounter{theo}[section]\setcounter{theo}{0}
\renewcommand{\thetheo}{\arabic{section}.\arabic{theo}}
\newenvironment{theo}[2][]{%
\refstepcounter{theo}%
\ifstrempty{#1}%
{\mdfsetup{%
frametitle={%
\tikz[baseline=(current bounding box.east),outer sep=0pt]
\node[anchor=east,rectangle,fill=blue!20]
{\strut Theorem~\thetheo};}}
}%
{\mdfsetup{%
frametitle={%
\tikz[baseline=(current bounding box.east),outer sep=0pt]
\node[anchor=east,rectangle,fill=blue!20]
{\strut Theorem~\thetheo:~#1};}}%
}%
\mdfsetup{innertopmargin=10pt,linecolor=blue!20,%
linewidth=2pt,topline=true,%
frametitleaboveskip=\dimexpr-\ht\strutbox\relax
}
\begin{mdframed}[]\relax%
\label{#2}}{\end{mdframed}}

\newcommand*\widefbox[1]{\fbox{\hspace{2em}#1\hspace{2em}}}


\title{曲线积分}
\author{}
\date{\today}
\begin{document}

\maketitle


\section{第一型曲线积分}
\begin{tcolorbox}[colback=green!5,colframe=green!40!black,title= 定义]
设$D \subset \mathbf R^3$是一个区域,函数$f: D \rightarrow \mathbf R$。可求长曲线$\Gamma \subset D$,其两个端点分别记为$A$和$B$。在$\Gamma$上依次取一列点$\{p_i : i = 0, 1, \cdots, n \}$,使得$p_0 = A$,$ p_n = B$。称$\wideparen{p_{i-1}p_i}$为$\Gamma$的第$i$段曲线,令$\Delta s_i = s(\wideparen{p_{i-1}p_i})$,即$\Gamma$的第$i$段曲线的弧长。在$\wideparen{p_{i-1}p_i}$上任取一点$\xi_i(i = 1, 2, \cdots, n)$,若极限
\begin{equation}
\underset{\text{max} \Delta s_i \rightarrow 0}\lim \sum_{i=1}^n f(\xi_i) \Delta s_i ~,
\end{equation}
为一个有限数,且其值不依赖于点$\xi_i$在$\wideparen{p_{i-1}p_i}$上的选择,该极限值记为
\begin{equation*}
\int_\Gamma f(p) \dif s ~\text{或者} ~\int_\Gamma f(x, y, z) \dif s
\end{equation*}
称为函数$f$在$\Gamma$上的第一型曲线积分。
\end{tcolorbox}
若$f(p) = 1$对$p \in \Gamma$成立,
\begin{equation*}
\int_\Gamma \dif s = s(\Gamma) ~,
\end{equation*}
即曲线$\Gamma$的弧长。



\begin{tcolorbox}[colback=green!5,colframe=green!40!black,title= 定理]
设区域$D \subset \mathbf R^3$,光滑曲线$\Gamma \subset D$,函数$f: D \rightarrow \mathbf R$连续。设$\Gamma$有向量参数表示$\vec{r} = \vec{r}(t), t \in [\alpha, \beta]$,
\begin{equation}
\int_\Gamma f \dif s = \int_\alpha^\beta f\circ \vec{r}(t) ||\vec{r}^\prime(t) || \dif t~.
\end{equation}
\end{tcolorbox}


\begin{tcolorbox}[colback=green!5,colframe=green!40!black,title= 推论]
设平面曲线$\Gamma$有显式表达$y = \varphi(x), x \in [a, b]$,其中$\varphi$在$[a, b]$上连续,那么
\begin{equation}
\int_\Gamma f \dif s = \int_a^b f(x, \varphi(x)) \sqrt{1 +(\varphi^\prime(x) )^2} \dif x~.
\end{equation}
\end{tcolorbox}
1. 求出$\Gamma$的一个向量参数方程$\vec{r} = \vec{r}(t)$,\\
2. 计算弧元$\dif s = || \vec{r}^\prime(t) || \dif t$,\\
3. 计算定积分$\int_\alpha^\beta f\circ \vec{r}(t) ||\vec{r}^\prime(t) || \dif t$。

\section{第二型曲线积分}
设区域$D \subset \mathbf R^3$,在$D$上定义了一个向量值函数$\vec{F} = \vec{F}(\vec{p}), \vec{p} \in D$。$\vec{F}$是在$D$上定义的一个\textcolor{red}{向量场}。


\begin{tcolorbox}[colback=green!5,colframe=green!40!black,title= 定义]
设$D \subset \mathbf R^3$是一个区域,映射$\vec{F}: D \rightarrow \mathbf R^3$。可求长的有向曲线$\Gamma \subset D$,其起点记为$A$,终点记为$B$。在$\Gamma$上依从$A$到$B$的方向顺次取一列点$\{\vec{p}_i : i = 0, 1, \cdots, n \}$,使得$\vec{p}_0 = A$,$\vec{p}_n = B$。置$\Delta \vec{p}_i = \vec{p}_i -\vec{p}_{i-1}, (i = 1, 2, \cdots, n)$。若对于在$\Gamma$的弧段$\wideparen{\vec{p}_{i-1}\vec{p}_i}$上任取的点$\vec{\xi}_i$,极限
\begin{equation}
\underset{\text{max} ||\Delta p_i || \rightarrow 0}\lim \sum_{i=1}^n \vec{F}(\vec{\xi}_i) \cdot \Delta \vec{p}_i ~,
\end{equation}
为一确定的有限数,记为
\begin{equation*}
\int_\Gamma \vec{F}(\vec{p})\cdot \dif \vec{p} ~,
\end{equation*}
称它是向量函数$\vec{F}$沿有向曲线$\Gamma$上的第二型曲线积分。
\end{tcolorbox}


\begin{tcolorbox}[colback=green!5,colframe=green!40!black,title= 定理]
设区域$D \subset \mathbf R^3$,连续映射$\vec{F}: D \rightarrow \mathbf R^3$。设$\Gamma \subset D$是一条有向光滑曲线,它具有参数向量方程$\vec{r} = \vec{r}(t), \alpha \leqslant t \leqslant \beta$,且参数$t$的增加对应着$\Gamma$的定向,则
\begin{equation*}
\int_\Gamma \vec{F}(\vec{p})\cdot \dif \vec{p} = \int_\alpha^\beta \vec{F}\circ \vec{r}(t) \cdot \vec{r}^\prime(t) \dif t ~.
\end{equation*}
\end{tcolorbox}









\section{Green公式}



\section{等周问题}



\section{有界变差函数}




































\end{document}