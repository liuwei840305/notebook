\documentclass[12pt,a4paper]{article}
%\usepackage{fontspec, xunicode, xltxtra}
%\setmainfont{Hiragino Sans GB}
%\usepackage{xeCJK}
%\setCJKmainfont[BoldFont=STZhongsong, ItalicFont=STKaiti]{STSong}
%\setCJKsansfont[BoldFont=STHeiti]{STXihei}
%\setCJKmonofont{STFangsong}

%使用Xelatex编译

% 设置页面
%==================================================
\linespread{2} %行距
% \usepackage[top=1in,bottom=1in,left=1.25in,right=1.25in]{geometry}
% \headsep=2cm
% \textwidth=16cm \textheight=24.2cm
%==================================================

% 其它需要使用的宏包
%==================================================
\usepackage[colorlinks,linkcolor=blue,anchorcolor=red,citecolor=green,urlcolor=blue]{hyperref} 
\usepackage{tabularx}
\usepackage{authblk}         % 作者信息
\usepackage{algorithm}     % 算法排版
\usepackage{amsmath}     % 数学符号与公式
\usepackage{amsfonts}     % 数学符号与字体
\usepackage{mathrsfs}      % 花体
\usepackage[framemethod=TikZ]{mdframed}
\usepackage{amssymb}

\usepackage{graphicx} 
\usepackage{graphics}
\usepackage{color}
\usepackage{xcolor}
\usepackage{tcolorbox}
\usepackage{lipsum}
\usepackage{empheq}

\usepackage{fancyhdr}       % 设置页眉页脚
\usepackage{fancyvrb}       % 抄录环境
\usepackage{float}              % 管理浮动体
\usepackage{geometry}     % 定制页面格式
\usepackage{hyperref}       % 为PDF文档创建超链接
\usepackage{lineno}          % 生成行号
\usepackage{listings}        % 插入程序源代码
\usepackage{multicol}       % 多栏排版
%\usepackage{natbib}         % 管理文献引用
\usepackage{rotating}       % 旋转文字,图形,表格
\usepackage{subfigure}    % 排版子图形
\usepackage{titlesec}       % 改变章节标题格式
\usepackage{moresize}   % 更多字体大小
\usepackage{anysize}
\usepackage{indentfirst}  % 首段缩进
\usepackage{booktabs}   % 使用\multicolumn
\usepackage{multirow}    % 使用\multirow

\usepackage{wrapfig}
\usepackage{titlesec}     % 改变标题样式
\usepackage{enumitem}
\usepackage{aas_macros}


\newcommand{\myvec}[1]%
   {\stackrel{\raisebox{-2pt}[0pt][0pt]{\small$\rightharpoonup$}}{#1}}  %矢量符号
\renewcommand{\vec}[1]{\boldsymbol{#1}}
\newcommand{\me}{\mathrm{e}}
\newcommand{\mi}{\mathrm{i}}
\newcommand{\dif}{\mathrm{d}}
\newcommand{\tabincell}[2]{\begin{tabular}{@{}#1@{}}#2\end{tabular}}

\def\kpc{{\rm kpc}}
\def\km{{\rm km}}
\def\cm{{\rm cm}}
\def\TeV{{\rm TeV}}
\def\GeV{{\rm GeV}}
\def\MeV{{\rm MeV}}
\def\GV{{\rm GV}}
\def\MV{{\rm MV}}
\def\yr{{\rm yr}}
\def\s{{\rm s}}
\def\ns{{\rm ns}}
\def\GHz{{\rm GHz}}
\def\muGs{{\rm \mu Gs}}
\def\arcsec{{\rm arcsec}}
\def\K{{\rm K}}
\def\microK{\mu{\rm K}}
\def\sr{{\rm sr}}
\newcolumntype{p}{D{,}{\pm}{-1}}

\renewcommand{\figurename}{Fig.}
\renewcommand{\tablename}{Tab.}

\renewcommand{\arraystretch}{1.5}

\newtheorem{theorem}{THEOREM}[section]

\setlength{\parindent}{0pt}  %取消每段开头的空格

\newcounter{theo}[section]\setcounter{theo}{0}
\renewcommand{\thetheo}{\arabic{section}.\arabic{theo}}
\newenvironment{theo}[2][]{%
\refstepcounter{theo}%
\ifstrempty{#1}%
{\mdfsetup{%
frametitle={%
\tikz[baseline=(current bounding box.east),outer sep=0pt]
\node[anchor=east,rectangle,fill=blue!20]
{\strut Theorem~\thetheo};}}
}%
{\mdfsetup{%
frametitle={%
\tikz[baseline=(current bounding box.east),outer sep=0pt]
\node[anchor=east,rectangle,fill=blue!20]
{\strut Theorem~\thetheo:~#1};}}%
}%
\mdfsetup{innertopmargin=10pt,linecolor=blue!20,%
linewidth=2pt,topline=true,%
frametitleaboveskip=\dimexpr-\ht\strutbox\relax
}
\begin{mdframed}[]\relax%
\label{#2}}{\end{mdframed}}

\newcommand*\widefbox[1]{\fbox{\hspace{2em}#1\hspace{2em}}}



\title{Eigenvalues and Eigenvectors}
\author{}
\date{\today}
\begin{document}

\maketitle

\section{Eigenvectors and Eigenvalues}
If $A$ is a stochastic matrix, then the steady-state vector $\vec{q}$ for $A$ satisfies the equation $A \vec{x} = \vec{x}$, i.e. $A \vec{q} = 1\cdot \vec{q}$.

\begin{tcolorbox}[colback=green!5,colframe=green!40!black,title= Definition]
An \textcolor{red}{eigenvector} of an $n\times n$ matrix $A$ is a nonzero vector $\vec{x}$ such that $A \vec{x} = \lambda \vec{x}$ for some scalar $\lambda$. A scalar $\lambda$ is called an \textcolor{red}{eigenvalue} of $A$ if there is a \textcolor{red}{nontrivial} solution $\vec{x}$ of $A \vec{x} = \lambda \vec{x}$; such an $\vec{x}$ is called an \textcolor{red}{eigenvector} corresponding to $\lambda$. An eigenvector must be nonzero, but an eigenvalue may be zero.
\end{tcolorbox}
\textcolor{red}{Although row reduction can be used to find eigenvalues, but it can not be used to find eigenvectors. An echelon form of a matrix $\mathbf{A}$ usually does not display the eigenvalues of $\mathbf{A}$.}


$\lambda$ is an eigenvalue of $\mathbf{A}$ if and only if the equation
\begin{equation}
(A -\lambda \boldsymbol{I}) \vec{x} = \vec{0}
\label{eigen_3}
\end{equation}
has a nontrivial solution. The set of all solutions of (\ref{eigen_3}) is just the \textcolor{red}{null space} of the matrix $A − \lambda I$.  This set is a \textcolor{red}{subspace} of $\mathbb{R}^n$ and is called the \textcolor{red}{eigenspace} of $\mathbf{A}$ corresponding to $\lambda$. The eigenspace consists of the zero vector and all the eigenvectors corresponding to $\lambda$.

\begin{tcolorbox}[colback=green!5,colframe=green!40!black,title=Theorem]
The eigenvalues of a triangular matrix are the  \textcolor{red}{entries on its main diagonal}.
\end{tcolorbox}

	
$0$ is an eigenvalue of $\mathbf{A}$ if and only if $\mathbf{A}$ is not invertible.

\begin{tcolorbox}[colback=green!5,colframe=green!40!black,title=Theorem]
If $\mathbf{v}_1, \cdots , \mathbf{v}_r$ are eigenvectors that correspond to \textcolor{red}{distinct eigenvalues} $\lambda_1, \cdots , \lambda_r$ of an $n\times n$ matrix $\mathbf{A}$, then the set $\{ \mathbf{v}1, \cdots, \mathbf{v}_r \}$ is \textcolor{red}{linearly independent}.
\end{tcolorbox}




\section{Characteristic Equation}
Let $\mathbf{A}$ be an $n\times n$ matrix, $U$ be any echelon form obtained from $\mathbf{A}$ by row replacements and row interchanges (without scaling), and $r$ be the number of such row interchanges. The \textcolor{red}{determinant} of $\mathbf{A}$, written as \textcolor{red}{det A}, is $(−1)^r$ times the product of the diagonal entries $u_{11}, \cdots , u_{nn}$ in $U$. If $\mathbf{A}$ is invertible, then $u_{11}, \cdots , u_{nn}$ are all \textcolor{red}{pivots} (because
$\mathbf{A} \sim I_n$ and the $u_{ii}$ have not been scaled to 1's). Otherwise, at least unn is zero, and the product $u_{11} \cdots u_{nn}$ is zero. 


\subsection{Determinants}
Let $A$ be an $n\times n$ matrix, $U$ be any echelon form obtained from $A$ by row replacements and row interchanges (without scaling), and $r$ be the number of such row interchanges. Then the determinant of $A$, written as det $A$, is $(-1)^r$ times the product of the diagonal entries $u_{11}, \cdots, u_{nn}$ in $U$. If $A$ is invertible, then $u_{11}, \cdots, u_{nn}$ are all pivots (because $A \sim I_n$ and the $u_{ii}$ have not been scaled to $1$'s). Otherwise, at least $u_{nn}$ is zero, and the product $u_{11}\cdots u_{nn}$ is zero.
\begin{equation}
\text{det} A = 
\left\{\begin{array}{ll}
(-1)^r \cdot (\text{product of pvots in $U$}), ~~\text{when $A$ is invertible} \\
0,  ~~\text{when $A$ is not invertible}
\end{array}\right.
\end{equation}


\begin{tcolorbox}[colback=green!5,colframe=green!40!black,title= The Invertible Matrix Theorem (continued)]
Let $A$ be an $n\times n$ matrix. Then $A$ is invertible if and only if : \\
s. The number $0$ is not an eigenvalue of $A$. \\
t. The determinant of $A$ is not zero.
\end{tcolorbox}


\begin{tcolorbox}[colback=green!5,colframe=green!40!black,title= Properties of Determinants]
Let $A$ and $B$ be an $n\times n$ matrix. \\
a. $A$ is invertible if and only if det $A \neq 0$. \\
b. det $A B = (\text{det} A)(\text{det} B)$. \\
c. det $A^T =$ det $A$. \\
d. If $A$ is triangular, then det $A$ is the product of the entries on the main diagonal of $A$. \\
e. A row replacement operation on $A$ does not change the determinant. A row interchange changes the sign of the determinant. A row scaling also scales the determinant by the same scalar factor. 
\end{tcolorbox}

\subsection{The Characteristic Equation}
The scalar equation det $(A - \lambda \boldsymbol{I}) = 0$ is called the \textcolor{red}{characteristic equation of $A$}.

A scalar $\lambda$ is an eigenvalue of an $n\times n$ matrix $A$ if and only if $\lambda$ satisfies the characteristic equation
\begin{equation*}
\text{det} (A - \lambda \boldsymbol{I}) = 0 ~.
\end{equation*}





\section{Diagonalization}
The eigenvalue-eigenvector information contained within a matrix $A$ can be displayed in a useful factorization of the form $A = PDP^{-1}$. 

A square matrix $A$ is said to be diagonalizable if $A$ is similar to a diagonal matrix, that is, if $A = PDP^{-1}$ for some invertible matrix $P$ and some diagonal matrix $D$.

\begin{tcolorbox}[colback=green!5,colframe=green!40!black,title= The Diagonalization Theorem]
An $n\times n$ matrix $A$ is diagonalizable if and only if $A$ has n linearly independent eigenvectors. \\
In fact, $A = PDP^{-1}$, with $D$ a diagonal matrix, if and only if the columns of $P$ are $n$ linearly independent eigenvectors of $A$. In this case, the diagonal entries of $D$ are eigenvalues of $A$ that correspond, respectively, to the eigenvectors in $P$.
\end{tcolorbox}
In other words, $A$ is diagonalizable if and only if there are enough eigenvectors to form a basis of $\mathbb R^n$. Such a basis is called an \textcolor{red}{eigenvector basis}.


\subsection{Diagonalizing Matrices}

\begin{tcolorbox}[colback=green!5,colframe=green!40!black,title= Theorem]
An $n\times n$ matrix with n distinct eigenvalues is diagonalizable.
\end{tcolorbox}


\subsection{Matrices Whose Eigenvalues Are Not Distinct}



\begin{tcolorbox}[colback=green!5,colframe=green!40!black,title= Theorem]
Let $A$ be an $n\times n$ matrix whose distinct eigenvalues are $\lambda_1, \cdots, \lambda_p$. \\
a. For $1 \leqslant k \leqslant p$, the dimension of the eigenspace for $\lambda_k$ is less than or equal to the multiplicity of the eigenvalue $\lambda_k$. \\
b. The matrix $A$ is diagonalizable if and only if the sum of the dimensions of the distinct eigenspaces equals $n$, and this happens if and only if the dimension of the eigenspace for each $\lambda_k$ equals the multiplicity of $\lambda_k$. \\
c. If $A$ is diagonalizable and $\mathcal B_k$ is a basis for the eigenspace corresponding to λk for each k, then the total collection of vectors in the sets $\mathcal B_1, \cdots, \mathcal B_p$ forms an eigenvector basis for $\mathbb R^n$.
\end{tcolorbox}




\section{Eigenvectors and Linear Transformations}
Any linear transformation $T$ from $\mathbb R^n$ to $\mathbb R^m$ can be implemented via left-multiplication by a matrix $A$, called the standard matrix of $T$.

\subsection{The Matrix of a Linear Transformation}
Let $V$ be an $n$-dimensional vector space, $W$ an $m$-dimensional vector space, and $T$ any linear transformation from $V$ to $W$. To associate a matrix with $T$, choose (ordered) bases $\mathcal B$ and $\mathcal C$ for $V$ and $W$, respectively.





\subsection{Linear Transformations from $V$ into $V$}




\subsection{Linear Transformations on $\mathbb R^n$}
\begin{tcolorbox}[colback=green!5,colframe=green!40!black,title= Diagonal Matrix Representation]
Suppose $A = PDP^{-1}$, where $D$ is a diagonal $n\times n$ matrix. If $\mathcal B$ is the basis for $\mathbb R^n$ formed from the columns of $P$, then $D$ is the $\mathcal B$-matrix for the transformation $\vec{x} \mapsto A \vec{x}$.
\end{tcolorbox}



\subsection{Similarity of Matrix Representations}





\section{Complex Eigenvalues}



\section{Discrete Dynamical Systems}



\section{Applications to Differential Equations}





\section{Iterative Estimates for Eigenvalues}




























\end{document}
