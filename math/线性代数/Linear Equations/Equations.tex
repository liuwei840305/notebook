\documentclass[12pt,a4paper]{article}
%\usepackage{fontspec, xunicode, xltxtra}
%\setmainfont{Hiragino Sans GB}
%\usepackage{xeCJK}
%\setCJKmainfont[BoldFont=STZhongsong, ItalicFont=STKaiti]{STSong}
%\setCJKsansfont[BoldFont=STHeiti]{STXihei}
%\setCJKmonofont{STFangsong}

%使用Xelatex编译

% 设置页面
%==================================================
\linespread{2} %行距
% \usepackage[top=1in,bottom=1in,left=1.25in,right=1.25in]{geometry}
% \headsep=2cm
% \textwidth=16cm \textheight=24.2cm
%==================================================

% 其它需要使用的宏包
%==================================================
\usepackage[colorlinks,linkcolor=blue,anchorcolor=red,citecolor=green,urlcolor=blue]{hyperref} 
\usepackage{tabularx}
\usepackage{authblk}         % 作者信息
\usepackage{algorithm}     % 算法排版
\usepackage{amsmath}     % 数学符号与公式
\usepackage{amsfonts}     % 数学符号与字体
\usepackage{mathrsfs}      % 花体
\usepackage[framemethod=TikZ]{mdframed}

\usepackage{graphicx} 
\usepackage{graphics}
\usepackage{color}
\usepackage{xcolor}
\usepackage{tcolorbox}
\usepackage{lipsum}
\usepackage{empheq}

\usepackage{fancyhdr}       % 设置页眉页脚
\usepackage{fancyvrb}       % 抄录环境
\usepackage{float}              % 管理浮动体
\usepackage{geometry}     % 定制页面格式
\usepackage{hyperref}       % 为PDF文档创建超链接
\usepackage{lineno}          % 生成行号
\usepackage{listings}        % 插入程序源代码
\usepackage{multicol}       % 多栏排版
%\usepackage{natbib}         % 管理文献引用
\usepackage{rotating}       % 旋转文字,图形,表格
\usepackage{subfigure}    % 排版子图形
\usepackage{titlesec}       % 改变章节标题格式
\usepackage{moresize}   % 更多字体大小
\usepackage{anysize}
\usepackage{indentfirst}  % 首段缩进
\usepackage{booktabs}   % 使用\multicolumn
\usepackage{multirow}    % 使用\multirow

\usepackage{wrapfig}
\usepackage{titlesec}     % 改变标题样式
\usepackage{enumitem}
\usepackage{aas_macros}

\renewcommand{\vec}[1]{\boldsymbol{#1}}
\newcommand{\me}{\mathrm{e}}
\newcommand{\mi}{\mathrm{i}}
\newcommand{\dif}{\mathrm{d}}
\newcommand{\tabincell}[2]{\begin{tabular}{@{}#1@{}}#2\end{tabular}}

\def\kpc{{\rm kpc}}
\def\km{{\rm km}}
\def\cm{{\rm cm}}
\def\TeV{{\rm TeV}}
\def\GeV{{\rm GeV}}
\def\MeV{{\rm MeV}}
\def\GV{{\rm GV}}
\def\MV{{\rm MV}}
\def\yr{{\rm yr}}
\def\s{{\rm s}}
\def\ns{{\rm ns}}
\def\GHz{{\rm GHz}}
\def\muGs{{\rm \mu Gs}}
\def\arcsec{{\rm arcsec}}
\def\K{{\rm K}}
\def\microK{\mu{\rm K}}
\def\sr{{\rm sr}}
\newcolumntype{p}{D{,}{\pm}{-1}}

\renewcommand{\figurename}{Fig.}
\renewcommand{\tablename}{Tab.}

\renewcommand{\arraystretch}{1.5}

\setlength{\parindent}{0pt}  %取消每段开头的空格

\newcounter{theo}[section]\setcounter{theo}{0}
\renewcommand{\thetheo}{\arabic{section}.\arabic{theo}}
\newenvironment{theo}[2][]{%
\refstepcounter{theo}%
\ifstrempty{#1}%
{\mdfsetup{%
frametitle={%
\tikz[baseline=(current bounding box.east),outer sep=0pt]
\node[anchor=east,rectangle,fill=blue!20]
{\strut Theorem~\thetheo};}}
}%
{\mdfsetup{%
frametitle={%
\tikz[baseline=(current bounding box.east),outer sep=0pt]
\node[anchor=east,rectangle,fill=blue!20]
{\strut Theorem~\thetheo:~#1};}}%
}%
\mdfsetup{innertopmargin=10pt,linecolor=blue!20,%
linewidth=2pt,topline=true,%
frametitleaboveskip=\dimexpr-\ht\strutbox\relax
}
\begin{mdframed}[]\relax%
\label{#2}}{\end{mdframed}}

\newcommand*\widefbox[1]{\fbox{\hspace{2em}#1\hspace{2em}}}



\title{Linear Equations in Linear Algebra}
\author{}
\date{\today}
\begin{document}

\maketitle
\section{Systems of Linear Equations}



















\section{Row Reduction and Echelon Forms}
















\section{Vector Equations}

















\section{The Matrix Equation $A\vec{x} = \vec{b}$}















\section{Solution Sets of Linear Systems}















\section{Applications of Linear Systems}













\section{Linear Independence}
\begin{tcolorbox}[colback=green!5,colframe=green!40!black,title= Definition]
An indexed set of vectors $\{\vec{v}_1, \cdots, \vec{v}_p\}$ in $\mathbb R^n$ is said to be \textcolor{red}{linearly independent} if the vector equation
\begin{equation}
x_1\vec{v}_1 +x_2\vec{v}_2 +\cdots +x_p\vec{v}_p = \vec{0} ~,
\end{equation}
has only the \textcolor{red}{trivial solution}. The set $\{\vec{v}_1, \cdots, \vec{v}_p\}$ is said to be \textcolor{red}{linearly dependent} if there exist weights $c_1, \cdots, c_p$, \textcolor{red}{not all zero}, such that
\begin{equation}
c_1\vec{v}_1 +c_2\vec{v}_2 +\cdots +c_p\vec{v}_p = \vec{0} ~.
\label{linear_dep}
\end{equation}
\end{tcolorbox}
Equation (\ref{linear_dep}) is called a linear dependence relation among $\{\vec{v}_1, \cdots, \vec{v}_p\}$ when the weights are not all zero. An indexed set is linearly dependent if and only if it is not linearly independent.  $\{\vec{v}_1, \cdots, \vec{v}_p\}$ are linearly dependent when we mean that $\{\vec{v}_1, \cdots, \vec{v}_p\}$ is a linearly dependent set.

\subsection{Linear Independence of Matrix Columns}
The matrix equation $A\vec{x} = \vec{0}$ can be written as
\begin{equation}
x_1\vec{a}_1 +x_2\vec{a}_2 + \cdots +x_n\vec{a}_n = \vec{0} ~.
\end{equation}
Each linear dependence relation among the columns of $A$ corresponds to a nontrivial solution of $A\vec{x} = \vec{0}$.

The columns of a matrix $A$ are linearly independent if and only if the equation $A\vec{x} = \vec{0}$ has only the trivial solution.
 
\subsection{Sets of One or Two Vectors}
A set of two vectors $\{\vec{v}_1 , \vec{v}_2 \}$ is linearly dependent if at least one of the vectors is a multiple of the other. The set is linearly independent if and only if neither of the vectors is a multiple of the other.



\subsection{Sets of Two or More Vectors}
\begin{tcolorbox}[colback=green!5,colframe=green!40!black,title= Characterization of Linearly Dependent Sets]
An indexed set $S = \{\vec{v}_1, \cdots, \vec{v}_p\}$ of two or more vectors is linearly dependent if and only if at least one of the vectors in $S$ is a linear combination of the others. In fact, if $S$ is linearly dependent and $\vec{v}_1 \neq \vec{0}$, then some $\vec{v}_j$ (with $j > 1$) is a linear combination of the preceding vectors, $\vec{v}_1, \cdots, \vec{v}_{j-1}$.
\end{tcolorbox}


\begin{tcolorbox}[colback=green!5,colframe=green!40!black,title= Theorem]
If a set contains more vectors than there are entries in each vector, then the set is linearly dependent. That is, any set $\{\vec{v}_1, \cdots, \vec{v}_p\}$ in $\mathbb R^n$ is linearly dependent if $p > n$.
\end{tcolorbox}


\begin{tcolorbox}[colback=green!5,colframe=green!40!black,title= Theorem]
If a set $S = \{\vec{v}_1, \cdots, \vec{v}_p\}$ in $\mathbb R^n$ contains the zero vector, then the set is linearly dependent.
\end{tcolorbox}



\section{Introduction to Linear Transformations}
Solving the equation $A\vec{x} = \vec{b}$ amounts to finding all vectors $\vec{x}$ in $\mathbb R^4$ that are transformed into the vector $\vec{b}$ in $\mathbb R^2$ under the ``action" of multiplication by $A$.

A \textcolor{red}{transformation} (or \textcolor{red}{function} or \textcolor{red}{mapping}) $T$ from $\mathbb R^n$ to $\mathbb R^m$ is a rule that assigns to each vector $\vec{x}$ in $\mathbb R^n$ a vector $T(\vec{x})$ in $\mathbb R^m$. The set $\mathbb R^n$ is called the \textcolor{red}{domain} of $T$, and $\mathbb R^m$ is called the \textcolor{red}{codomain} of $T$. The notation $T : \mathbb R^n \rightarrow \mathbb R^m$ indicates that the domain of $T$ is $\mathbb R^n$ and the codomain is $\mathbb R^m$. For $\vec{x}$ in $\mathbb R^n$, the vector $T(\vec{x})$ in $\mathbb R^m$ is called the \textcolor{red}{image} of $\vec{x}$ (under the action of $T$). The set of all images $T(\vec{x})$ is called the \textcolor{red}{range} of $T$.


\subsection{Matrix Transformations}











\section{The Matrix of a Linear Transformation}
















\section{Linear Models in Business, Science, and Engineering}




















\end{document}
