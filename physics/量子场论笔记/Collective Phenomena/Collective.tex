\documentclass[11pt,a4paper]{article}
%\usepackage{fontspec, xunicode, xltxtra}  
%\setmainfont{Hiragino Sans GB}  
%\usepackage{xeCJK}
%\setCJKmainfont[BoldFont=STZhongsong, ItalicFont=STKaiti]{STSong}
%\setCJKsansfont[BoldFont=STHeiti]{STXihei}
%\setCJKmonofont{STFangsong}

%使用Xelatex编译

% 设置页面
%==================================================
\linespread{2} %行距
% \usepackage[top=1in,bottom=1in,left=1.25in,right=1.25in]{geometry}
% \headsep=2cm
% \textwidth=16cm \textheight=24.2cm
%==================================================

% 其它需要使用的宏包
%==================================================
\usepackage[colorlinks,linkcolor=blue,anchorcolor=red,citecolor=green,urlcolor=blue]{hyperref} 
\usepackage{tabularx}
\usepackage{authblk}         % 作者信息
\usepackage{algorithm}     % 算法排版
\usepackage{amsmath}     % 数学符号与公式
\usepackage{amsfonts}     % 数学符号与字体
\usepackage{mathrsfs}      % 花体
\usepackage{amssymb}
\usepackage[framemethod=TikZ]{mdframed}

\usepackage{graphicx} 
\usepackage{graphics}
\usepackage{color}
\usepackage{xcolor}
\usepackage{tcolorbox}
\usepackage{lipsum}
\usepackage{empheq}

\usepackage{fancyhdr}       % 设置页眉页脚
\usepackage{fancyvrb}       % 抄录环境
\usepackage{float}              % 管理浮动体
\usepackage{geometry}     % 定制页面格式
\usepackage{hyperref}       % 为PDF文档创建超链接
\usepackage{lineno}          % 生成行号
\usepackage{listings}        % 插入程序源代码
\usepackage{multicol}       % 多栏排版
%\usepackage{natbib}         % 管理文献引用
\usepackage{rotating}       % 旋转文字,图形,表格
\usepackage{subfigure}    % 排版子图形
\usepackage{titlesec}       % 改变章节标题格式
\usepackage{moresize}   % 更多字体大小
\usepackage{anysize}
\usepackage{indentfirst}  % 首段缩进
\usepackage{booktabs}   % 使用\multicolumn
\usepackage{multirow}    % 使用\multirow

\usepackage{wrapfig}
\usepackage{titlesec}     % 改变标题样式
\usepackage{enumitem}
\usepackage{aas_macros}
\usepackage{bigints}

\renewcommand{\vec}[1]{\boldsymbol{#1}}
\newcommand{\me}{\mathrm{e}}
\newcommand{\mi}{\mathrm{i}}
\newcommand{\dif}{\mathrm{d}}
\newcommand{\tabincell}[2]{\begin{tabular}{@{}#1@{}}#2\end{tabular}}

\def\kpc{{\rm kpc}}
\def\km{{\rm km}}
\def\cm{{\rm cm}}
\def\TeV{{\rm TeV}}
\def\GeV{{\rm GeV}}
\def\MeV{{\rm MeV}}
\def\GV{{\rm GV}}
\def\MV{{\rm MV}}
\def\yr{{\rm yr}}
\def\s{{\rm s}}
\def\ns{{\rm ns}}
\def\GHz{{\rm GHz}}
\def\muGs{{\rm \mu Gs}}
\def\arcsec{{\rm arcsec}}
\def\K{{\rm K}}
\def\microK{\mu{\rm K}}
\def\sr{{\rm sr}}
\newcolumntype{p}{D{,}{\pm}{-1}}

\renewcommand{\figurename}{Fig.}
\renewcommand{\tablename}{Tab.}

\renewcommand{\arraystretch}{1.5}

\setlength{\parindent}{0pt}  %取消每段开头的空格

\newcounter{theo}[section]\setcounter{theo}{0}
\renewcommand{\thetheo}{\arabic{section}.\arabic{theo}}
\newenvironment{theo}[2][]{%
\refstepcounter{theo}%
\ifstrempty{#1}%
{\mdfsetup{%
frametitle={%
\tikz[baseline=(current bounding box.east),outer sep=0pt]
\node[anchor=east,rectangle,fill=blue!20]
{\strut Theorem~\thetheo};}}
}%
{\mdfsetup{%
frametitle={%
\tikz[baseline=(current bounding box.east),outer sep=0pt]
\node[anchor=east,rectangle,fill=blue!20]
{\strut Theorem~\thetheo:~#1};}}%
}%
\mdfsetup{innertopmargin=10pt,linecolor=blue!20,%
linewidth=2pt,topline=true,%
frametitleaboveskip=\dimexpr-\ht\strutbox\relax
}
\begin{mdframed}[]\relax%
\label{#2}}{\end{mdframed}}

\newcommand*\widefbox[1]{\fbox{\hspace{2em}#1\hspace{2em}}}


\title{Field Theory and Collective Phenomena}
\author{}
\date{\today}
\begin{document}

\maketitle

\section{Superfluids}
\cite{2010qftn.book.....Z}
Consider a finite density $\overline{\rho}$ of nonrelativistic bosons interacting with a short ranged repulsion.
\begin{equation}
\mathcal L = i \varphi^{\dagger} \partial_0 \varphi - \dfrac{1}{2m} \partial_i \varphi^{\dagger} \partial_i \varphi - g^2 (\varphi^{\dagger} \varphi - \overline{\rho})^2
\end{equation}
The last term is exactly the Mexican well potential,  forcing the magnitude of $\varphi$ to be close to $\sqrt{\overline{\rho}}$, thus suggesting that we use polar variables $\varphi \equiv \sqrt{\rho} e^{i\theta}$. Plugging in and dropping the total derivative $(i/2)\partial_0 \rho$, 
\begin{equation}
\mathcal L = -\rho \partial_0 -\dfrac{1}{2m} \left[\dfrac{1}{4 \rho} (\partial_i \rho)^2 +\rho (\partial_i \theta)^2 \right] - g^2 (\rho - \overline{\rho})^2 
\end{equation}

$\sqrt{\rho} = \sqrt{\overline{\rho}} +h$(the vacuum expectation value of $\varphi$ is $\sqrt{\overline{\rho}}$), assume $h \ll \sqrt{\overline{\rho}}$, and expand(dropped the (potentially interesting) term $ -\rho \partial_0$ because it is a total divergence)
\begin{align}
\nonumber \mathcal L &= \overline{\rho} \partial_0 \theta \dfrac{1}{4 g^2 \overline{\rho} - (1/2m)\partial_i^2} \partial_0 \theta - \dfrac{\overline{\rho}}{2m} (\partial_i \theta)^2 +\cdots \\
&= \dfrac{1}{4g^2} (\partial_0 \theta)^2 - \dfrac{\overline{\rho}}{2m} (\partial_i \theta)^2
\end{align}
In the second equality we assumed that we are looking at processes with wave number $k$ small compared to $\sqrt{8g^2\overline{ρ}m}$ so that $(1/2m)\partial_i^2$ is negligible compared to $4g^2 \overline{\rho}$. There exists in this fluid of bosons a gapless mode (often referred to as the phonon) with the dispersion
\begin{equation}
\omega^2 = \dfrac{2g^2 \overline{\rho}}{m} \vec{k}^2
\label{disper}
\end{equation}
where Bogoliubov's classic result are obtained without ever doing a Bogoliubov rotation.

A linearly dispersing mode (that is, $\omega$ is linear in $k$) implies superfluidity. Consider a mass $M$ of fluid flowing down a tube with velocity $v$. It could lose momentum and slow down to velocity $v^\prime$ by creating an excitation of momentum $k : Mv = Mv^\prime + \hbar k$. This is only possible with sufficient energy to spare if $Mv^2/2 \geqslant Mv^{\prime 2}/2 +\hbar \omega(k)$. Eliminating $v^\prime$ we obtain for $M$ macroscopic $v \geqslant \omega/k$. For a linearly dispersing mode this gives a critical velocity $v_c \equiv \omega/k$ below which the fluid cannot lose momentum and is hence super. [Thus, from (\ref{disper}) the idealized $v_c = g\sqrt{2 \overline{\rho}/m}$.]

Suitably scaling the distance variable, we can summarize the low energy physics of superfluidity in the compact Lagrangian
\begin{equation}
\mathcal L = \dfrac{1}{4g^2} (\partial_\mu \theta)^2 ~,
\end{equation}
which we recognize as the massless version of the scalar field theory, but with the important proviso that $\theta$ is a phase angle field, that is, $\theta(x)$ and $\theta(x) + 2\pi$ are really the same. This gapless mode is evidently the Nambu-Goldstone boson associated with the spontaneous breaking of the global $U(1)$ symmetry $\varphi \rightarrow e^{i\alpha} \varphi$.

If we think about a gas of free bosons, we can give a momentum  $\hbar \vec{k}$ to any given boson at the cost of only $(\hbar \vec{k})^2/2m$ in energy. There exist many low energy excitations in a free boson system. But as soon as a short ranged repulsion is turned on between the bosons, a boson moving with momentum $\vec{k}$ would affect all the other bosons. A density wave is set up as a result, with energy proportional to k as we have shown in (\ref{disper}). The gapless mode has gone from quadratically dispersing to linearly dispersing. There are far fewer low energy excitations. Specifically, recall that the density of states is given by $N(E) \approx k^{D-1} (\dif k/\dif E)$. For example, for $D = 2$ the density of states goes from $N(E) \approx \rm constant$ (in the presence of quadratically dispersing modes) to $N(E) \approx E$ (in the presence of linearly dispersing modes) at low energies.

The physics of superfluidity lies not in the presence of gapless excitations, but in the paucity of gapless excitations. (After all, the Fermi liquid has a continuum of gapless modes.) There are too few modes that the superfluid can lose energy and momentum to.


\section{Euclid, Boltzmann, Hawking, and Field Theory at Finite Temperature}





































\section{Landau-Ginzburg Theory of Critical Phenomena}
Consider a \textcolor{red}{ferromagnetic material in thermal equilibrium at temperature $T$}. The \textcolor{red}{magnetization $\vec{M}(x)$} is defined as the \textcolor{red}{average of the atomic magnetic moments taken over a region of a size much larger than the length scale characteristic of the relevant microscopic physics}. (we discuss a nonrelativistic theory and $x$ denotes the spatial coordinates only.) At \textcolor{red}{low temperatures}, \textcolor{red}{rotational invariance is spontaneously broken and the material exhibits a bulk magnetization pointing in some direction}. As the \textcolor{red}{temperature is raised past some critical temperature $T_c$}, the \textcolor{red}{bulk magnetization suddenly disappears}. With \textcolor{red}{increased thermal agitation}, the \textcolor{red}{atomic magnetic moments point in increasingly random directions, canceling each other out}. More precisely, it was found experimentally that just \textcolor{red}{below $T_c$}, the magnetization $|\vec{M}|$ vanishes as $\sim (T_c - T)^\beta$, where the so-called \textcolor{red}{critical exponent $\beta \simeq 0.37$}. This sudden change is known as a \textcolor{red}{second order phase transition}. In principle, we are to compute the partition function $Z = {\rm tr} ~e^{-\mathcal H/T}$ with the microscopic Hamiltonian $\mathcal H$, but $Z$ is apparently smooth in $T$ except possibly at $T = 0$. An infinite sum of terms each of which may be analytic in some variable need not be analytic in that variable. The trace in ${\rm tr} ~e^{-\mathcal H/T}$ sums over an infinite number of terms.

The form of the free energy $G$ as a function of $\vec{M}$ for a system with volume $V$ could be argued from general principles. First, for $\vec{M}$ constant in $x$, we have by rotational invariance
\begin{equation}
G = V [a \vec{M}^2 + b (\vec{M}^2)^2 +\cdots] ~
\end{equation}
where $a, b, \cdots$ are unknown (but expected to be smooth) functions of $T$. Suppose that $a$ vanishes at some temperature $T_c$. We expect that for $T$ near $T_c$ we have $a = a_1(T - T_c) + \cdots$ [rather than, say, $a = a_2(T - T_c)^2 + \cdots]$. For $T > T_c$, $G$ is minimized at $\vec{M} = 0$, but as $T$ drops below $T_c$, new minima suddenly develop at $|\vec{M}| = \sqrt{(-a/2b)} \sim (T_c - T)^{1/2}$. Rotational symmetry is spontaneously broken.

To include the possibility of $\vec{M}$ varying in space, $G$ must have the form
\begin{equation}
G = \int \dif^3 x \{\partial_i \vec{M} \partial_i \vec{M} + a\vec{M}^2 +b(\vec{M}^2)^2 +\cdots \}
\label{G}
\end{equation}
where the coefficient of the $(\partial_i \vec{M})^2$ term has been set to $1$ by rescaling $\vec{M}$. Equ. (\ref{G}) is the Euclidean version of the scalar field theory. By dimensional analysis, $1/\sqrt{a}$ sets the length scale. More precisely, for $T > T_c$, turn on a perturbing external magnetic field $\vec{H}(x)$ by adding the term $-\vec{H}\cdot \vec{M}$. Assuming $\vec{M}$ small and minimizing $G$, we obtain $(-\partial^2 + a) \vec{M} \simeq \vec{H}$, with the solution
\begin{align}
\nonumber \vec{M}(x) &= \int \dif^3 y \int \dfrac{\dif^3 k}{(2\pi)^3} \dfrac{e^{i\vec{k}\cdot (\vec{x}-\vec{y})}}{\vec{k}^2 +a} \vec{H}(y) \\
&= \int \dif^3 y \dfrac{1}{4\pi |\vec{x} -\vec{y}|} e^{-\sqrt{a} |\vec{x} -\vec{y}|} \vec{H}(y) 
\end{align}
Define a correlation function $\langle \vec{M}(x) \vec{M}(0)\rangle$ by asking what the magnetization $ \vec{M}(x)$ will be if we use a magnetic field sharply localized at the origin to create a magnetization $ \vec{M}(0)$ there. The correlation function is expected to die off as $e^{-|\vec{x}|/\xi}$ over some correlation length $\xi$ that goes to infinity as $T$ approaches $T_c$ from above. The critical exponent $\nu$ is traditionally defined by $\xi \sim 1/(T - T_c)^\nu$. In Landau-Ginzburg theory, also known as \textcolor{red}{mean field theory}, $\xi = 1/\sqrt{a}$ and hence $\nu = 1/2$. The important point is not how well the predicted critical exponents such as $\beta$ and $\nu$ agree with experiment but how easily they emerge from Landau-Ginzburg theory. 

\section{Superconductivity}




















































































%%%%%%%%%%%%%%%%%%%%%%%%%%%%%%%%%%%%%%%%%%%%%%%%%%%%%%%%%%%%%%%%%%%%%%
\bibliographystyle{unsrt_update}
\bibliography{ref}
%%%%%%%%%%%%%%%%%%%%%%%%%%%%%%%%%%%%%%%%%%%%%%%%%%%%%%%%%%%%%%%%%%%%%%


\end{document}