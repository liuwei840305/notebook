\documentclass[11pt,a4paper]{article}
%\usepackage{fontspec, xunicode, xltxtra}  
%\setmainfont{Hiragino Sans GB}  
\usepackage{xeCJK}
%\setCJKmainfont[BoldFont=STZhongsong, ItalicFont=STKaiti]{STSong}
%\setCJKsansfont[BoldFont=STHeiti]{STXihei}
%\setCJKmonofont{STFangsong}

%使用Xelatex编译

% 设置页面
%==================================================
\linespread{2} %行距
% \usepackage[top=1in,bottom=1in,left=1.25in,right=1.25in]{geometry}
% \headsep=2cm
% \textwidth=16cm \textheight=24.2cm
%==================================================

% 其它需要使用的宏包
%==================================================
\usepackage[colorlinks,linkcolor=blue,anchorcolor=red,citecolor=green,urlcolor=blue]{hyperref} 
\usepackage{tabularx}
\usepackage{authblk}         % 作者信息
\usepackage{algorithm}     % 算法排版
\usepackage{amsmath}     % 数学符号与公式
\usepackage{amsfonts}     % 数学符号与字体
\usepackage{mathrsfs}      % 花体
\usepackage{graphics}
\usepackage{color}
\usepackage{fancyhdr}       % 设置页眉页脚
\usepackage{fancyvrb}       % 抄录环境
\usepackage{float}              % 管理浮动体
\usepackage{geometry}     % 定制页面格式
\usepackage{hyperref}       % 为PDF文档创建超链接
\usepackage{lineno}          % 生成行号
\usepackage{listings}        % 插入程序源代码
\usepackage{multicol}       % 多栏排版
\usepackage{natbib}         % 管理文献引用
\usepackage{rotating}       % 旋转文字,图形,表格
\usepackage{subfigure}    % 排版子图形
\usepackage{titlesec}       % 改变章节标题格式
\usepackage{moresize}   % 更多字体大小
\usepackage{anysize}
\usepackage{indentfirst}  % 首段缩进
\usepackage{booktabs}   % 使用\multicolumn
\usepackage{multirow}    % 使用\multirow
\usepackage{graphicx} 
\usepackage{wrapfig}
\usepackage{xcolor}
\usepackage{titlesec}     % 改变标题样式
\usepackage{enumitem}

\newcommand{\myvec}[1]%
   {\stackrel{\raisebox{-2pt}[0pt][0pt]{\small$\rightharpoonup$}}{#1}}  %矢量符号
\renewcommand{\vec}[1]{\boldsymbol{#1}}
\newcommand{\me}{\mathrm{e}}
\newcommand{\mi}{\mathrm{i}}
\newcommand{\dif}{\mathrm{d}}
\newcommand{\tabincell}[2]{\begin{tabular}{@{}#1@{}}#2\end{tabular}}

\def\kpc{{\rm kpc}}
\def\km{{\rm km}}
\def\cm{{\rm cm}}
\def\TeV{{\rm TeV}}
\def\GeV{{\rm GeV}}
\def\MeV{{\rm MeV}}
\def\GV{{\rm GV}}
\def\MV{{\rm MV}}
\def\yr{{\rm yr}}
\def\s{{\rm s}}
\def\ns{{\rm ns}}
\def\GHz{{\rm GHz}}
\def\muGs{{\rm \mu Gs}}
\def\arcsec{{\rm arcsec}}
\def\K{{\rm K}}
\def\microK{\mu{\rm K}}
\def\sr{{\rm sr}}
\newcolumntype{p}{D{,}{\pm}{-1}}

\renewcommand{\figurename}{Fig.}
\renewcommand{\tablename}{Tab.}

\renewcommand{\arraystretch}{1.5}

\setlength{\parindent}{0pt}  %取消每段开头的空格

\title{原子核物理}
\author{}
\date{\today}
\begin{document}

\maketitle

1. 核中不含有电子,讨论四种论据。\\
解:1.) 统计性。原子核的统计性可以用双原子分子的转动光谱研究。若核$(A, Z)$是由$A$个质子,$A-Z$个电子构成,那么奇奇核或奇偶核的自旋与实验结果不符。偶数个质子耦合成整数自旋,奇数个电子耦合成半整数自旋。${}^{14}N$的总自旋是半整数,即为费米子,与实验不符,说明核不是由质子和电子组成。

2.) 能量。电子是不参与强作用的轻子,若存在于核内,则是库仑相互作用产生的束缚态,束缚能的数量级为
\begin{equation*}
E \approx -\frac{Ze^2}{r} ~,
\end{equation*}
核电磁半径$r = 1.2 A^{1/3}$ (fm),
\begin{align*}
E &\approx -\frac{Ze^2}{r}  \approx -Z \frac{e^2}{c\hbar} \frac{c\hbar}{1.2 A^{1/3} } \\
&\approx -Z \frac{197 ~{\rm MeV\cdot fm}}{137 \times 1.2 A^{1/3} ~{\rm fm}} \\
&\approx -1.2 \frac{Z}{A^{1/3} } ({\rm MeV}) ~.
\end{align*}
当$A \approx 125$,$Z \approx \frac{A}{2}$,
\begin{equation}
E \approx -15 ~{\rm MeV} ~.
\end{equation}
电子的德布罗意波长$\lambda$为
\begin{equation}
\lambda = \frac{\hbar}{p} = \frac{c\hbar}{cp} = \frac{197 ~{\rm MeV\cdot fm}}{15 ~{\rm MeV} } \approx 13 ~{\rm fm}
\end{equation}
$\lambda \gg r$,电子不可能束缚在核内。

3.) 核磁矩。若核是由中子和质子组成,核磁矩应为两种核子磁矩的共同贡献(不同耦合形式有些差异),即核磁矩的量级应是$\mu_N$的量级。若核是由质子和电子构成,则应是$\mu_e$的量级,但$\mu_e \approx 1800 \mu_N$。实验结果支持前者,与后者相差很大。

4.) $\beta$衰变。原子核发生$\beta$衰变时放出电子,若核内只有质子和电子,则核放出电子并剩下子核,实为二体衰变,故电子应该是单能谱,与实际$\beta$连续谱矛盾。


2. 原子核的大小可以通过下面方法测定:a) 电子散射;b) $\mu$原子的能级;c) 同位旋多重态的基态能量。













\end{document}