\documentclass[12pt,a4paper]{article}
%\usepackage{fontspec, xunicode, xltxtra}  
%\setmainfont{Hiragino Sans GB}  
%\usepackage{xeCJK}
%\setCJKmainfont[BoldFont=STZhongsong, ItalicFont=STKaiti]{STSong}
%\setCJKsansfont[BoldFont=STHeiti]{STXihei}
%\setCJKmonofont{STFangsong}

%使用Xelatex编译

% 设置页面
%==================================================
\linespread{2} %行距
% \usepackage[top=1in,bottom=1in,left=1.25in,right=1.25in]{geometry}
% \headsep=2cm
% \textwidth=16cm \textheight=24.2cm
%==================================================

% 其它需要使用的宏包
%==================================================
\usepackage[colorlinks,linkcolor=blue,anchorcolor=red,citecolor=green,urlcolor=blue]{hyperref} 
\usepackage{tabularx}
\usepackage{authblk}         % 作者信息
\usepackage{algorithm}     % 算法排版
\usepackage{amsmath}     % 数学符号与公式
\usepackage{amsfonts}     % 数学符号与字体
\usepackage{mathrsfs}      % 花体
\usepackage{amssymb}
\usepackage[framemethod=TikZ]{mdframed}

\usepackage{graphicx} 
\usepackage{graphics}
\usepackage{color}
\usepackage{xcolor}
\usepackage{tcolorbox}
\usepackage{lipsum}
\usepackage{empheq}

\usepackage{fancyhdr}       % 设置页眉页脚
\usepackage{fancyvrb}       % 抄录环境
\usepackage{float}              % 管理浮动体
\usepackage{geometry}     % 定制页面格式
\usepackage{hyperref}       % 为PDF文档创建超链接
\usepackage{lineno}          % 生成行号
\usepackage{listings}        % 插入程序源代码
\usepackage{multicol}       % 多栏排版
%\usepackage{natbib}         % 管理文献引用
\usepackage{rotating}       % 旋转文字,图形,表格
\usepackage{subfigure}    % 排版子图形
\usepackage{titlesec}       % 改变章节标题格式
\usepackage{moresize}   % 更多字体大小
\usepackage{anysize}
\usepackage{indentfirst}  % 首段缩进
\usepackage{booktabs}   % 使用\multicolumn
\usepackage{multirow}    % 使用\multirow

\usepackage{wrapfig}
\usepackage{titlesec}     % 改变标题样式
\usepackage{enumitem}
\usepackage{aas_macros}
\usepackage{bigints}

\renewcommand{\vec}[1]{\boldsymbol{#1}}
\newcommand{\me}{\mathrm{e}}
\newcommand{\mi}{\mathrm{i}}
\newcommand{\dif}{\mathrm{d}}
\newcommand{\tabincell}[2]{\begin{tabular}{@{}#1@{}}#2\end{tabular}}

\def\kpc{{\rm kpc}}
\def\km{{\rm km}}
\def\cm{{\rm cm}}
\def\TeV{{\rm TeV}}
\def\GeV{{\rm GeV}}
\def\MeV{{\rm MeV}}
\def\GV{{\rm GV}}
\def\MV{{\rm MV}}
\def\yr{{\rm yr}}
\def\s{{\rm s}}
\def\ns{{\rm ns}}
\def\GHz{{\rm GHz}}
\def\muGs{{\rm \mu Gs}}
\def\arcsec{{\rm arcsec}}
\def\K{{\rm K}}
\def\microK{\mu{\rm K}}
\def\sr{{\rm sr}}
\newcolumntype{p}{D{,}{\pm}{-1}}

\renewcommand{\figurename}{Fig.}
\renewcommand{\tablename}{Tab.}

\renewcommand{\arraystretch}{1.5}

\setlength{\parindent}{0pt}  %取消每段开头的空格

\newcounter{theo}[section]\setcounter{theo}{0}
\renewcommand{\thetheo}{\arabic{section}.\arabic{theo}}
\newenvironment{theo}[2][]{%
\refstepcounter{theo}%
\ifstrempty{#1}%
{\mdfsetup{%
frametitle={%
\tikz[baseline=(current bounding box.east),outer sep=0pt]
\node[anchor=east,rectangle,fill=blue!20]
{\strut Theorem~\thetheo};}}
}%
{\mdfsetup{%
frametitle={%
\tikz[baseline=(current bounding box.east),outer sep=0pt]
\node[anchor=east,rectangle,fill=blue!20]
{\strut Theorem~\thetheo:~#1};}}%
}%
\mdfsetup{innertopmargin=10pt,linecolor=blue!20,%
linewidth=2pt,topline=true,%
frametitleaboveskip=\dimexpr-\ht\strutbox\relax
}
\begin{mdframed}[]\relax%
\label{#2}}{\end{mdframed}}

\newcommand*\widefbox[1]{\fbox{\hspace{2em}#1\hspace{2em}}}


\title{Angular momentum}
\author{}
\date{\today}
\begin{document}

\maketitle

\cite{binney2013physics} The angular-momentum operators $J_i$ can be introduced as the generators of rotations. They form a \textcolor{yellow}{pseudovector}, so $J^2 = \sum_i J_i^2$ is a scalar. By considering the effect of rotations on vectors and scalars, the $J_i$ commute with all scalar operators, including $J^2$, and the commutator of $J_i$ with a component of a vector operator is given by 
\begin{equation}
[J_i, v_j] = i \sum_k \epsilon_{ijk} v_k ~.
\end{equation}
$J_i$ do not commute with one another, but satisfy $[J_i, J_j] = i \sum_k \epsilon_{ijk} J_k$.

When the Hamiltonian is invariant under rotations about some axis $\vec{\hat{\alpha}}$, and the \textcolor{yellow}{system starts in an eigenstate of the corresponding angular-momentum operator $\vec{\hat{\alpha}} \cdot \vec{J}$}, it will subsequently \textcolor{yellow}{remain in that eigenstate}. Consequently, the corresponding \textcolor{yellow}{eigenvalue is a conserved quantity}. In classical mechanics dynamical symmetry about some axis implies that the component of angular momentum about that axis is conserved, so it is plausible that the \textcolor{yellow}{conserved eigenvalue is a measure of angular momentum}. How the orientation of a system is encoded in the amplitudes for it to be found in different eigenstates of appropriate angular-momentum operators.

\section{Eigenvalues of $J_z$ and $J^2$}
\cite{binney2013physics} Since no two components of $\vec{J}$ commute, we cannot find a complete set of simultaneous eigenkets of two components of $\vec{J}$. We can, however, find a complete set of mutual eigenkets of $J^2$ and one component of $\vec{J}$ because $[J^2, J_i] = 0$. Orient the coordinates so that the chosen component of $\vec{J}$ is $J_z$. Label a ket which is simultaneously an eigenstate of $J^2$ and $J_z$ as $|\beta, m\rangle$, where
\begin{equation}
J^2 |\beta, m \rangle  = \beta  |\beta, m \rangle ~, ~~~ J_z |\beta, m \rangle = m |\beta, m \rangle ~.
\end{equation}
Define 
\begin{equation}
J_\pm = J_x \pm i J_y ~.
\end{equation}
These objects clearly commute with $J^2$, while their commutation relations with $J_z$ are
\begin{align}
[J_+, J_z] = [J_x, J_z] +i[J_y, J_z] = -iJ_y -J_x  = -J_+ ~, \\
[J_-, J_z] = [J_x, J_z] -i[J_y, J_z] = -iJ_y +J_x  = J_- ~.
\end{align}
Since $J_\pm$ commutes with $J^2$, the kets $J_\pm |\beta, m\rangle$ are eigenkets of $J^2$ with eigenvalue $\beta$. Operating with $J_z$ on these kets 
\begin{align}
J_z J_+ |\beta, m\rangle = (J_+ J_z +[J_z, J_+]) |\beta, m\rangle = (m+1) J_+ |\beta, m\rangle ~, \\
J_z J_- |\beta, m\rangle = (J_- J_z +[J_z, J_-]) |\beta, m\rangle = (m-1) J_- |\beta, m\rangle ~.
\end{align}
Thus, $J_+ |\beta, m\rangle$ and $J_- |\beta, m\rangle$ are also members of the complete set of states that are eigenstates of both $J^2$ and $J_z$, but their eigenvalues with respect to $J_z$ differ from that of $|\beta, m\rangle$ by $\pm 1$. 
\begin{equation}
J_\pm |\beta, m\rangle = \alpha_\pm |\beta, m\pm 1\rangle ~,
\end{equation}



























\section{Spin and orbital angular momentum}
\cite{binney2013physics} 






























\section{Physics of spin}
\cite{binney2013physics} The wavefunction of a particle with non-zero spin $s$ has $2s+1$ components $\psi_m(\vec{x})$, the $m$-th component giving the amplitude to be found at $\vec{x}$ in the spin state $|s, m\rangle$. The spin operators $S_i$ act on these multi-component wavefunctions by matrix multiplication.



























\section{Orbital angular-momentum eigenfunctions}
\cite{binney2013physics} In the position representation, the $L_i$ become differential operators.
\begin{equation}
L_z = \dfrac{1}{\hbar} (xp_y - yp_x) = -i \left(x \dfrac{\partial }{\partial y} -y \dfrac{\partial }{\partial x}  \right) ~.
\end{equation}
Let $(r, \theta, \phi)$ be standard spherical polar coordinates. The chain rule states that
\begin{equation}

\end{equation}










\section{Three-dimensional harmonic oscillator}
\cite{binney2013physics} 

















\section{Addition of angular momenta}
\cite{binney2013physics} Imagine two gyros in a box and the first gyro has total angular-momentum quantum number $j_1$, while the second gyro has total quantum number $j_2$, $j_1 \geqslant j_2$. A ket describing the state of the first gyro is of the
form
\begin{equation}
|\psi_1 \rangle  = \sum_{m = -j_1}^{j_1} c_m |j_1, m \rangle ~,
\end{equation}
while the state of the second is
\begin{equation}
|\psi_2 \rangle  = \sum_{m = -j_2}^{j_2} d_m |j_2, m \rangle ~,
\end{equation}
and the state of the box is
\begin{equation}
|\psi \rangle = |\psi_1 \rangle |\psi_2 \rangle  ~.
\end{equation}
The coefficients $c_m$ and $d_m$ are the amplitudes to find the individual gyros in particular orientations with respect to the $z$-axis.

The operators of interest are the operators $J_i^2, J_{iz}$ and $J_{i \pm}$ of the $i$-th gyro and the corresponding operators of the box. The operators $J_z$ and $J_\pm$ for the box are simply sums of the corresponding operators for the gyros
\begin{equation}
J_z = J_{1z} +J_{2z} ~, ~~~~ J_\pm = j_{1\pm} +J_{2\pm} ~.
\end{equation}
Operators belonging to different systems always commute, so $[J_{1i}, J_{2j}] = 0$ for any values of $i, j$. The operator for the square of the box's angular momentum is
\begin{equation}
J^2 = (\vec{J}_1 +\vec{J}_2)^2 = J_1^2 +J_2^2 +2\vec{J}_1\cdot \vec{J}_2 ~.
\end{equation}
\begin{equation}
J_{1+}J_{2-} = (J_{1x} +iJ_{1y} )(J_{2x} -iJ_{2y} ) = (J_{1x}J_{2x} +J_{1y}J_{2y} ) +i(J_{1y}J_{2x}  - J_{1x}J_{2y} ) ~.
\end{equation}
\begin{equation}
J_{1+}J_{2-} +J_{1-}J_{2+} +2J_{1z}J_{2z} = 2 \vec{J}_1\cdot \vec{J}_2 ~.
\end{equation}
\begin{equation}
J^2 = J_1^2 +J_2^2 + J_{1+}J_{2-} +J_{1-}J_{2+} +2J_{1z}J_{2z} ~.
\end{equation}
While the total angular momenta of the individual gyros are fixed, that of the box is variable because it depends on the mutual orientation of the two gyros: if the latter are parallel, the squared angular momentum in the box might be expected to have quantum number $j_1 + j_2$, while if they are antiparallel, the box’s angular momentum might be expected to have quantum number $j_1 - j_2$.
\begin{align}
\nonumber J^2 |j_1, j_1 \rangle |j_2, j_2 \rangle &= (J_1^2 +J_2^2 +J_{1+}J_{2-} +J_{1-}J_{2+} +2 J_{1z} J_{2z}) |j_1, j_1 \rangle |j_2, j_2 \rangle \\
\nonumber &= \{j_1(j_1 +1) +j_2(j_2 +1) +2j_1 j_2 \} |j_1, j_1 \rangle |j_2, j_2 \rangle ~, \\
&= (j_1+j_2)(j_1+j_2 +1)  |j_1, j_1 \rangle |j_2, j_2 \rangle ~.
\end{align}
where $J_{i+} |j_i, j_i \rangle = 0$. The expression in curly brackets in equation equals $j(j+1)$ with $j = j_1 +j_2$. Hence $|j_1, j_1\rangle |j_2, j_2 \rangle$ satisfies both the defining equations of the state $|j_1 + j_2, j_1 + j_2 \rangle$ and 
\begin{equation}
|j_1 + j_2, j_1 + j_2 \rangle =  |j_1, j_1 \rangle |j_2, j_2 \rangle ~.
\end{equation}
\begin{align}
J_-  |j, j \rangle &= (J_{1-} +J_{2-} )  |j_1, j_1 \rangle |j_2, j_2 \rangle ~, \\
&= \sqrt{j(j+1) -j(j-1)} |j, j-1 \rangle = \sqrt{2 j} |j, j-1 \rangle ~.
\end{align}
\begin{align}
\nonumber &\sqrt{j_1(j_1+1) -j_1(j_1-1)} |j_1, j_1-1 \rangle |j_2, j_2 \rangle +\sqrt{j_2(j_2+1) -j_2(j_2-1)} |j_1, j_1 \rangle |j_2, j_2 -1 \rangle \\
&= \sqrt{2j_1} |j_1, j_1-1 \rangle |j_2, j_2 \rangle +\sqrt{2j_2} |j_1, j_1 \rangle |j_2, j_2 -1 \rangle  ~.
\end{align}
\begin{equation}
|j, j-1 \rangle = \sqrt{\dfrac{j_1}{j} } |j_1, j_1-1 \rangle |j_2, j_2 \rangle  + \sqrt{\dfrac{j_2}{j} }  |j_1, j_1 \rangle |j_2, j_2 -1 \rangle ~.
\end{equation}


The numbers
\begin{equation}
C(j,m; j_1, j_2, m_1, m_2) \equiv \langle j,m| j_1, m_1\rangle | j_2, m_2\rangle ~,
\end{equation}
re called \textcolor{red}{\bf Clebsch-Gordan coefficients}. They have a simple physical interpretation: $C(j, m; j_1, j_2, m_1, m_2)$ is the amplitude that, on opening the box when it's in a state of well-defined angular momentum, we will find the first and second gyros to be oriented with amounts $m_1$ and $m_2$ of their spins parallel to the $z$-axis. 



\cite{greiner1994quantum} Consider the case of two angular momenta $\vec{\hat{J}}_1$ and $\vec{\hat{J}}_2$ which are combined to form a total angular momentum $\vec{\hat{J}}$,
\begin{equation*}
\vec{\hat{J}} = \vec{\hat{J}}_1 +\vec{\hat{J}}_2 ~.
\end{equation*}
The sum of angular momentum operators obeys the same commutation relations as the individual angular momentum operators. Let $\psi_{j_1,m_1}$ and $\psi_{j_2,m_2}$ be the orthonormal set of eigenfunctions of the operators $\vec{\hat{J}}_1,  \hat{J}_{1z}$ and $\vec{\hat{J}}_2,  \hat{J}_{2z}$, respectively. 
\begin{align*}
& \vec{\hat{J}}_1^2 \psi_{j_1,m_1}^{(1)} = j_1 (j_1+1) \psi_{j_1,m_1}^{(1)} ~,  ~~~ \vec{\hat{J}}_2^2 \psi_{j_2,m_2}^{(2)} = j_2 (j_2+1) \psi_{j_2,m_2}^{(2)} ~, \\
& \hat{J}_{1z} \psi_{j_1,m_1}^{(1)} = m_1 \psi_{j_1,m_1}^{(1)} ~, ~~~ \hat{J}_{2z} \psi_{j_2,m_2}^{(2)} = m_2 \psi_{j_2,m_2}^{(2)} ~.
\end{align*}
The arguments $\vec{r}_1, t_1$ and $\vec{r}_2, t_2$ are here abbreviated by ``$1$" and ``$2$", respectively. However, these will generally be omitted in the following unless necessary for clearer understanding.

This way of coupling the angular momentum also arises in the theory of many-body problems, e.g. the two-electron problem, in which the single electrons can be described by the wave function $\psi_{j_1,m_1}(1)$ and $\psi_{j_1,m_1}(2)$. The total wave
function of the two-electron system is then given by $\psi_{j,m}(1,2)$, with a total angular momentum of j and its $2$-component $m$. But also in the two-electron problem there is a coupling between spin and orbital angular momentum to a total angular momentum and we have to use the technique which will be introduced now.

The eigenfunctions of the total angular momentum operators $\vec{\hat{J}}_2$ and $\hat{J}_z$ are denoted by $\psi_{j,m}(1,2)$. As Iong as there is no coupling between the two systems, the angular momenta $\vec{\hat{J}}_1, \vec{\hat{J}}_2$ and $\vec{\hat{J}}$ are fixed. The wave function $\psi$ separates into $\psi(1)$ and $\psi(2)$, and we can write $\psi_{jm}$ as a product of $\psi_{j_1 m_1} \times \psi_{j_2 m_2}$. If there is any coupling, $\psi_{jm}$ can always be described as a linear combination of products $\psi_{j_1 m_1} \times \psi_{j_2 m_2}$. We write the coefficients in the form $(j_1 j_2 j |m_1m_2m)$ in order to show the dependence on the various quantum numbers. These coefficients are called \textcolor{red}{\bf Clebsch-Gordan coefficients}.

The total wave function is
\begin{equation}
\psi_{jm} (1,2) = \sum_{m_1m_2} (j_1 j_2 j |m_1m_2m) \psi_{j_1 m_1}^{(1)} \psi_{j_2 m_2}^{(2)}  ~.
\label{eq:total_wavefunc}
\end{equation}
If there is coupling between the angular momenta, then $\psi_{j_1 m_1}$ and $\psi_{j_2 m_2}$ are no longer good eigenfunctions (which means that $m_1$ and $m_2$ are not conserved quantum numbers), since the constituent momenta are precessing around the total angular momentum. This is already expressed in the sum over $m_1$ and $m_2$ in (\ref{eq:total_wavefunc}). The relation (\ref{eq:total_wavefunc}) gives a transformation of the Hilbert space, spanned by the orthonormal vectors $\psi_{j_1 m_1}$ and $\psi_{j_2 m_2}$, to the new orthonormal basis set $\psi_{j m}$ of the same subspace. The total product space is invariant, but can be further decomposed, whereas the invariant subspaces, spanned by $\psi_{j m}$ for a fixed $j$, cannot be decomposed. We decompose, mathematically speaking, the representation of the rotation group generated by the product space into its irreducible parts. 

If $\psi_{j m}$ denotes an eigenfunction of the operators $\vec{\hat{J}}^2$ and $\hat{J}_z$, then conditions for the evaluation of the coefficients $(j_1 j_2 j | m_1 m_2 m)$ can be found, and we write
\begin{align*}
\hat{J}_z \psi_{j m} &= (\hat{J}_{1z} + \hat{J}_{2z}) \sum_{m_1, m_2} (j_1 j_2 j | m_1 m_2 m) \psi_{j_1 m_1} \psi_{j_2 m_2} \\
&= \sum_{m_1, m_2} (m_1 +  m_2 ) (j_1 j_2 j | m_1 m_2 m) \psi_{j_1 m_1} \psi_{j_2 m_2}  ~,
\end{align*}
and in the same way
\begin{equation}
\hat{J}_z \psi_{j m} = m \psi_{j m} \sum_{m_1, m_2} m (j_1 j_2 j | m_1 m_2 m) \psi_{j_1 m_1} \psi_{j_2 m_2} ~.
\end{equation}
The $m_1$ sum is performed over all values $- j_1 \leqslant m_1 \leqslant  j_1$, and analogously in the $m_2$ case over the interval $- j_2 \leqslant m_2 \leqslant  j_2$. By identifying the above relations, because of linear independence, the condition
\begin{equation}
(m-m_1 -m_2)( j_1 j_2 j|m_1 m_2 m ) = 0 ~.
\end{equation}
It follows that the Clebsch-Gordan coefficient vanishes if $m \neq m_1 + m_2$. This means that the double sum reduces to a single sum, since either the coefficient vanishes, or $m_2$ can be determined by $m_2 = m - m_1$. Equation becomes
\begin{equation}
 \psi_{j m} = \sum_{m_1} ( j_1 j_2 j|m_1 m_2 m ) \psi_{j_1, m_1} \psi_{j_2, m- m_1} ~.
\end{equation}
The conservation of angular momentum (more precisely: of the projection of the angular momentum on the quantization direction) is expressed by the relation $m_1 + m_2 = m$. In the next step we want to calculate the possible values of the quantum number $j$ defined by
\begin{equation}
\vec{\hat{J}}^2 \psi_{jm} = j(j+1)  \psi_{jm} ~.
\end{equation}

























\section{Evaluation of Clebsch-Gordan Coefficients}
\cite{greiner1994quantum} 































\section{Recursion Relations for Clebsch-Gordan Coefficients}
\cite{greiner1994quantum} The operator of total angular momentum is
\begin{equation*}
\vec{\hat{J}} = \vec{\hat{J}}_1 +\vec{\hat{J}}_2 ~,
\end{equation*}
with its spherical components
\begin{align*}
\hat{J}_\pm &= \hat{J}_{1\pm} +\hat{J}_{2\pm} = \hat{J}_x \pm i \hat{J}_y = (\hat{J}_{1x} +\hat{J}_{2x}) \pm i (\hat{J}_{1y} +\hat{J}_{2y}) ~, \\
\hat{J}_0 &= \hat{J}_{10} +\hat{J}_{20} = \hat{J}_z = (\hat{J}_{1z} +\hat{J}_{2z} ) ~.
\end{align*}
\begin{align*}
\hat{J}_+ |jm\rangle &= (\hat{J}_{1+} +\hat{J}_{2+}) |jm\rangle = (\hat{J}_{1+} +\hat{J}_{2+}) |m_1 m_2\rangle \langle m_1 m_2 |jm\rangle \\
&= [\hat{J}_{1+} |m_1 m_2\rangle + \hat{J}_{2+}|m_1 m_2\rangle ] \langle m_1 m_2 |jm\rangle
\end{align*}
\begin{align*}
& [j(j + 1) - m(m + 1)]^{1/2} |j,m + 1\rangle = \{ [j_1 (j_1 + 1) - m_1 (m_1 + 1)]^{1/2} | m_1 +1, m_2 \rangle \\
&+ [j_2(j_2 + 1) - m_2(m_2 + 1)]^{1/2} |m_1, m_2 + 1 \rangle \} \times \langle m_1 m_2 | jm \rangle ~.
\end{align*}
\begin{align*}
 & [j(j + 1) - m(m + 1)]^{1/2} |m_1 m_2\rangle \langle m_1 m_2 |j,m + 1\rangle  \\
 &= [j_1 (j_1 + 1) - m^\prime_1 (m^\prime_2 - 1)]^{1/2} | m^\prime_1 , m_2 \rangle \langle m^\prime_1 -1, m_2 | jm \rangle \\
 &+ [j_2 (j_2 + 1) - m^\prime_2 (m^\prime_2 - 1)]^{1/2} | m_1 , m^\prime_2 \rangle \langle m_1 , m^\prime_2 -1 | jm \rangle  ~.
\end{align*}
Here we have introduced $m^\prime_1 = m_1 + 1$ in the first term on the rhs, and in the second term $m^\prime_2 = m_2 + 1$. The sum goes,just as before, over $m^\prime_1$ from $- j_1$ to $j_1$, and over $m^\prime_2$ from $- j_2$ to $j_2$, the reason being the vanishing factor in front of the terms with $m_1 = j_1$ and $m_2 = j_2$. Thus the terms with $m^\prime_1 = j_1 + 1$ and $m^\prime_2 = j_2 + 1$ do not contribute. 

The terms with $m^\prime_1 = - j_1$ and $m^\prime_2 = -j_2$ belong to the null vectors $|-j_1 -1, m_2 \rangle$ and $|m_1, -j_2 -1 \rangle$ and, therefore, do not contribute to the result. Since $m^\prime_1$and $m^\prime_2$ are summation indices, they can be renamed $m_1$ for $m^\prime_1$ and $m_2$ for $m^\prime_2$.
\begin{align*}
& [j(j + 1) - m(m + 1)]^{1/2} | m_1 m_2 \rangle \langle m_1 m_2| j m + 1 \rangle \\
&= [j_1 (j_1 + 1) - m_1(m_1 - 1)]^{1/2} |m_1 m_2\rangle \langle m_1 - 1, m_2 | jm\rangle \\
&+ [j_2 (j_2 + 1) - m_2(m_2 - 1)]^{1/2} \langle m_1, m_2 - 1| jm \rangle ~,
\end{align*}
is obtained, Here $m_1$ and $m_2$ are fixed numbers and not summation indices. Repeating this procedure with $j_-$ gives the analogous result
\begin{align*}
& [j(j + 1) - m(m + 1)]^{1/2} \langle m_1 m_2| j m - 1 \rangle \\
&= [j_1 (j_1 + 1) - m_1(m_1 - 1)]^{1/2} \langle m_1 + 1, m_2 | jm\rangle \\
&+ [j_2 (j_2 + 1) - m_2(m_2 - 1)]^{1/2} \langle m_1, m_2 + 1| jm \rangle ~,
\end{align*}
This recursion relation allows us to derive the Clebsch-Gordan coefficients for the same total angular momenta $j$, having same $j_1$ and $j_2$ but different $m$.











\section{Explicit Calculation of Clebsch-Gordan Coefficients}
\cite{greiner1994quantum} 





















\section{Irreducible Representations of the Rotation Group}
\cite{greiner1994quantum} Angular momentum is a conserved quantity in central potentials, i.e. its eigenvalues can be used to classify the states. The commutation relations of the angular momentum operators represent the Lie algebra of SO(3).

Angular momentum is classically defined by the relation
\begin{equation}
\vec{L} = \vec{r} \times \vec{p} ~.
\end{equation}
If replacing the momentum variable by the operator
\begin{equation}
\hat{\vec{p}} = -i\hbar \nabla ~,
\end{equation}
the commutation relations is
\begin{equation}
[ \hat{L}_x,  \hat{L}_y] = i\hbar \hat{L}_z ~,
\label{eq:commu_rela}
\end{equation}
for the components of the angular momentum operator. If the total angular momentum $\vec{L}$ is the sum of angular momenta $\vec{L}^{(n)}$ of single systems (particles, spin, orbital angular momentum, etc.), the commutation relation holds for the sum as well. With
\begin{equation}
\vec{L} = \sum_n \vec{L}^{(n)} ~, ~~ [ \hat{L}_i^{(m)},  \hat{L}_j^{(n)}] = i\hbar \varepsilon_{ijk} \hat{L}_k^{(n)} \delta_{nm} ~,
\end{equation}
\begin{equation}
[ \hat{L}_x,  \hat{L}_y] = \left[\sum_n \hat{L}_x^{(n)}, \sum_m \hat{L}_y^{(n)} \right] = \sum_n [\hat{L}_x^{(n)}, \hat{L}_y^{(n)} ] = \sum_n i\hbar \hat{L}_z^{(n)} = i\hbar \hat{L}_z ~.
\end{equation}
The angular momentum operators belonging to different systems commute, as they act in different spaces, i.e. they act on
different coordinates. As the commutation relation (\ref{eq:commu_rela}) holds generally, define:

\textit{Any vector operator $\vec{\hat{J}}$, whose components are observables and satisfy the commutation relation (\ref{eq:commu_rela}), is called an angular momentum operator.}

Based only upon the commutation relation [the Lie algebra of SO(3)] we can infer several fundamental properties of the angular momentum operator and its eigenfunctions.

The square of the angular momentum commutes with all components, that is
\begin{equation}
[\vec{\hat{J}}, \vec{\hat{J}}^2] = 0 ~, ~~~  \vec{\hat{J}}^2 =  \hat{J}^2_x + \hat{J}^2_y +\hat{J}^2_z
\label{eq:J_J2}
\end{equation}
From (\ref{eq:J_J2}) we may deduce that one component of the angular momentum and its total square can be measured simultaneously, i.e. they have a common eigenfunction.

To obtain the spectrum of these operators it is appropriate to use the following Hermitian conjugate operators:
\begin{equation}
\hat{J}_+ = \hat{J}_x +i \hat{J}_y ~, ~~~ \hat{J}_- = \hat{J}_x -i \hat{J}_y ~.
\end{equation}
These operators, which are not Hermitian, are called \textcolor{red}{\bf step operators} or \textcolor{red}{\bf shift operators}.

From relation (\ref{eq:J_J2}), 
\begin{equation}
[\hat{J}_+ , \vec{\hat{J}}^2] = [\hat{J}_-,  \vec{\hat{J}}^2] = 0 ~.
\end{equation}
The three operators $\hat{J}_+, \hat{J}_-, \hat{J}_z$ determine the vector operator j entirely and are more convenient for algebraic transformations than the operators $\hat{J}_x, \hat{J}_y, \hat{J}_z$.



\section{Matrix Representations of Angular Momentum Operators}
\cite{greiner1994quantum} 







































%%%%%%%%%%%%%%%%%%%%%%%%%%%%%%%%%%%%%%%%%%%%%%%%%%%%%%%%%%%%%%%%%%%%%%
\bibliographystyle{unsrt_update}
\bibliography{ref}
%%%%%%%%%%%%%%%%%%%%%%%%%%%%%%%%%%%%%%%%%%%%%%%%%%%%%%%%%%%%%%%%%%%%%%


\end{document}