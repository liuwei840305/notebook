\documentclass[11pt,a4paper]{article}
%\usepackage{fontspec, xunicode, xltxtra}  
%\setmainfont{Hiragino Sans GB}  
%\usepackage{xeCJK}
%\setCJKmainfont[BoldFont=STZhongsong, ItalicFont=STKaiti]{STSong}
%\setCJKsansfont[BoldFont=STHeiti]{STXihei}
%\setCJKmonofont{STFangsong}

%使用Xelatex编译

% 设置页面
%==================================================
\linespread{2} %行距
% \usepackage[top=1in,bottom=1in,left=1.25in,right=1.25in]{geometry}
% \headsep=2cm
% \textwidth=16cm \textheight=24.2cm
%==================================================

% 其它需要使用的宏包
%==================================================
\usepackage[colorlinks,linkcolor=blue,anchorcolor=red,citecolor=green,urlcolor=blue]{hyperref} 
\usepackage{tabularx}
\usepackage{authblk}         % 作者信息
\usepackage{algorithm}     % 算法排版
\usepackage{amsmath}     % 数学符号与公式
\usepackage{amsfonts}     % 数学符号与字体
\usepackage{mathrsfs}      % 花体
\usepackage{amssymb}
\usepackage[framemethod=TikZ]{mdframed}

\usepackage{graphicx} 
\usepackage{graphics}
\usepackage{color}
\usepackage{xcolor}
\usepackage{tcolorbox}
\usepackage{lipsum}
\usepackage{empheq}

\usepackage{fancyhdr}       % 设置页眉页脚
\usepackage{fancyvrb}       % 抄录环境
\usepackage{float}              % 管理浮动体
\usepackage{geometry}     % 定制页面格式
\usepackage{hyperref}       % 为PDF文档创建超链接
\usepackage{lineno}          % 生成行号
\usepackage{listings}        % 插入程序源代码
\usepackage{multicol}       % 多栏排版
%\usepackage{natbib}         % 管理文献引用
\usepackage{rotating}       % 旋转文字,图形,表格
\usepackage{subfigure}    % 排版子图形
\usepackage{titlesec}       % 改变章节标题格式
\usepackage{moresize}   % 更多字体大小
\usepackage{anysize}
\usepackage{indentfirst}  % 首段缩进
\usepackage{booktabs}   % 使用\multicolumn
\usepackage{multirow}    % 使用\multirow

\usepackage{wrapfig}
\usepackage{titlesec}     % 改变标题样式
\usepackage{enumitem}
\usepackage{aas_macros}
\usepackage{bigints}

\renewcommand{\vec}[1]{\boldsymbol{#1}}
\newcommand{\me}{\mathrm{e}}
\newcommand{\mi}{\mathrm{i}}
\newcommand{\dif}{\mathrm{d}}
\newcommand{\tabincell}[2]{\begin{tabular}{@{}#1@{}}#2\end{tabular}}

\def\kpc{{\rm kpc}}
\def\km{{\rm km}}
\def\cm{{\rm cm}}
\def\TeV{{\rm TeV}}
\def\GeV{{\rm GeV}}
\def\MeV{{\rm MeV}}
\def\GV{{\rm GV}}
\def\MV{{\rm MV}}
\def\yr{{\rm yr}}
\def\s{{\rm s}}
\def\ns{{\rm ns}}
\def\GHz{{\rm GHz}}
\def\muGs{{\rm \mu Gs}}
\def\arcsec{{\rm arcsec}}
\def\K{{\rm K}}
\def\microK{\mu{\rm K}}
\def\sr{{\rm sr}}
\newcolumntype{p}{D{,}{\pm}{-1}}

\renewcommand{\figurename}{Fig.}
\renewcommand{\tablename}{Tab.}

\renewcommand{\arraystretch}{1.5}

\setlength{\parindent}{0pt}  %取消每段开头的空格

\newcounter{theo}[section]\setcounter{theo}{0}
\renewcommand{\thetheo}{\arabic{section}.\arabic{theo}}
\newenvironment{theo}[2][]{%
\refstepcounter{theo}%
\ifstrempty{#1}%
{\mdfsetup{%
frametitle={%
\tikz[baseline=(current bounding box.east),outer sep=0pt]
\node[anchor=east,rectangle,fill=blue!20]
{\strut Theorem~\thetheo};}}
}%
{\mdfsetup{%
frametitle={%
\tikz[baseline=(current bounding box.east),outer sep=0pt]
\node[anchor=east,rectangle,fill=blue!20]
{\strut Theorem~\thetheo:~#1};}}%
}%
\mdfsetup{innertopmargin=10pt,linecolor=blue!20,%
linewidth=2pt,topline=true,%
frametitleaboveskip=\dimexpr-\ht\strutbox\relax
}
\begin{mdframed}[]\relax%
\label{#2}}{\end{mdframed}}

\newcommand*\widefbox[1]{\fbox{\hspace{2em}#1\hspace{2em}}}


\title{The Quantum Theory of Radiation}
\author{}
\date{\today}
\begin{document}

\maketitle

\section{The Euler-Lagrange Equations}
\cite{2015lqm..book.....W} The \textcolor{red}{canonical variables $q_N(t)$ in general field theories are fields $\psi_n(\vec{x}, t)$}, for which $N$ is a compound index, including a discrete label $n$ indicating the type of field and a spatial coordinate $\vec{x}$. Correspondingly, the \textcolor{red}{Lagrangian $L(t)$ is a functional of $\psi_n(\vec{x},t)$ and $\dot{\psi}_n(\vec{x},t)$}, depending on the form of all of the functions $\psi_n(\vec{x},t)$ and $\dot{\psi}_n(\vec{x},t)$ for all $\vec{x}$, but at a \textcolor{red}{fixed time $t$}. In consequence, the partial derivatives with respect to $q_N$ and $\dot{q}_N$ in the equations of motion must be interpreted as \textcolor{red}{functional derivatives with respect to $\psi_n(\vec{x},t)$ and $\dot{\psi}_n(\vec{x},t)$}, so that these equations read
\begin{equation}
\dfrac{\partial}{\partial t} \left(\dfrac{\delta L(t)}{\delta \dot{\psi}_n(\vec{x},t)} \right) = \dfrac{\delta L(t)}{\delta \psi_n(\vec{x},t)}
\end{equation}
where the functional derivatives $\delta L /\delta \dot{\psi}_n$ and $\delta L /\delta \psi_n$ are defined so that the \textcolor{red}{change in the Lagrangian produced by independent infinitesimal changes $\delta \psi_n(\vec{x},t)$ and $\delta \dot{\psi}_n(\vec{x},t)$ in $\psi_n(\vec{x},t)$ and $\dot{\psi}_n(\vec{x},t)$ at a fixed time $t$} is
\begin{equation}
\delta L(t) = \sum_n \int \dif^3 x \left[\dfrac{\delta L(t)}{\delta \psi_n(\vec{x},t)} \delta \psi_n(\vec{x},t) +\dfrac{\delta L(t)}{\delta \dot{\psi}_n(\vec{x},t)} \delta \dot{\psi}_n(\vec{x},t) \right] ~.
\end{equation}
The canonical conjugate to $\psi_n(\vec{x},t)$ is
\begin{equation}
\pi_n(\vec{x},t) = \dfrac{\delta L(t)}{\delta \dot{\psi}_n(\vec{x},t)}  ~,
\end{equation}
and in a theory with no constraints, the canonical commutation relations are
\begin{align}
& [\psi_n(\vec{x},t), \pi_m(\vec{y},t)] = i\hbar \delta_{nm} \delta^3(\vec{x} -\vec{y}) ~, \\
& [\psi_n(\vec{x},t), \psi_m(\vec{y},t)] = [\pi_n(\vec{x},t), \pi_m(\vec{y},t)] = 0 ~. 
\end{align}
Typically (though not always), the Lagrangian in a field theory will be an integral of a \textcolor{red}{local Lagrangian density $\mathcal L$}:
\begin{equation}
L(t) = \int \dif^3 x \mathcal L \left(\psi(\vec{x},t), \nabla \psi(\vec{x},t), \dot{\psi}(\vec{x},t) \right) ~.
\end{equation}
The variation of the Lagrangian action due to infinitesimal changes in the $\psi_n$ and their space and time derivatives that vanish for $|\vec{x}| \rightarrow \infty$ is
\begin{equation*}
\delta L(t) = \int \dif^3 x \sum_n \left[\dfrac{\partial \mathcal L}{\partial \psi_n} \delta \psi_n +\sum_i \dfrac{\delta \mathcal L}{\partial (\partial_i \psi_n)} \dfrac{\partial}{\partial x_i}\delta \psi_n + \dfrac{\partial \mathcal L}{\delta \dot{\psi}_n} \dfrac{\partial}{\partial t} \delta \psi_n   \right] ~.
\end{equation*}
Integrating by parts, 
\begin{equation}
\delta L(t) = \int \dif^3 x \sum_n \left[\left( \dfrac{\partial \mathcal L}{\partial \psi_n} -\sum_i \dfrac{\partial}{\partial x_i} \dfrac{\partial \mathcal L}{\partial (\partial_i \psi_n)} \right) \delta \psi_n + \dfrac{\partial \mathcal L}{\delta \dot{\psi}_n} \dfrac{\partial}{\partial t} \delta \psi_n   \right] ~.
\end{equation}
This may be expressed as formulas for the variational derivatives of the Lagrangian
\begin{align}
& \dfrac{\delta L}{\delta \psi_n} = \dfrac{\partial \mathcal L}{\partial \psi_n} -\sum_i \dfrac{\partial}{\partial x_i} \dfrac{\partial \mathcal L}{\partial (\partial_i \psi_n)} ~, \\
& \dfrac{\delta L}{\delta \dot{\psi}_n} = \dfrac{\partial \mathcal L}{\partial \dot{\psi}_n} ~.
\end{align}
The Euler-Lagrange field equations are
\begin{equation}
\dfrac{\partial \mathcal L}{\partial \psi_n} -\sum_i \dfrac{\partial}{\partial x_i} \dfrac{\partial \mathcal L}{\partial (\partial_i \psi_n)} = \dfrac{\partial}{\partial t}\dfrac{\partial \mathcal L}{\partial \dot{\psi}_n} ~.
\end{equation}
In relativistically invariant theories,
\begin{equation}
\dfrac{\partial \mathcal L}{\partial \psi_n} = \sum_\mu \dfrac{\partial}{\partial x^\mu}\dfrac{\partial \mathcal L}{\partial (\partial_\mu \psi_n)} ~.
\end{equation}
In theories with a local Lagrangian density, the field variable that is canonically conjugate to $\psi_n(\vec{x}, t)$ is
\begin{equation}
\pi_n = \dfrac{\delta L}{\delta \dot{\psi}_n} = \dfrac{\partial \mathcal L}{\partial \dot{\psi}_n} ~.
\end{equation}





























































































%%%%%%%%%%%%%%%%%%%%%%%%%%%%%%%%%%%%%%%%%%%%%%%%%%%%%%%%%%%%%%%%%%%%%%
\bibliographystyle{unsrt_update}
\bibliography{ref}
%%%%%%%%%%%%%%%%%%%%%%%%%%%%%%%%%%%%%%%%%%%%%%%%%%%%%%%%%%%%%%%%%%%%%%


\end{document}