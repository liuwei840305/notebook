\documentclass[11pt,a4paper]{article}
%\usepackage{fontspec, xunicode, xltxtra}  
%\setmainfont{Hiragino Sans GB}  
\usepackage{xeCJK}
%\setCJKmainfont[BoldFont=STZhongsong, ItalicFont=STKaiti]{STSong}
%\setCJKsansfont[BoldFont=STHeiti]{STXihei}
%\setCJKmonofont{STFangsong}

%使用Xelatex编译

% 设置页面
%==================================================
\linespread{2} %行距
% \usepackage[top=1in,bottom=1in,left=1.25in,right=1.25in]{geometry}
% \headsep=2cm
% \textwidth=16cm \textheight=24.2cm
%==================================================

% 其它需要使用的宏包
%==================================================
\usepackage[colorlinks,linkcolor=blue,anchorcolor=red,citecolor=green,urlcolor=blue]{hyperref} 
\usepackage{tabularx}
\usepackage{authblk}         % 作者信息
\usepackage{algorithm}     % 算法排版
\usepackage{amsmath}     % 数学符号与公式
\usepackage{amsfonts}     % 数学符号与字体
\usepackage{mathrsfs}      % 花体
\usepackage{amssymb}
\usepackage[framemethod=TikZ]{mdframed}

\usepackage{graphicx} 
\usepackage{graphics}
\usepackage{color}
\usepackage{xcolor}
\usepackage{tcolorbox}
\usepackage{lipsum}
\usepackage{empheq}

\usepackage{fancyhdr}       % 设置页眉页脚
\usepackage{fancyvrb}       % 抄录环境
\usepackage{float}              % 管理浮动体
\usepackage{geometry}     % 定制页面格式
\usepackage{hyperref}       % 为PDF文档创建超链接
\usepackage{lineno}          % 生成行号
\usepackage{listings}        % 插入程序源代码
\usepackage{multicol}       % 多栏排版
%\usepackage{natbib}         % 管理文献引用
\usepackage{rotating}       % 旋转文字,图形,表格
\usepackage{subfigure}    % 排版子图形
\usepackage{titlesec}       % 改变章节标题格式
\usepackage{moresize}   % 更多字体大小
\usepackage{anysize}
\usepackage{indentfirst}  % 首段缩进
\usepackage{booktabs}   % 使用\multicolumn
\usepackage{multirow}    % 使用\multirow

\usepackage{wrapfig}
\usepackage{titlesec}     % 改变标题样式
\usepackage{enumitem}
\usepackage{aas_macros}
\usepackage{bigints}

\renewcommand{\vec}[1]{\boldsymbol{#1}}
\newcommand{\me}{\mathrm{e}}
\newcommand{\mi}{\mathrm{i}}
\newcommand{\dif}{\mathrm{d}}
\newcommand{\tabincell}[2]{\begin{tabular}{@{}#1@{}}#2\end{tabular}}

\def\kpc{{\rm kpc}}
\def\km{{\rm km}}
\def\cm{{\rm cm}}
\def\TeV{{\rm TeV}}
\def\GeV{{\rm GeV}}
\def\MeV{{\rm MeV}}
\def\GV{{\rm GV}}
\def\MV{{\rm MV}}
\def\yr{{\rm yr}}
\def\s{{\rm s}}
\def\ns{{\rm ns}}
\def\GHz{{\rm GHz}}
\def\muGs{{\rm \mu Gs}}
\def\arcsec{{\rm arcsec}}
\def\K{{\rm K}}
\def\microK{\mu{\rm K}}
\def\sr{{\rm sr}}
\newcolumntype{p}{D{,}{\pm}{-1}}

\renewcommand{\figurename}{Fig.}
\renewcommand{\tablename}{Tab.}

\renewcommand{\arraystretch}{1.5}

\setlength{\parindent}{0pt}  %取消每段开头的空格

\newcounter{theo}[section]\setcounter{theo}{0}
\renewcommand{\thetheo}{\arabic{section}.\arabic{theo}}
\newenvironment{theo}[2][]{%
\refstepcounter{theo}%
\ifstrempty{#1}%
{\mdfsetup{%
frametitle={%
\tikz[baseline=(current bounding box.east),outer sep=0pt]
\node[anchor=east,rectangle,fill=blue!20]
{\strut Theorem~\thetheo};}}
}%
{\mdfsetup{%
frametitle={%
\tikz[baseline=(current bounding box.east),outer sep=0pt]
\node[anchor=east,rectangle,fill=blue!20]
{\strut Theorem~\thetheo:~#1};}}%
}%
\mdfsetup{innertopmargin=10pt,linecolor=blue!20,%
linewidth=2pt,topline=true,%
frametitleaboveskip=\dimexpr-\ht\strutbox\relax
}
\begin{mdframed}[]\relax%
\label{#2}}{\end{mdframed}}

\newcommand*\widefbox[1]{\fbox{\hspace{2em}#1\hspace{2em}}}


\title{Foundation}
\author{}
\date{\today}
\begin{document}

\maketitle

\section{量子力学的假定}
\cite{1986qmv1.book.....C} 对一个物理体系的经典描述可归结如下
\begin{enumerate}[label=(\roman*)]
\item 体系在确定时刻$t_0$的态,决定于$N$个广义坐标$q_i(t_0)$和$N$个共轭动量$p_i(t_0)$的数值;
\item 如果知道了体系在指定时刻的态,各物理量在该时刻的值便完全确定。即知道了体系在时刻$t_0$的态,就可以确切地预言在该时刻进行的任何一种测量的结果;
\item 体系的态随时间演变由Hamilton-Jacobi方程组来确定。只要给定在指定时刻$t_0$的函数值$\{q_i(t_0), p_i(t_0)\}$,此方程组的解$\{q_i(t), p_i(t)\}$就是唯一的。只要知道了体系的初态,便可以确定它在任意时刻的态。
\end{enumerate}

1. 体系的态的描述

用一个平方可积波函数来描述粒子在指定时刻的态。用态空间$\mathscr{E}_r$中的一个右矢和每一个波函数联系起来。给出$\mathscr{E}_r$空间中的右矢$|\psi \rangle$等价于给出对应的波函数$\psi(\vec{r}) = \langle \vec{r}|\psi\rangle$。一个粒子在确定时刻的量子态可由$\mathscr{E}_r$空间中的一个右矢来描述。
\begin{tcolorbox}[colback=green!15,colframe=green!40!black,title= ]
假定一:在确定的时刻$t_0$,一个物理体系的态由态空间$\mathscr{E}$中一个特定的右矢$|\psi(t_0) \rangle$来确定。
\end{tcolorbox}

由于$\mathscr{E}$是一个矢量空间,该假定隐含着叠加原理:若干态矢量的线性组合也是一个态矢量。

2. 物理量的描述
\begin{tcolorbox}[colback=green!15,colframe=green!40!black,title= ]
假定二:每一个可以测量的物理量$\mathscr{A}$都可以用在$\mathscr{E}$空间中起作用的一个算符$A$来描述。这个算符是一个观察算符。
\end{tcolorbox}

$A$应当是观察算符。

与经典力学对比,量子力学是以不同的方式来描述体系的态及有关物理量的:\textcolor{red}{态}用\textcolor{red}{矢量}来表示,\textcolor{red}{物理量}用\textcolor{red}{算符}表示。

3. 物理量的测量

a. 可能的结果
\begin{tcolorbox}[colback=green!15,colframe=green!40!black,title= ]
假定三:每次测量物理量$\mathscr{A}$,可能得到的结果,只能是对应的观察算符$A$的本征值之一。
\end{tcolorbox}

$A$是厄密算符,所以测量$\mathscr{A}$所得的结果总是实数。

若$A$的谱是分立的,测量$\mathscr{A}$可能得到的结果是量子化的。

b. 谱分解原理

考虑一个体系,它在指定时刻的态由右矢$|\psi\rangle$描述,假设这个右矢已归一化为$1$:
\begin{equation}
\langle \psi |\psi\rangle = 1
\end{equation}
想要预言在该时刻测量体系的物理量$\mathscr{A}$(它与观察算符$A$相联系)所得的结果。

$\alpha$. 分立谱的情况

假设$A$的谱是分立谱。若$A$的全体本征值$a_n$都是非简并的,与一个本征值相联系的本征矢只有一个(除相位因子以外),即$|u_n\rangle$
\begin{equation}
A|u_n\rangle = a_n |u_n \rangle
\end{equation}
由于$A$是观察算符,故已归一化的$|u_n\rangle$的集合构成$\mathscr{E}$中的一个正交归一基,态矢量$ |\psi\rangle$写作
\begin{equation}
|\psi\rangle = \sum_n c_n |u_n \rangle
\end{equation}
假定:测量$\mathscr{A}$时得到结果$a_n$的几率$\mathscr{P}(a_n)$是
\begin{equation}
\mathscr{P}(a_n) = |c_n|^2 = |\langle u_n |\psi\rangle|^2
\end{equation}

\begin{tcolorbox}[colback=green!15,colframe=green!40!black,title= ]
假定四:(非简并的离散谱)若体系处于已归一化的态$|\psi \rangle$中,则测量物理量$\mathscr{A}$得到的结果为对应观察算符$A$的非简并本征值$a_n$的概率$\mathscr{P}(a_n)$是
\begin{equation*}
\mathscr{P}(a_n) = |\langle u_n|\psi \rangle|^2
\end{equation*}
$|u_n \rangle$是的已归一化的本征矢,属于本征值$a_n$。
\end{tcolorbox}
若某些本征值$a_n$是简并的,与之对应的正交归一本征矢$|u_n \rangle$就有若干个:
\begin{equation}
A|u_n^i \rangle = a_n |u_n^i \rangle ~, ~~ i = 1, 2, \cdots, g_n
\end{equation}
$|\psi \rangle$仍然可以按正交归一基$\{|u_n^i \rangle \}$展开,
\begin{equation}
|\psi \rangle = \sum_n \sum_{i=1}^{g_n} c_n^i |u_n^i \rangle
\end{equation}
几率$\mathscr{P}(a_n)$
\begin{equation}
\mathscr{P}(a_n) = \sum_{i=1}^{g_n} |c_n^i|^2 = \sum_{i=1}^{g_n} |\langle u_n^i |\psi \rangle|^2
\end{equation}
\begin{tcolorbox}[colback=green!15,colframe=green!40!black,title= ]
假定四:(离散谱)若体系处于已归一化的态$|\psi \rangle$中,则测量物理量$\mathscr{A}$得到的结果为对应观察算符$A$的本征值$a_n$的概率$\mathscr{P}(a_n)$是
\begin{equation*}
\mathscr{P}(a_n) = \sum_{i=1}^{g_n} |\langle u_n^i|\psi \rangle|^2
\end{equation*}
$g_n$是$a_n$的简并度,$\{|u^i_n \rangle\} (i = 1, 2, \cdots, g_n)$是一组正交归一矢量,它们在对应于$A$的本征值$a_n$的本征子空间空间$\mathscr{E}_n$中构成一个基。
\end{tcolorbox}
该假定要有意义,在$a_n$有简并时,几率$\mathscr{P}(a_n)$必须与$\mathscr{E}$中基$\{|u_n^i \rangle \}$的选择无关。

$\beta$. 连续谱的情况

假定$A$的谱是连续的,且假设并没有简并。$A$的广义上已正交归一化的本征矢集$|v_a \rangle$
\begin{equation}
A |v_a \rangle = a |v_a \rangle
\end{equation}
构成$\mathscr{E}$空间中的一个连续基。在这个基中可将任意右矢$|\psi \rangle$分解为
\begin{equation}
|\psi \rangle = \int \dif a~ c(a) |v_a \rangle
\end{equation}
由于测量$\mathscr{A}$的可能结果构成一个连续集合定义几率密度:测量$\mathscr{A}$的值介于$a$和$a+\dif a$之间的几率是
\begin{equation*}
\dif \mathscr{P}(a_n) = \rho(a) \dif a
\end{equation*}
其中
\begin{equation*}
\rho(a) = |c(a)|^2 = |\langle v_a | \psi \rangle|^2
\end{equation*}

\begin{tcolorbox}[colback=green!15,colframe=green!40!black,title= ]
假定四:(非简并连续谱)测量处于已归一化的态$|\psi \rangle$的体系的物理量$\mathscr{A}$时,得到介于$a$和$a +\dif a$之间结果的概率$\dif \mathscr{P}(a)$是
\begin{equation*}
\dif \mathscr{P}(a) = |\langle v_a|\psi \rangle|^2 \dif a
\end{equation*}
$|v_a \rangle$是与$\mathscr{A}$相联系的观察算符$A$的本征矢,属于本征值$a$。
\end{tcolorbox}


$\gamma$. 重要后果

考虑两个右矢$|\psi \rangle$和$|\psi^\prime \rangle$
\begin{equation}
|\psi^\prime \rangle = e^{i\theta} |\psi \rangle
\end{equation}
$\theta$为实数。若$|\psi \rangle$是归一化的,则$|\psi^\prime \rangle$也是归一化的
\begin{equation}
\langle \psi^\prime |\psi^\prime \rangle = \langle \psi | e^{-i\theta} e^{i\theta}| \psi \rangle = \langle \psi | \psi \rangle
\end{equation}


互成比例的两个态矢量表示同一个物理状态。



总的相位因子对于物理预言没有影响,但展开式中各系数的相对相位则是有影响的。

c. 波包的收缩


\begin{tcolorbox}[colback=green!15,colframe=green!40!black,title= ]
假定五:如果处于态$|\psi \rangle$的体系测量物理量$\mathscr{A}$得到的结果是$a_n$,则刚测量之后体系的态是$|\psi \rangle$在属于$a_n$的本征子空间上的归一化的投影$\dfrac{P_n|\psi \rangle}{\sqrt{\langle \psi|P_n|\psi \rangle}}$。
\end{tcolorbox}


4. 体系随时间的演变
\begin{tcolorbox}[colback=green!15,colframe=green!40!black,title= ]
假定六:态矢量$|\psi(t) \rangle$随时间的演变遵从薛定谔方程
\begin{equation*}
i\hbar \dfrac{\dif }{\dif t} |\psi(t) \rangle = H(t) |\psi(t) \rangle
\end{equation*}
$H(t)$是与体系的总能量相联系的观察算符。$H$叫做体系的哈密顿算符。
\end{tcolorbox}

5. 量子化规则

对于经典力学中已定义的物理量$\mathscr{A}$,怎样构成在量子力学中描述该物理量的算符$A$。

a. 规则的陈述

考虑处在标量势场中的一个无自旋粒子构成的体系,
\begin{tcolorbox}[colback=green!15,colframe=green!40!black,title= ]
与粒子的位置$\vec{r}(x, y, z)$相联系的是观察算符$\vec{R}(X, Y, Z)$。\\
与粒子的动量$\vec{p}(p_x, p_y, p_z)$相联系的是观察算符$\vec{P}(P_x, P_y, P_z)$。
\end{tcolorbox}
$\vec{R}$和$\vec{P}$的诸分量满足正则对易关系:
\begin{align}
& [R_i, R_j] = [P_i, P_j] = 0 \\
& [R_i, P_j] = i\hbar \delta_{ij}
\end{align}
粒子的任何一个物理量$\mathscr{A}$都可以表示为基本力学量$\vec{r}$和$\vec{p}$的函数:$\mathscr{A}(\vec{r}, \vec{p}, t)$。要得到对应的观察算符$A$,在$\mathscr{A}(\vec{r}, \vec{p}, t)$的表达式中,将变量$\vec{r}$和$\vec{p}$换成观察算符$\vec{R}$和$\vec{P}$:
\begin{equation}
A(t) = \mathscr{A}(\vec{R}, \vec{P}, t) ~.
\end{equation}

\begin{equation}
\vec{r} \cdot \vec{p} = xp_x + yp_y +zp_z = \vec{p} \cdot \vec{r} = p_x x + p_y y +p_z z
\end{equation}
在经典力学中标量积$\vec{r} \cdot \vec{p}$是可以对易的。但是若将$\vec{r}$和$\vec{p}$换成对应的观察算符$\vec{R}$和$\vec{P}$,
\begin{equation}
\vec{R} \cdot \vec{P} \neq \vec{P} \cdot \vec{R} 
\end{equation}
此外,$\vec{R} \cdot \vec{P}$和$\vec{P} \cdot \vec{R} $都不是厄密算符

对称化规则。和$\vec{r} \cdot \vec{p}$相联系的观察算符是
\begin{equation}
\dfrac{1}{2} (\vec{R} \cdot \vec{P} +\vec{P} \cdot \vec{R})
\end{equation}

\begin{tcolorbox}[colback=green!15,colframe=green!40!black,title= ]
要得到描述一个已有经典定义的物理量$\mathscr{A}$的观察算符$A$,只需在$\mathscr{A}$的经过适当对称化的表达式中,将$\vec{r}$和$\vec{p}$分别换成观察算符$\vec{R}$和$\vec{P}$。
\end{tcolorbox}






























































%%%%%%%%%%%%%%%%%%%%%%%%%%%%%%%%%%%%%%%%%%%%%%%%%%%%%%%%%%%%%%%%%%%%%%
\bibliographystyle{unsrt_update}
\bibliography{ref}
%%%%%%%%%%%%%%%%%%%%%%%%%%%%%%%%%%%%%%%%%%%%%%%%%%%%%%%%%%%%%%%%%%%%%%


\end{document}