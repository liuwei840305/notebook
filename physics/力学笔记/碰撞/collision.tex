\documentclass[11pt,a4paper]{article}
%\usepackage{fontspec, xunicode, xltxtra}  
%\setmainfont{Hiragino Sans GB}  
\usepackage{xeCJK}
%\setCJKmainfont[BoldFont=STZhongsong, ItalicFont=STKaiti]{STSong}
%\setCJKsansfont[BoldFont=STHeiti]{STXihei}
%\setCJKmonofont{STFangsong}

%使用Xelatex编译

% 设置页面
%==================================================
\linespread{2} %行距
% \usepackage[top=1in,bottom=1in,left=1.25in,right=1.25in]{geometry}
% \headsep=2cm
% \textwidth=16cm \textheight=24.2cm
%==================================================

% 其它需要使用的宏包
%==================================================
\usepackage[colorlinks,linkcolor=blue,anchorcolor=red,citecolor=green,urlcolor=blue]{hyperref} 
\usepackage{tabularx}
\usepackage{authblk}         % 作者信息
\usepackage{algorithm}     % 算法排版
\usepackage{amsmath}     % 数学符号与公式
\usepackage{amsfonts}     % 数学符号与字体
\usepackage{mathrsfs}      % 花体
\usepackage{graphics}
\usepackage{color}
\usepackage{fancyhdr}       % 设置页眉页脚
\usepackage{fancyvrb}       % 抄录环境
\usepackage{float}              % 管理浮动体
\usepackage{geometry}     % 定制页面格式
\usepackage{hyperref}       % 为PDF文档创建超链接
\usepackage{lineno}          % 生成行号
\usepackage{listings}        % 插入程序源代码
\usepackage{multicol}       % 多栏排版
\usepackage{natbib}         % 管理文献引用
\usepackage{rotating}       % 旋转文字,图形,表格
\usepackage{subfigure}    % 排版子图形
\usepackage{titlesec}       % 改变章节标题格式
\usepackage{moresize}   % 更多字体大小
\usepackage{anysize}
\usepackage{indentfirst}  % 首段缩进
\usepackage{booktabs}   % 使用\multicolumn
\usepackage{multirow}    % 使用\multirow
\usepackage{graphicx} 
\usepackage{wrapfig}
\usepackage{xcolor}
\usepackage{titlesec}     % 改变标题样式
\usepackage{enumitem}

\newcommand{\myvec}[1]%
   {\stackrel{\raisebox{-2pt}[0pt][0pt]{\small$\rightharpoonup$}}{#1}}  %矢量符号
\renewcommand{\vec}[1]{\boldsymbol{#1}}
\newcommand{\me}{\mathrm{e}}
\newcommand{\mi}{\mathrm{i}}
\newcommand{\dif}{\mathrm{d}}
\newcommand{\tabincell}[2]{\begin{tabular}{@{}#1@{}}#2\end{tabular}}

\def\kpc{{\rm kpc}}
\def\km{{\rm km}}
\def\cm{{\rm cm}}
\def\TeV{{\rm TeV}}
\def\GeV{{\rm GeV}}
\def\MeV{{\rm MeV}}
\def\GV{{\rm GV}}
\def\MV{{\rm MV}}
\def\yr{{\rm yr}}
\def\s{{\rm s}}
\def\ns{{\rm ns}}
\def\GHz{{\rm GHz}}
\def\muGs{{\rm \mu Gs}}
\def\arcsec{{\rm arcsec}}
\def\K{{\rm K}}
\def\microK{\mu{\rm K}}
\def\sr{{\rm sr}}
\newcolumntype{p}{D{,}{\pm}{-1}}

\renewcommand{\figurename}{Fig.}
\renewcommand{\tablename}{Tab.}

\renewcommand{\arraystretch}{1.5}

\setlength{\parindent}{0pt}  %取消每段开头的空格

\title{质点碰撞}
\author{}
\date{\today}
\begin{document}

\maketitle

微观粒子的碰撞又称\textcolor{red}{散射};

两物体的碰撞,通常发生在比较短暂的时间内,外界对这两物体或无作用,或虽有力的作用,但只要作用力是有限的,其冲量就可以忽略,体系的动量就是守恒的;
\section{正碰}
碰撞前两物体速度$\vec{u}_1$、$\vec{u}_2$均沿两球中心连线;

碰后两球的速度$\vec{v}_1$、$\vec{v}_2$和$\vec{u}_1$、$\vec{u}_2$在同一直线上;

取两球中心连线为坐标轴,以$\vec{u}_1$的方向为轴的正方向,设两球的质量分别为$m_1$、$m_2$,
\begin{equation}
m_1 u_1 + m_2 u_2 = m_1 v_1 + m_2 v_2
\end{equation}

\subsection{弹性碰撞}
在碰撞过程中没有机械能损失的碰撞;
\begin{equation}
\frac{1}{2} m_1 u_1^2 +\frac{1}{2} m_2 u_2^2 = \frac{1}{2} m_1 v_1^2 +\frac{1}{2} m_2 v_2^2
\end{equation}

\begin{eqnarray}
\nonumber v_1 &=& \frac{m_1 -m_2}{m_1 +m_2} u_1 +\frac{2m_2}{m_1 +m_2} u_2 ~,\\
v_2 &=& \frac{2m_1}{m_1 +m_2} u_1 -\frac{m_1 -m_2}{m_1 +m_2} u_2 ~,
\end{eqnarray}

1. $m_1 = m_2$

$v_1 = u_2$,$v_2 = u_1$,即\textcolor{red}{碰后两球速度互相变换};

2. $u_2 = 0$,即\textcolor{red}{受碰球原先静止};

\begin{eqnarray}
\nonumber v_1 &=& \frac{m_1 -m_2}{m_1 +m_2} u_1 ~, \\
v_2 &=& \frac{2m_1}{m_1 +m_2} u_1 ~,
\end{eqnarray}
若\textcolor{red}{$m_1 \ll m_2$},则$v_1 = -u_1$,$v_2 \approx 0$,即碰后,\textcolor{red}{大球几乎仍保持静止},\textcolor{red}{小球以相等的速率返回};

若\textcolor{red}{$m_1 = m_2$},则$v_1 = 0$,$v_2 = u_1$,即碰后,两球运动速度交换;

若\textcolor{red}{$m_1 \gg m_2$},则$v_1 \approx u_1$,$v_2 \approx 2u_1$,即\textcolor{red}{大球几乎以原来速度继续前进},\textcolor{red}{而小球以两倍于大球的速度前进};

$m_2$所得到的动能$\Delta E_k$与碰前$m_1$的动能$E_{k1}$之比:
\begin{eqnarray}
\nonumber \frac{\Delta E_k}{E_{k1}} &=& \frac{4m_1 m_2}{(m_1 +m_2)^2} ~, \\
&=& \frac{4m_1/m_2}{(1+m_1/m_2)^2}
\end{eqnarray}
$m_1$与$m_2$越接近,$m_2$获得的能量越多;


\subsection{非弹性碰撞}
碰撞过程中形变不能完全消失,机械能不守恒,其中一部分转化为内能;

\subsubsection{完全非弹性碰撞}
设形变是范性的,一旦发生形变就永久保持,完全不能消失;只有压缩阶段,没有恢复阶段,两球达到相同速度后即停止,尔后两球以共同速度前进;

动量仍然守恒
\begin{equation}
m_1 u_1 + m_2 u_2 = (m_1 +m_2) v
\end{equation}
但机械能不再守恒,损失的机械能为
\begin{eqnarray}
\nonumber \Delta E &=& \frac{1}{2} m_1 u_1^2 +\frac{1}{2} m_2 u_2^2 -\frac{1}{2}(m_1 +m_2)v^2 ~,\\
&=& \frac{1}{2} \frac{m_1 m_2(u_1-u_2)^2}{(m_1 +m_2)}
\end{eqnarray}

当$u_2 = 0$时,
\begin{eqnarray}
\Delta E = \frac{1}{2} m_1 u_1^2 \frac{m_2}{m_1 +m_2} = E_{k1} \frac{m_2}{m_1 +m_2}
\end{eqnarray}
从而
\begin{equation}
\frac{\Delta E_k}{E_{k1}} = \frac{1}{1+m_1/m_2}
\end{equation}
为了在碰撞中,获得大的激发能或电离能,要求撞击粒子原来的能量大,且撞击粒子与被撞击粒子相比质量必须要小;

撞击粒子的动能不可能完全转化为激发能或电离能,因为在碰撞前后,质心的速度不变,与质心相联系的那部分动能即质心动能不可能转变成其他形式的能量;只有相当于质心的动能,才可能转变为其他形式的能量,称为\textcolor{red}{资用能};使质心动能尽量小;

\subsubsection{一般非弹性碰撞}
定义\textcolor{red}{恢复系数}:
\begin{equation}
e = \frac{v_2 -v_1}{u_1 -u_2}
\end{equation}
弹性碰撞中,
\begin{equation}
\frac{v_2 -v_1}{u_1 -u_2} = 1
\end{equation}
完全非弹性碰撞中,
\begin{equation}
\frac{v_2 -v_1}{u_1 -u_2} = 0
\end{equation}
恢复系数只与两球的质料有关,与碰前速度无关;

对一般非弹性碰撞,
\begin{equation}
0 < e < 1
\end{equation}

\begin{eqnarray}
\nonumber v_1 = \frac{m_1 -em_2}{m_1 +m_2}u_1 +\frac{(1+e)m_2}{m_1 +m_2} u_2 ~, \\
v_2 = \frac{(1+e)m_1}{m_1 +m_2} u_1 - \frac{em_1 -m_2}{m_1 +m_2}u_2 ~,
\end{eqnarray}

碰撞中损失的机械能为
\begin{equation}
\Delta E = \frac{1}{2}(1-e^2) \frac{m_1m_2}{m_1+m_2} (u_1 -u_2)^2
\end{equation}


\section{斜碰}
碰撞前两球的速度$\vec{u}_1$、$\vec{u}_2$不在两球中心连线上;一般情况,斜碰为三维问题,碰撞后的速度$\vec{v}_1$、$\vec{v}_2$不一定在$\vec{u}_1$、$\vec{u}_2$组成的平面上;

设弹性碰撞前一个小球处在静止状态,即$\vec{u}_2 = 0$,
\begin{eqnarray}
\nonumber m_1 \vec{u}_1 &=& m_1 \vec{v}_1 +  m_2 \vec{v}_2 ~,\\
\frac{1}{2} m_1 u_1^2 &=& \frac{1}{2} m_1 v_1^2 +\frac{1}{2} m_2 v_2^2 ~,
\end{eqnarray}
取$\vec{u}_1$方向为$x$轴,碰撞所在面为$x-y$面,
\begin{eqnarray}
m_1 u_1 &=& m_1 v_1 \cos \theta_1 +m_2 v_2 \cos \theta_2 ~, \\
0 &=& m_1 v_1 \sin \theta_1 -m_2 v_2 \sin \theta_2 ~, 
\end{eqnarray}
$\theta_1$、$\theta_2$称为散射角;

碰撞结果与碰前两小球中心在$y$方向的距离$b$有关,$b$称为\textcolor{red}{碰撞参数};$b = 0$时为正碰;

\section{碰撞与质心坐标系}
碰撞过程中,质心系仍是惯性系;

设$\vec{u}_2 = 0$,

\subsection{正碰}
\begin{equation}
v_C = \frac{m_1}{m_1 +m_2} u_1 ~,
\end{equation}
质心系中,碰前两球的速度为
\begin{eqnarray}
\nonumber u_1^{\prime} &=& u_1 -v_C = \frac{m_2}{m_1+m_2} u_1 ~, \\
u_2^{\prime} &=& u_2 -v_C = -\frac{m_1}{m_1 +m_2} u_1 ~, 
\end{eqnarray}
从而
\begin{equation}
m_1 u_1^{\prime} + m_2 u_2^{\prime} = 0
\end{equation}
即质心系中,体系的动量为$0$,

碰后,
\begin{equation}
m_1 v_1^{\prime} + m_2 v_2^{\prime} = 0
\end{equation}
质心系中碰撞前后体系动能仍相等,
\begin{eqnarray}
\nonumber v_1^{\prime} &=& -u_1^{\prime}  ~, \\
v_2^{\prime}  &=& -u_2^{\prime}  ~,
\end{eqnarray}
\textcolor{red}{在质心系中,粒子的速率在碰撞前后保持不变,但粒子速度反向};

回到实验室坐标系:
\begin{eqnarray}
v_1 &=& v_1^{\prime}  +v_C = \frac{m_1-m_2}{m_1+m_2} u_1 ~, \\
v_2 &=& v_2^{\prime}  +v_C =  \frac{2m_1}{m_1+m_2} u_1 ~, 
\end{eqnarray}

\subsection{斜碰}
质心速度
\begin{equation}
\vec{v}_C = \frac{m_1}{m_1 +m_2} \vec{u}_1
\end{equation}
质心系中,两质点碰前速度
\begin{eqnarray}
\nonumber \vec{u}_1^{\prime} &=& \vec{u}_1 -\vec{v}_C = \frac{m_2}{m_1+m_2} \vec{u}_1 ~, \\
 \vec{u}_2^{\prime} &=& \vec{u}_2 -\vec{v}_C = -\frac{m_1}{m_1+m_2} \vec{u}_1 ~,
\end{eqnarray}
碰撞前后,质心系中体系的动量为$0$,
\begin{equation}
m_1 \vec{u}_1^{\prime} + m_2 \vec{u}_2^{\prime} = m_1 \vec{v}_1^{\prime} + m_2 \vec{v}_2^{\prime} = 0
\end{equation}
\textcolor{red}{$\vec{v}_1^{\prime}$和$\vec{v}_2^{\prime}$仍在一直线上};且碰撞前后体系动能相等,
\begin{equation}
|\vec{v}^{\prime}_1| = |\vec{u}^{\prime}_1| ~, |\vec{v}^{\prime}_2| = |\vec{u}^{\prime}_2|
\end{equation}
质心系中,二粒子碰撞后,速度只改变方向,不改变大小,碰后两速度仍在一直线上,但直线的方位改变了;用粒子入射方向和出射方向的夹角$\Theta$表示粒子运动方向改变的程度,其值在$0 \sim \pi$之间,与碰撞参量有关;

回到实验室坐标系,
\begin{eqnarray}
\nonumber \vec{v}_1 &=& \vec{v}_1^{\prime}  +\vec{v}_C  ~, \\
\vec{v}_2 &=& \vec{v}_2^{\prime}  +\vec{v}_C  ~, 
\end{eqnarray}
由于
\begin{eqnarray}
\nonumber |\vec{v}^{\prime}_1| &=& |\vec{u}^{\prime}_1| = \frac{m_2}{m_1 +m_2}u_1 ~, \\ 
\nonumber |\vec{v}^{\prime}_2| &=& |\vec{u}^{\prime}_2| = \frac{m_1}{m_1+m_2} u_1 ~, \\
 |\vec{v}_C| &=& |\vec{u}^{\prime}_2| = \frac{m_1}{m_1+m_2} u_1 ~, 
\end{eqnarray}
当\textcolor{red}{$m_1 < m_2$}时,
\begin{equation}
|\vec{v}_C| < |\vec{v}_1^{\prime}| ~,
\end{equation}
$m_1$的散射角$\theta_1$取\textcolor{red}{$0 \sim \pi$};
当\textcolor{red}{$m_1 > m_2$}时,
\begin{equation}
|\vec{v}_C| > |\vec{v}_1^{\prime}| ~,
\end{equation}
$m_1$的散射角$\theta_1$\textcolor{red}{不可能大于}
\begin{equation}
\theta_{1 \text{max}} = \text{arcsin} \frac{m_2}{m_1} ~;
\end{equation}



\section{Compton散射}
\begin{eqnarray}
\nonumber E_{\lambda} + E_0 &=& E_{\lambda^{\prime}} +E ~, \\
\vec{p}_{\lambda} &=& \vec{p}_{\lambda^{\prime}} +\vec{p}
\end{eqnarray}
$\vec{p}_{\lambda}$和$\vec{p}_{\lambda}^{\prime}$分别表示光子碰撞前后的动量;且
\begin{eqnarray}
\nonumber E_{\lambda} &=& h\nu ~, \\
\nonumber E_0 &=& m_e c^2 ~, \\
\nonumber E &=& \gamma m_{e} c^2 ~, \\
E^2 &=& E_0^2 +p^2c^2 ~, 
\end{eqnarray}
平方后可得
\begin{equation}
p_{\lambda}^2 +p_{\lambda^{\prime}}^2 -2 p_{\lambda} p_{\lambda^{\prime}} \cos \theta = p^2
\end{equation}
\textcolor{red}{Compton散射公式}
\begin{equation}
\lambda^{\prime} -\lambda = \Delta \lambda = \frac{h}{m_e c} (1-\cos \theta)
\end{equation}
散射光子的能量
\begin{equation}
h\nu^{\prime} = \frac{h\nu}{1+\kappa (1-\cos \theta)} ~, ~\kappa = \frac{h\nu}{m_e c^2} ~, 
\end{equation}
反冲电子动能
\begin{equation}
E_k = h\nu -h\nu^{\prime} = h\nu \frac{\kappa(1-\cos \theta)}{1+\kappa (1-\cos \theta)}
\end{equation}
Compton散射引起的最大位移($\theta = \pi$)
\begin{equation}
\Delta \lambda = \frac{2h}{m_e c}  = 0.0049 ~~\text{nm}
\end{equation}
反冲电子的最大动能
\begin{equation}
E_{k, \text{max}} = h\nu \frac{2\kappa}{1+2\kappa} ~,
\end{equation}
相应光子的最小能量
\begin{equation}
E_{\lambda^{\prime}} = (h\nu^{\prime})_{\text{min}} = \frac{h\nu}{1+2\kappa} ~,
\end{equation}
电子的Compton波长
\begin{equation}
\lambda = \frac{hc}{m_e c^2} = \frac{1.24 ~\text{nm}\cdot \text{keV}}{511 ~\text{keV}} = 0.002426 ~\text{nm} ~,
\end{equation}
经典电子半径
\begin{equation}
m_e c^2 = \frac{e^2}{4\pi \epsilon_0 r_e} ~, \Longrightarrow r_e = \frac{e^2}{4\pi \epsilon_0 m_e c^2} \approx 2.8 ~\text{fm} ~,
\end{equation}





















































\end{document}