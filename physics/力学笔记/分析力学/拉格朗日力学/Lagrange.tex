\documentclass[11pt,a4paper]{article}
%\usepackage{fontspec, xunicode, xltxtra}  
%\setmainfont{Hiragino Sans GB}  
\usepackage{xeCJK}
%\setCJKmainfont[BoldFont=STZhongsong, ItalicFont=STKaiti]{STSong}
%\setCJKsansfont[BoldFont=STHeiti]{STXihei}
%\setCJKmonofont{STFangsong}

%使用Xelatex编译

% 设置页面
%==================================================
\linespread{2} %行距
% \usepackage[top=1in,bottom=1in,left=1.25in,right=1.25in]{geometry}
% \headsep=2cm
% \textwidth=16cm \textheight=24.2cm
%==================================================

% 其它需要使用的宏包
%==================================================
\usepackage[colorlinks,linkcolor=blue,anchorcolor=red,citecolor=green,urlcolor=blue]{hyperref} 
\usepackage{tabularx}
\usepackage{authblk}         % 作者信息
\usepackage{algorithm}     % 算法排版
\usepackage{amsmath}     % 数学符号与公式
\usepackage{amsfonts}     % 数学符号与字体
\usepackage{mathrsfs}      % 花体
\usepackage{amssymb}
\usepackage[framemethod=TikZ]{mdframed}

\usepackage{graphicx} 
\usepackage{graphics}
\usepackage{color}
\usepackage{xcolor}
\usepackage{tcolorbox}
\usepackage{lipsum}
\usepackage{empheq}

\usepackage{fancyhdr}       % 设置页眉页脚
\usepackage{fancyvrb}       % 抄录环境
\usepackage{float}              % 管理浮动体
\usepackage{geometry}     % 定制页面格式
\usepackage{hyperref}       % 为PDF文档创建超链接
\usepackage{lineno}          % 生成行号
\usepackage{listings}        % 插入程序源代码
\usepackage{multicol}       % 多栏排版
%\usepackage{natbib}         % 管理文献引用
\usepackage{rotating}       % 旋转文字,图形,表格
\usepackage{subfigure}    % 排版子图形
\usepackage{titlesec}       % 改变章节标题格式
\usepackage{moresize}   % 更多字体大小
\usepackage{anysize}
\usepackage{indentfirst}  % 首段缩进
\usepackage{booktabs}   % 使用\multicolumn
\usepackage{multirow}    % 使用\multirow

\usepackage{wrapfig}
\usepackage{titlesec}     % 改变标题样式
\usepackage{enumitem}
\usepackage{aas_macros}
\usepackage{bigints}
\usepackage{extarrows}

\newcommand{\myvec}[1]%
   {\stackrel{\raisebox{-2pt}[0pt][0pt]{\small$\rightharpoonup$}}{#1}}  %矢量符号
\renewcommand{\vec}[1]{\boldsymbol{#1}}
\newcommand{\me}{\mathrm{e}}
\newcommand{\mi}{\mathrm{i}}
\newcommand{\dif}{\mathrm{d}}
\newcommand{\tabincell}[2]{\begin{tabular}{@{}#1@{}}#2\end{tabular}}

\def\kpc{{\rm kpc}}
\def\km{{\rm km}}
\def\cm{{\rm cm}}
\def\TeV{{\rm TeV}}
\def\GeV{{\rm GeV}}
\def\MeV{{\rm MeV}}
\def\GV{{\rm GV}}
\def\MV{{\rm MV}}
\def\yr{{\rm yr}}
\def\s{{\rm s}}
\def\ns{{\rm ns}}
\def\GHz{{\rm GHz}}
\def\muGs{{\rm \mu Gs}}
\def\arcsec{{\rm arcsec}}
\def\K{{\rm K}}
\def\microK{\mu{\rm K}}
\def\sr{{\rm sr}}
\newcolumntype{p}{D{,}{\pm}{-1}}

\renewcommand{\figurename}{Fig.}
\renewcommand{\tablename}{Tab.}

\renewcommand{\arraystretch}{1.5}

\setlength{\parindent}{0pt}  %取消每段开头的空格

\newcounter{theo}[section]\setcounter{theo}{0}
\renewcommand{\thetheo}{\arabic{section}.\arabic{theo}}
\newenvironment{theo}[2][]{%
\refstepcounter{theo}%
\ifstrempty{#1}%
{\mdfsetup{%
frametitle={%
\tikz[baseline=(current bounding box.east),outer sep=0pt]
\node[anchor=east,rectangle,fill=blue!20]
{\strut Theorem~\thetheo};}}
}%
{\mdfsetup{%
frametitle={%
\tikz[baseline=(current bounding box.east),outer sep=0pt]
\node[anchor=east,rectangle,fill=blue!20]
{\strut Theorem~\thetheo:~#1};}}%
}%
\mdfsetup{innertopmargin=10pt,linecolor=blue!20,%
linewidth=2pt,topline=true,%
frametitleaboveskip=\dimexpr-\ht\strutbox\relax
}
\begin{mdframed}[]\relax%
\label{#2}}{\end{mdframed}}

\newcommand*\widefbox[1]{\fbox{\hspace{2em}#1\hspace{2em}}}

\title{Lagrange力学}
\author{}
\date{\today}
\begin{document}

\maketitle


系统的\textcolor{red}{自由度}

唯一确定系统位置所需独立变量的数目;对完整系统;

对非完整系统的自由度不能这样定义

质点在空间的位置由径矢$\myvec{r}$确定;

质点的速度,
\begin{equation}
\myvec{v} = \frac{\dif \myvec{r}}{\dif t}
\end{equation}
质点的加速度,
\begin{equation}
\myvec{a} = \frac{\dif \myvec{v}}{\dif t} = \frac{\dif^2 \myvec{r}}{\dif t^2}
\end{equation}

任意$s$个可以完全刻画系统($s$个自由度)位置的变量$q_1, q_2, \cdots, q_s$,称为该系统的\textcolor{red}{广义坐标},其导数称为\textcolor{red}{广义速度}。

加速度与坐标、速度的关系式,运动方程

\textcolor{red}{最小作用量原理(Hamilton原理)}

描述每一个力学系统都可以用一个相应的函数$\mathscr{L}(q_1, q_2, \cdots, q_s, \dot{q}_1, \dot{q}_2, \cdots, \dot{q}_s; t)$或者$\mathscr{L}(q, \dot{q}, t)$;

在时刻$t=t_1$和$t=t_2$系统的位置由两个坐标和确定。系统在这两个位置之间的运动使得积分
\begin{equation}
\mathcal{S} = \int_{t_1}^{t_2} \mathscr{L}(q, \dot{q}, t) \dif t
\end{equation}
取最小值。

$\mathscr{L}$:给定系统的\textcolor{red}{拉格朗日(Lagrange)函数};

$\mathcal{S}$:\textcolor{red}{作用量}。

Lagrange函数$\mathscr{L}$可以附加任意一个关于时间和坐标的函数的全导数。

\textcolor{red}{Eular-Lagrange方程}
\begin{equation}
\frac{\dif }{\dif t} \frac{\partial \mathscr{L}}{\partial \dot{q}_i} -\frac{\partial \mathscr{L}}{\partial q_i} = 0, ~~ (i = 1, 2, 3, \cdots)
\end{equation}


\textcolor{red}{约束}


\section{Generalized Coordinates}







\section{D'Alembert Principle}
A virtual displacement $\delta \vec{r}$ is an infinitesimal displacement of the system that is compatible with the constraints. Contrary to the case of a real infinitesimal displacement $\dif \vec{r}$, in a virtual displacement the forces and constraints acting on the system do not change. A virtual displacement will be characterized by the symbol $\delta$, a real displacement by $\dif$. Mathematically we operate with the element $\delta$ just as with a differential. 




called the \textcolor{red}{principle of virtual work}. 





\textcolor{red}{d'Alembert principle}


\section{Lagrange Equations}








\textcolor{red}{Eular-Lagrange方程}
\begin{equation}
\frac{\dif }{\dif t} \frac{\partial \mathscr{L}}{\partial \dot{q}_i} -\frac{\partial \mathscr{L}}{\partial q_i} = 0, ~~ (i = 1, 2, 3, \cdots)
\end{equation}







\subsection{Lagrange Equation for Nonholonomic Constraints}
For systems with holonomic constraints, the dependent coordinates can be eliminated by introducing generalized coordinates.







\section{Special Problems}

\subsection{Velocity-DependentPotentials}




\subsection{Nonconservative Forces and Dissipation Function (Friction Function)}




\subsection{Nonholonomic Systems and Lagrange Multipliers}


%%%%%%%%%%%%%%%%%%%%%%%%%%%%%%%%%%%%%%%%%%%%%%%%%%%%%%%%%%%%%%%%%%%%%%
\bibliographystyle{unsrt_update}
\bibliography{ref}
%%%%%%%%%%%%%%%%%%%%%%%%%%%%%%%%%%%%%%%%%%%%%%%%%%%%%%%%%%%%%%%%%%%%%%


\end{document}