\documentclass[12pt,a4paper]{article}
%\usepackage{fontspec, xunicode, xltxtra}  
%\setmainfont{Hiragino Sans GB}  
\usepackage{xeCJK}
%\setCJKmainfont[BoldFont=STZhongsong, ItalicFont=STKaiti]{STSong}
%\setCJKsansfont[BoldFont=STHeiti]{STXihei}
%\setCJKmonofont{STFangsong}

%使用Xelatex编译

% 设置页面
%==================================================
\linespread{2} %行距
% \usepackage[top=1in,bottom=1in,left=1.25in,right=1.25in]{geometry}
% \headsep=2cm
% \textwidth=16cm \textheight=24.2cm
%==================================================

% 其它需要使用的宏包
%==================================================
\usepackage[colorlinks,linkcolor=blue,anchorcolor=red,citecolor=green,urlcolor=blue]{hyperref} 
\usepackage{tabularx}
\usepackage{authblk}         % 作者信息
\usepackage{algorithm}     % 算法排版
\usepackage{amsmath}     % 数学符号与公式
\usepackage{amsfonts}     % 数学符号与字体
\usepackage{mathrsfs}      % 花体
\usepackage{graphics}
\usepackage{color}
\usepackage{fancyhdr}       % 设置页眉页脚
\usepackage{fancyvrb}       % 抄录环境
\usepackage{float}              % 管理浮动体
\usepackage{geometry}     % 定制页面格式
\usepackage{hyperref}       % 为PDF文档创建超链接
\usepackage{lineno}          % 生成行号
\usepackage{listings}        % 插入程序源代码
\usepackage{multicol}       % 多栏排版
\usepackage{natbib}         % 管理文献引用
\usepackage{rotating}       % 旋转文字,图形,表格
\usepackage{subfigure}    % 排版子图形
\usepackage{titlesec}       % 改变章节标题格式
\usepackage{moresize}   % 更多字体大小
\usepackage{anysize}
\usepackage{indentfirst}  % 首段缩进
\usepackage{booktabs}   % 使用\multicolumn
\usepackage{multirow}    % 使用\multirow
\usepackage{graphicx} 
\usepackage{wrapfig}
\usepackage{xcolor}
\usepackage{titlesec}     % 改变标题样式
\usepackage{enumitem}

\newcommand{\myvec}[1]%
   {\stackrel{\raisebox{-2pt}[0pt][0pt]{\small$\rightharpoonup$}}{#1}}  %矢量符号
\renewcommand{\vec}[1]{\boldsymbol{#1}}
\newcommand{\me}{\mathrm{e}}
\newcommand{\mi}{\mathrm{i}}
\newcommand{\dif}{\mathrm{d}}
\newcommand{\tabincell}[2]{\begin{tabular}{@{}#1@{}}#2\end{tabular}}

\def\kpc{{\rm kpc}}
\def\km{{\rm km}}
\def\cm{{\rm cm}}
\def\TeV{{\rm TeV}}
\def\GeV{{\rm GeV}}
\def\MeV{{\rm MeV}}
\def\GV{{\rm GV}}
\def\MV{{\rm MV}}
\def\yr{{\rm yr}}
\def\s{{\rm s}}
\def\ns{{\rm ns}}
\def\GHz{{\rm GHz}}
\def\muGs{{\rm \mu Gs}}
\def\arcsec{{\rm arcsec}}
\def\K{{\rm K}}
\def\microK{\mu{\rm K}}
\def\sr{{\rm sr}}
\newcolumntype{p}{D{,}{\pm}{-1}}

\renewcommand{\figurename}{Fig.}
\renewcommand{\tablename}{Tab.}

\renewcommand{\arraystretch}{1.5}

\setlength{\parindent}{0pt}  %取消每段开头的空格

\title{电磁波的发射}
\author{}
\date{\today}
\begin{document}

\maketitle
\section{Scalar and vector potential}

\subsection{Gauge transformation}
\subsubsection{gauge freedom}

\subsubsection{Coulomb Gauge}


\subsubsection{Lorenz Gauge}

\section{Retarded potential}
\begin{eqnarray}
\phi(\vec{r}, t) &=& \frac{1}{4\pi \epsilon_0} \int \frac{\rho(\vec{r}^{\prime}, t_r)}{R} \dif \tau^{\prime} \\
\vec{A}(\vec{r}, t) &=& \frac{\mu_0}{4\pi \epsilon_0} \int \frac{\vec{J}(\vec{r}^{\prime}, t_r)}{R} \dif \tau^{\prime}
\end{eqnarray}

\subsection{Jefimenko's equations}
\begin{eqnarray}
\vec{E}(\vec{r}, t) &=& \frac{1}{4\pi \epsilon_0} \int \left[\frac{\rho(\vec{r}^{\prime}, t_r)}{R^2}\vec{\hat{R}} +\frac{\dot\rho(\vec{r}^{\prime}, t_r)}{cR}\vec{\hat{R}} -\frac{\vec{\dot J}(\vec{r}^{\prime}, t_r)}{c^2 R}  \right] \dif \tau^{\prime} \\
\vec{B}(\vec{r}, t) &=& \frac{\mu_0}{4\pi} \int \left[\frac{\vec{J}(\vec{r}^{\prime}, t_r)}{R^2} +\frac{\vec{\dot J}(\vec{r}^{\prime}, t_r)}{c R} \right] \times \vec{\hat R} \dif \tau^{\prime}
\end{eqnarray}

\section{Li\'{e}nard-Wiechert Potential}
The Li\'{e}nard-Wiechert potential for a moving point charge
\begin{eqnarray}
\phi(\vec{r}, t) &=& \frac{1}{4\pi \epsilon_0} \frac{qc}{(1-\vec{n}\cdot\vec{\beta}) R} \\ 
\vec{A}(\vec{r}, t) &=& \frac{\mu_0}{4\pi} \frac{qc \vec{v}}{(1-\vec{n}\cdot\vec{\beta}) R} = \frac{\vec{v}}{c^2} \phi(\vec{r}, t)
\end{eqnarray}

\section{Fields of a moving point charge}
The field of a point charge $q$ in arbitrary motion in SI unit
\begin{eqnarray}
\vec{E}(\vec{r}, t) &=& \frac{q}{4\pi \epsilon_0} \left[\frac{\vec{n} -\vec{\beta}}{\gamma^2 (1-\vec{n}\cdot\vec{\beta})^3 R^2} \right]_{\rm ret} +\frac{q}{4\pi \epsilon_0 c} \left[\frac{\vec{n} \times \{(\vec{n} -\vec{\beta}) \times \vec{\dot\beta} \}}{(1-\vec{n}\cdot\vec{\beta})^3 R} \right]_{\rm ret} \\
\vec{B}  &=& \frac{1}{c}[\vec{n} \times \vec{E} ]_{\rm ret}
\end{eqnarray}
and in Gaussian (CGS) units
\begin{eqnarray}
\vec{E}(\vec{r}, t) &=& q\left[\frac{\vec{n} -\vec{\beta}}{\gamma^2 (1-\vec{n}\cdot\vec{\beta})^3 R^2} \right]_{\rm ret} +\frac{q}{c} \left[\frac{\vec{n} \times \{(\vec{n} -\vec{\beta}) \times \vec{\dot\beta} \}}{(1-\vec{n}\cdot\vec{\beta})^3 R} \right]_{\rm ret} \\
\vec{B}  &=& [\vec{n} \times \vec{E} ]_{\rm ret}
\end{eqnarray}
The first term is \textcolor{red}{velocity field} and the second is \textcolor{red}{acceleration field}.

The \textcolor{red}{Poynting vector} is
\begin{equation}
\vec{S} = \frac{1}{\mu_0} (\vec{E} \times \vec{B}) = \frac{1}{\mu_0 c} [\vec{E} \times (\vec{n} \times \vec{E})] = \frac{1}{\mu_0 c} [E^2 \vec{n} -(\vec{n}\cdot \vec{E})\vec{E}]
\end{equation}













\end{document}