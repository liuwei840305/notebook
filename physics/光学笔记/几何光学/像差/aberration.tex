\documentclass[12pt,a4paper]{article}
%\usepackage{fontspec, xunicode, xltxtra}
%\setmainfont{Hiragino Sans GB}
\usepackage{xeCJK}
%\setCJKmainfont[BoldFont=STZhongsong, ItalicFont=STKaiti]{STSong}
%\setCJKsansfont[BoldFont=STHeiti]{STXihei}
%\setCJKmonofont{STFangsong}

%使用Xelatex编译

% 设置页面
%==================================================
\linespread{2} %行距
% \usepackage[top=1in,bottom=1in,left=1.25in,right=1.25in]{geometry}
% \headsep=2cm
% \textwidth=16cm \textheight=24.2cm
%==================================================

% 其它需要使用的宏包
%==================================================
\usepackage[colorlinks,linkcolor=blue,anchorcolor=red,citecolor=green,urlcolor=blue]{hyperref}
\usepackage{tabularx}
\usepackage{authblk}         % 作者信息
\usepackage{algorithm}     % 算法排版
\usepackage{amsmath}     % 数学符号与公式
\usepackage{amsfonts}     % 数学符号与字体
\usepackage{mathrsfs}      % 花体
\usepackage{graphics}
\usepackage{color}
\usepackage{fancyhdr}       % 设置页眉页脚
\usepackage{fancyvrb}       % 抄录环境
\usepackage{float}              % 管理浮动体
\usepackage{geometry}     % 定制页面格式
\usepackage{hyperref}       % 为PDF文档创建超链接
\usepackage{lineno}          % 生成行号
\usepackage{listings}        % 插入程序源代码
\usepackage{multicol}       % 多栏排版
\usepackage{natbib}         % 管理文献引用
\usepackage{rotating}       % 旋转文字,图形,表格
\usepackage{subfigure}    % 排版子图形
\usepackage{titlesec}       % 改变章节标题格式
\usepackage{moresize}   % 更多字体大小
\usepackage{anysize}
\usepackage{indentfirst}  % 首段缩进
\usepackage{booktabs}   % 使用\multicolumn
\usepackage{multirow}    % 使用\multirow
\usepackage{graphicx}
\usepackage{wrapfig}
\usepackage{xcolor}
\usepackage{titlesec}     % 改变标题样式
\usepackage{enumitem}
%\usepackage[dvipsnames]{color}


\renewcommand{\vec}[1]{\boldsymbol{#1}}
\newcommand{\me}{\mathrm{e}}
\newcommand{\mi}{\mathrm{i}}
\newcommand{\dif}{\mathrm{d}}
\newcommand{\tabincell}[2]{\begin{tabular}{@{}#1@{}}#2\end{tabular}}

\def\kpc{{\rm kpc}}
\def\km{{\rm km}}
\def\cm{{\rm cm}}
\def\TeV{{\rm TeV}}
\def\GeV{{\rm GeV}}
\def\MeV{{\rm MeV}}
\def\GV{{\rm GV}}
\def\MV{{\rm MV}}
\def\yr{{\rm yr}}
\def\s{{\rm s}}
\def\ns{{\rm ns}}
\def\GHz{{\rm GHz}}
\def\muGs{{\rm \mu Gs}}
\def\arcsec{{\rm arcsec}}
\def\K{{\rm K}}
\def\microK{\mu{\rm K}}
\def\sr{{\rm sr}}
\newcolumntype{p}{D{,}{\pm}{-1}}

\renewcommand{\figurename}{Fig.}
\renewcommand{\tablename}{Tab.}

\renewcommand{\arraystretch}{1.5}

\setlength{\parindent}{0pt}  %取消每段开头的空格

\title{像差}
\author{}
\date{\today}
\begin{document}

\maketitle

共轴光具组:\\
1. 物方每点发出的同心光束在像方仍保持为同心光束;\\
2. 垂直于光轴的物平面上各点的像仍在垂直于光轴的一个平面上;\\
3. 在每个像平面内横向放大率是常数,保持物、像之间的几何相似性;

共轴球面组满足不了这些要求,

任何偏离理想成像的现象,称为\textcolor{red}{像差};

在\textcolor{orange}{傍轴条件}下,理想成像是能近似实现的,

傍轴条件要求\textcolor{orange}{成像光束的孔径小}、\textcolor{orange}{仪器的视场小}

像差分为:\\
1. 单色像差 \\
i. 球面像差;\\
ii. 慧形像差;\\
iii. 像散;\\
iv. 像场弯曲(场曲);\\
v. 畸变;


2. 色像差 \\
由色散引起

\begin{equation}
\sin \theta = \theta -\frac{\theta^3}{3!} +\frac{\theta^5}{5!} +\cdots
\end{equation}


赛德耳系数:\\
把$\theta^3/3!$考虑进去,计算从物点发出的每一根光线的横向像差,即该光线经光具组后与理想像平面交点的位置偏离理想点的距离,得到表达式有五项,每一项的系数为赛德耳系数

初级像差、三级像差理论

光线追迹法:严格按照几何光学三定律计算每根光线的折射或反射,求出它们对理想像点的偏离;

几何像差

若存在衍射效应,波动光学

\section{球面像差、球差}
\textcolor{red}{孔径较大}时,由光轴上一物点发出的光束经球面折射后\textcolor{red}{不再交于一点};

球差大小与\textcolor{orange}{光线的孔径}有关

孔径:由\textcolor{orange}{孔径角$u$}或\textcolor{orange}{光线在折射面上的高度$h$}表征;

\textcolor{red}{配曲法}

用配曲法不可能将一个透镜的球差完全消除;凸透镜的球差是正的,凹透镜的球差是负的,可以把凸凹透镜粘合起来,组成一个复合透镜,可使某个高度$h$上的球差完全抵消;透镜有一定厚度时,不能同时在任何高度上消除球差,但可使剩余的球差减到比单透镜小得多的程度;


\section{慧形像差、慧差}
\textcolor{red}{傍轴物点}发出的\textcolor{red}{宽阔光束}经光具组后在像平面上不再交于一点,而是\textcolor{red}{形成状如彗星的亮斑};

在光瞳上做一系列同心圆$1, 2, 3, 4, \cdots$,经过各个圆周的光线在像平面上仍将落在一系列圆周$1', 2', 3', 4', \cdots$上,但是这些圆不再是同心的,半径越大的圆,其中心离$P'$越远;

用\textcolor{purple}{配曲法}可消除单个透镜的慧差,也可利用\textcolor{purple}{粘合透镜}消除慧差;

消球差和消慧差的条件往往不一致,二者不容易同时消除;



\section{正弦条件、齐明点}


\begin{equation}
ny\sin u = n' y' \sin u'
\end{equation}
\textcolor{red}{阿贝正弦条件},是在轴上以消球差的前提下,傍轴物点以大孔径的光束成像的充分必要条件;

\textcolor{red}{齐明点}:\\
光轴上已消除球差且满足阿贝正弦条件的共轭点;


N.B. 比较
阿贝正弦条件
\begin{equation}
ny\sin u = n' y' \sin u'
\end{equation}
和
亥姆霍兹公式
\begin{equation}
ny\tan u = n' y' \tan u'
\end{equation}


\section{像散、像场弯曲}
二者都由于物点离光轴较远、光束倾斜度较大引起;


像散:\\
出射光束的截面一般呈\textcolor{blue}{椭圆形},但\textcolor{blue}{在两处退化为直线},称为\textcolor{red}{焦散线};两焦散线互相垂直,分别称为\textcolor{red}{子午焦线}和\textcolor{red}{弧矢焦线};在两焦线之间的某个地方光束的截面呈圆形,称为\textcolor{red}{明晰圈},可以认为这里是光束聚焦最清晰的地方,是放置照相底片或屏幕的最佳位置;


像场弯曲:\\
对于物平面上的所有点,焦散线和明晰圈的轨迹一般是个\textcolor{blue}{曲面};

单个透镜,像场弯曲可通过在透镜前适当位置放一个光阑来矫正;像散需通过复杂的透镜组来消除;


\section{畸变}
由于光束的倾斜度较大引起的,并不破坏光束的同心性,不影响像的清晰度;

像平面内图形各部分与原物不成比例;

放在物平面内的方格,若\textcolor{cyan}{远光轴区域}的放大率比\textcolor{cyan}{光轴附近}\textcolor{red}{大},在像平面内产生\textcolor{red}{枕形畸变};若\textcolor{cyan}{远光轴区域}的放大率比\textcolor{cyan}{光轴附近}\textcolor{red}{小},在像平面内产生\textcolor{red}{桶形畸变}:

产生哪种畸变,与\textcolor{red}{孔径光阑}的位置有关;


如对凸透镜,光阑放在前面,产生桶形畸变;放在后面,产生枕形畸变;

对称镜头,将光阑放在一对相同的透镜中间,使两种相反的畸变互相抵消;

\section{色像差}
不同颜色的光所成的像,位置和大小都可能不同,前者称为\textcolor{red}{位置色差}(\textcolor{red}{轴向色差});后者称为\textcolor{red}{放大率色差}(\textcolor{red}{横向色差});






目镜

惠更斯目镜


冉斯登目镜

\end{document}
