\documentclass[12pt,a4paper]{article}
%\usepackage{fontspec, xunicode, xltxtra}  
%\setmainfont{Hiragino Sans GB}  
%\usepackage{xeCJK}
%\setCJKmainfont[BoldFont=STZhongsong, ItalicFont=STKaiti]{STSong}
%\setCJKsansfont[BoldFont=STHeiti]{STXihei}
%\setCJKmonofont{STFangsong}

%使用Xelatex编译

% 设置页面
%==================================================
\linespread{2} %行距
% \usepackage[top=1in,bottom=1in,left=1.25in,right=1.25in]{geometry}
% \headsep=2cm
% \textwidth=16cm \textheight=24.2cm
%==================================================

% 其它需要使用的宏包
%==================================================
\usepackage[colorlinks,linkcolor=blue,anchorcolor=red,citecolor=green,urlcolor=blue]{hyperref} 
\usepackage{tabularx}
\usepackage{authblk}         % 作者信息
\usepackage{algorithm}     % 算法排版
\usepackage{amsmath}     % 数学符号与公式
\usepackage{amsfonts}     % 数学符号与字体
\usepackage{mathrsfs}      % 花体
\usepackage{amssymb}
\usepackage[framemethod=TikZ]{mdframed}

\usepackage{graphicx} 
\usepackage{graphics}
\usepackage{color}
\usepackage{xcolor}
\usepackage{tcolorbox}
\usepackage{lipsum}
\usepackage{empheq}

\usepackage{fancyhdr}       % 设置页眉页脚
\usepackage{fancyvrb}       % 抄录环境
\usepackage{float}              % 管理浮动体
\usepackage{geometry}     % 定制页面格式
\usepackage{hyperref}       % 为PDF文档创建超链接
\usepackage{lineno}          % 生成行号
\usepackage{listings}        % 插入程序源代码
\usepackage{multicol}       % 多栏排版
%\usepackage{natbib}         % 管理文献引用
\usepackage{rotating}       % 旋转文字,图形,表格
\usepackage{subfigure}    % 排版子图形
\usepackage{titlesec}       % 改变章节标题格式
\usepackage{moresize}   % 更多字体大小
\usepackage{anysize}
\usepackage{indentfirst}  % 首段缩进
\usepackage{booktabs}   % 使用\multicolumn
\usepackage{multirow}    % 使用\multirow
\usepackage{wrapfig}
\usepackage{enumitem}
\usepackage{harpoon}   %矢量符号

\usepackage{aas_macros}

\newcommand{\myvec}[1]%
   {\stackrel{\raisebox{-2pt}[0pt][0pt]{\small$\rightharpoonup$}}{#1}}  %矢量符号
\renewcommand{\vec}[1]{\boldsymbol{#1}}
\newcommand{\me}{\mathrm{e}}
\newcommand{\mi}{\mathrm{i}}
\newcommand{\dif}{\mathrm{d}}
\newcommand{\tabincell}[2]{\begin{tabular}{@{}#1@{}}#2\end{tabular}}

\def\kpc{{\rm kpc}}
\def\km{{\rm km}}
\def\cm{{\rm cm}}
\def\TeV{{\rm TeV}}
\def\GeV{{\rm GeV}}
\def\MeV{{\rm MeV}}
\def\GV{{\rm GV}}
\def\MV{{\rm MV}}
\def\yr{{\rm yr}}
\def\s{{\rm s}}
\def\ns{{\rm ns}}
\def\GHz{{\rm GHz}}
\def\muGs{{\rm \mu Gs}}
\def\arcsec{{\rm arcsec}}
\def\K{{\rm K}}
\def\microK{\mu{\rm K}}
\def\sr{{\rm sr}}
\newcolumntype{p}{D{,}{\pm}{-1}}

\renewcommand{\figurename}{Fig.}
\renewcommand{\tablename}{Tab.}

\renewcommand{\arraystretch}{1.5}

\setlength{\parindent}{0pt}  %取消每段开头的空格

\newcounter{theo}[section]\setcounter{theo}{0}
\renewcommand{\thetheo}{\arabic{section}.\arabic{theo}}
\newenvironment{theo}[2][]{%
\refstepcounter{theo}%
\ifstrempty{#1}%
{\mdfsetup{%
frametitle={%
\tikz[baseline=(current bounding box.east),outer sep=0pt]
\node[anchor=east,rectangle,fill=blue!20]
{\strut Theorem~\thetheo};}}
}%
{\mdfsetup{%
frametitle={%
\tikz[baseline=(current bounding box.east),outer sep=0pt]
\node[anchor=east,rectangle,fill=blue!20]
{\strut Theorem~\thetheo:~#1};}}%
}%
\mdfsetup{innertopmargin=10pt,linecolor=blue!20,%
linewidth=2pt,topline=true,%
frametitleaboveskip=\dimexpr-\ht\strutbox\relax
}
\begin{mdframed}[]\relax%
\label{#2}}{\end{mdframed}}

\newcommand*\widefbox[1]{\fbox{\hspace{2em}#1\hspace{2em}}}

%从右倾的积分号变为竖直的积分号
%define a new command for \rm font of int
\DeclareSymbolFont{rmlargesymbols}{OMX}{mdbch}{m}{n}
% or \DeclareSymbolFont{rmlargesymbols}{U}{euex}{m}{n}
\DeclareMathSymbol{\rmintop}{\mathop}{rmlargesymbols}{82}
\newcommand{\rmint}{\rmintop\nolimits}

\title{CHEBYSHEV POLYNOMIALS}
\author{}
\date{\today}
\begin{document}

\maketitle
\cite{riley2006mathematical}
The Chebyshev equation
\begin{equation}
(1-x^2) y^{\prime \prime} -xy^\prime +n^2 y = 0 ~,
\end{equation}
can be converted to an equation of Sturm-Liouville form by multiplying by the integrating factor $(1 - x^2)^{-1/2}$.
\begin{equation}
\left[(1-x^2)^{1/2} y^\prime \right]^\prime +n^2 (1 - x^2)^{-1/2} y = 0 ~.
\end{equation}
The solutions, the Chebyshev polynomials $T_n(x)$, are given by a Rodrigues' formula:
\begin{equation}
T_n(x) = \dfrac{(-2)^n n! (1-x^2)^{1/2}}{(2n)!} \dfrac{\dif^n }{\dif x^n} (1-x^2)^{n-1/2} ~.
\end{equation}
Their orthogonality over the range $-1 \leqslant x \leqslant 1$ and their normalisation are given by
\begin{equation}
\int_{-1}^1 (1-x^2)^{-1/2} T_m(x) T_n(x) \dif x = 
\left\{
\begin{array}{ll}
      0     & \text{for}~ ~m \neq n ~, \\
      \pi/2 & \text{for} ~~n = m \neq 0 ~, \\
      \pi    & \text{for} ~~n = m = 0 ~, \\
\end{array} 
\right.
\end{equation}
and their generating function is
\begin{equation}
G(x, h) = \dfrac{1-xh}{1 -2xh +h^2} = \sum_{n=0}^\infty T_n (x) h^2 ~.
\end{equation}



\cite{arfken} The generating function for the Legendre polynomials can be generalized to
\begin{equation}
\dfrac{1}{(1-2xt+t^2)^\alpha} = \sum_{n=0}^\infty C^{(\alpha)}_n (x) t^n ~,
\end{equation}
where the coefficients $C^{(\alpha)}_n(x)$ are known as the \textcolor{red}{ultraspherical polynomials} (also called \textcolor{red}{Gegenbauer polynomials}). For $\alpha = 1/2$, we recover the Legendre polynomials. The special cases $\alpha = 0$ and $\alpha = 1$ yield two types of Chebyshev polynomials. The primary importance of the Chebyshev polynomials is in numerical analysis.

\section{Type I Polynomials}
With $\alpha = 1$, $C^{(1)}_n(x)$ is written as $U_n(x)$, 





\section{Type II Polynomials}





































%%%%%%%%%%%%%%%%%%%%%%%%%%%%%%%%%%%%%%%%%%%%%%%%%%%%%%%%%%%%%%%%%%%%%%
\bibliographystyle{unsrt_update}
\bibliography{ref}
%%%%%%%%%%%%%%%%%%%%%%%%%%%%%%%%%%%%%%%%%%%%%%%%%%%%%%%%%%%%%%%%%%%%%%

\end{document}