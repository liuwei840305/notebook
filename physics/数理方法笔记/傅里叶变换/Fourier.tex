\documentclass[12pt,a4paper]{article}
%\usepackage{fontspec, xunicode, xltxtra}  
%\setmainfont{Hiragino Sans GB}  
%\usepackage{xeCJK}
%\setCJKmainfont[BoldFont=STZhongsong, ItalicFont=STKaiti]{STSong}
%\setCJKsansfont[BoldFont=STHeiti]{STXihei}
%\setCJKmonofont{STFangsong}

%使用Xelatex编译

% 设置页面
%==================================================
\linespread{2} %行距
% \usepackage[top=1in,bottom=1in,left=1.25in,right=1.25in]{geometry}
% \headsep=2cm
% \textwidth=16cm \textheight=24.2cm
%==================================================

% 其它需要使用的宏包
%==================================================
\usepackage[colorlinks,linkcolor=blue,anchorcolor=red,citecolor=green,urlcolor=blue]{hyperref} 
\usepackage{tabularx}
\usepackage{authblk}         % 作者信息
\usepackage{algorithm}     % 算法排版
\usepackage{amsmath}     % 数学符号与公式
\usepackage{amsfonts}     % 数学符号与字体
\usepackage{mathrsfs}      % 花体
\usepackage[framemethod=TikZ]{mdframed}

\usepackage{graphicx} 
\usepackage{graphics}
\usepackage{color}
\usepackage{xcolor}
\usepackage{tcolorbox}
\usepackage{lipsum}
\usepackage{empheq}

\usepackage{fancyhdr}       % 设置页眉页脚
\usepackage{fancyvrb}       % 抄录环境
\usepackage{float}              % 管理浮动体
\usepackage{geometry}     % 定制页面格式
\usepackage{hyperref}       % 为PDF文档创建超链接
\usepackage{lineno}          % 生成行号
\usepackage{listings}        % 插入程序源代码
\usepackage{multicol}       % 多栏排版
%\usepackage{natbib}         % 管理文献引用
\usepackage{rotating}       % 旋转文字,图形,表格
\usepackage{subfigure}    % 排版子图形
\usepackage{titlesec}       % 改变章节标题格式
\usepackage{moresize}   % 更多字体大小
\usepackage{anysize}
\usepackage{indentfirst}  % 首段缩进
\usepackage{booktabs}   % 使用\multicolumn
\usepackage{multirow}    % 使用\multirow
\usepackage{wrapfig}
\usepackage{enumitem}
\usepackage{harpoon}   %矢量符号

\usepackage{aas_macros}

\newcommand{\myvec}[1]%
   {\stackrel{\raisebox{-2pt}[0pt][0pt]{\small$\rightharpoonup$}}{#1}}  %矢量符号
\renewcommand{\vec}[1]{\boldsymbol{#1}}
\newcommand{\me}{\mathrm{e}}
\newcommand{\mi}{\mathrm{i}}
\newcommand{\dif}{\mathrm{d}}
\newcommand{\tabincell}[2]{\begin{tabular}{@{}#1@{}}#2\end{tabular}}

\def\kpc{{\rm kpc}}
\def\km{{\rm km}}
\def\cm{{\rm cm}}
\def\TeV{{\rm TeV}}
\def\GeV{{\rm GeV}}
\def\MeV{{\rm MeV}}
\def\GV{{\rm GV}}
\def\MV{{\rm MV}}
\def\yr{{\rm yr}}
\def\s{{\rm s}}
\def\ns{{\rm ns}}
\def\GHz{{\rm GHz}}
\def\muGs{{\rm \mu Gs}}
\def\arcsec{{\rm arcsec}}
\def\K{{\rm K}}
\def\microK{\mu{\rm K}}
\def\sr{{\rm sr}}
\newcolumntype{p}{D{,}{\pm}{-1}}

\renewcommand{\figurename}{Fig.}
\renewcommand{\tablename}{Tab.}

\renewcommand{\arraystretch}{1.5}

\setlength{\parindent}{0pt}  %取消每段开头的空格

\newcounter{theo}[section]\setcounter{theo}{0}
\renewcommand{\thetheo}{\arabic{section}.\arabic{theo}}
\newenvironment{theo}[2][]{%
\refstepcounter{theo}%
\ifstrempty{#1}%
{\mdfsetup{%
frametitle={%
\tikz[baseline=(current bounding box.east),outer sep=0pt]
\node[anchor=east,rectangle,fill=blue!20]
{\strut Theorem~\thetheo};}}
}%
{\mdfsetup{%
frametitle={%
\tikz[baseline=(current bounding box.east),outer sep=0pt]
\node[anchor=east,rectangle,fill=blue!20]
{\strut Theorem~\thetheo:~#1};}}%
}%
\mdfsetup{innertopmargin=10pt,linecolor=blue!20,%
linewidth=2pt,topline=true,%
frametitleaboveskip=\dimexpr-\ht\strutbox\relax
}
\begin{mdframed}[]\relax%
\label{#2}}{\end{mdframed}}

\newcommand*\widefbox[1]{\fbox{\hspace{2em}#1\hspace{2em}}}

%从右倾的积分号变为竖直的积分号
%define a new command for \rm font of int
\DeclareSymbolFont{rmlargesymbols}{OMX}{mdbch}{m}{n}
% or \DeclareSymbolFont{rmlargesymbols}{U}{euex}{m}{n}
\DeclareMathSymbol{\rmintop}{\mathop}{rmlargesymbols}{82}
\newcommand{\rmint}{\rmintop\nolimits}

\title{Fourier Transform}
\author{}
\date{\today}
\begin{document}

\maketitle

\section{Integral transforms}
The pairs of functions are related by an expression of the form
\begin{align}
\label{oper_equ}
& g(x) = \int_a^b f(t) K(x, t) \dif t = \mathcal L f(t) ~, \\
&  \mathcal L^{-1} g(x) = f(t) ~,
\end{align}
where $a$, $b$, and $K(x,t)$ (called the \textcolor{red}{kernel}) will be the same for all function pairs $f$ and $g$. Eq. (\ref{oper_equ}) can be interpreted as an operator equation. The function $g(x)$ is called the integral transform of $f(t)$ by the operator $\mathcal L$, with the specific transform determined by the choice of $a$, $b$, and $K(x,t)$. The operator defined by Eq. (\ref{oper_equ}) will be linear:
\begin{align}
& \int_a^b [f_1(t) +f_2(t)] K(x, t) \dif t = \int_a^b f_1(t) K(x, t) \dif t + \int_a^b f_2(t) K(x, t) \dif t ~, \\
& \int_a^b c f(t) K(x, t) \dif t = c \int_a^b f(t) K(x, t) \dif t =
\end{align}

\subsection{Some Important Transforms}
The Fourier transform
\begin{equation}
g(\omega) = \dfrac{1}{\sqrt{2\pi} } \int_{-\infty}^\infty f(t) e^{i\omega t} \dif t ~,
\end{equation}

The Laplace transform,
\begin{equation}
F(s) = \int_0^\infty e^{-st} f(t) \dif t ~,
\end{equation}



The \textcolor{red}{Hankel transform},
\begin{equation}
g(\alpha) = \int_0^\infty f(t) t J_n(\alpha t) \dif t ~,
\end{equation}
represents the continuum limit of the Bessel series. 

The \textcolor{red}{Mellin transform},
\begin{equation}
g(\alpha) = \int_0^\infty f(t) t^{\alpha-1} \dif t ~,
\end{equation}


\section{Fourier transform}
The Fourier and inverse Fourier transforms are
\begin{align}
& g(\omega) = \dfrac{1}{\sqrt{2\pi} } \int_{-\infty}^\infty f(t) e^{i\omega t} \dif t ~, \\
& f(t) = \dfrac{1}{\sqrt{2\pi} } \int_{-\infty}^\infty g(\omega) e^{-i\omega t} \dif \omega ~.
\end{align}

\begin{align}
& g_c(\omega) = \sqrt{\dfrac{2}{\pi} } \int_{0}^\infty f(t) \cos \omega t \dif t ~, \\
& f_c(t) = \sqrt{\dfrac{2}{\pi} } \int_{0}^\infty g(\omega) \cos \omega t \dif \omega ~.
\end{align}

\begin{align}
& g_s(\omega) = \sqrt{\dfrac{2}{\pi} } \int_{0}^\infty f(t) \sin \omega t \dif t ~, \\
& f_s(t) = \sqrt{\dfrac{2}{\pi} } \int_{0}^\infty g(\omega) \sin \omega t \dif \omega ~.
\end{align}


\begin{align}
& g(\vec{k}) = \dfrac{1}{(2\pi)^{3/2}} \int_{-\infty}^\infty f(\vec{r}) e^{i\vec{k} \cdot \vec{r} } \dif^3 r ~, \\
& f(\vec{r}) = \dfrac{1}{(2\pi)^{3/2}} \int_{-\infty}^\infty g(\vec{k}) e^{-i\vec{k} \cdot \vec{r}} \dif^3 k ~.
\end{align}

\section{Fourier convolution theorem}



\subsection{Parseval Relation}













\section{Discrete Fourier transform}










\section{Fast Fourier Transform}













































































































































\end{document}