\documentclass[12pt,a4paper]{article}
%\usepackage{fontspec, xunicode, xltxtra}  
%\setmainfont{Hiragino Sans GB}  
\usepackage{xeCJK}
%\setCJKmainfont[BoldFont=STZhongsong, ItalicFont=STKaiti]{STSong}
%\setCJKsansfont[BoldFont=STHeiti]{STXihei}
%\setCJKmonofont{STFangsong}

%使用Xelatex编译

% 设置页面
%==================================================
\linespread{2} %行距
% \usepackage[top=1in,bottom=1in,left=1.25in,right=1.25in]{geometry}
% \headsep=2cm
% \textwidth=16cm \textheight=24.2cm
%==================================================

% 其它需要使用的宏包
%==================================================
\usepackage[colorlinks,linkcolor=blue,anchorcolor=red,citecolor=green,urlcolor=blue]{hyperref} 
\usepackage{tabularx}
\usepackage{authblk}         % 作者信息
\usepackage{algorithm}     % 算法排版
\usepackage{amsmath}     % 数学符号与公式
\usepackage{amsfonts}     % 数学符号与字体
\usepackage{mathrsfs}      % 花体
\usepackage{amssymb}

\usepackage{graphicx} 
\usepackage{graphics}
\usepackage{color}
\usepackage{xcolor}

\usepackage{float}
\usepackage{fancyhdr}       % 设置页眉页脚
\usepackage{fancyvrb}       % 抄录环境
\usepackage{float}              % 管理浮动体
\usepackage{geometry}     % 定制页面格式
\usepackage{hyperref}       % 为PDF文档创建超链接
\usepackage{lineno}          % 生成行号
\usepackage{listings}        % 插入程序源代码
\usepackage{multicol}       % 多栏排版
%\usepackage{natbib}         % 管理文献引用
\usepackage{rotating}       % 旋转文字,图形,表格
\usepackage{subfigure}    % 排版子图形
\usepackage{titlesec}       % 改变章节标题格式
\usepackage{moresize}   % 更多字体大小
\usepackage{anysize}
\usepackage{indentfirst}  % 首段缩进
\usepackage{booktabs}   % 使用\multicolumn
\usepackage{multirow}    % 使用\multirow

\usepackage{wrapfig}
\usepackage{titlesec}     % 改变标题样式
\usepackage{enumitem}
\usepackage{aas_macros}

\newcommand{\myvec}[1]%
   {\stackrel{\raisebox{-2pt}[0pt][0pt]{\small$\rightharpoonup$}}{#1}}  %矢量符号
\renewcommand{\vec}[1]{\boldsymbol{#1}}
\newcommand{\me}{\mathrm{e}}
\newcommand{\mi}{\mathrm{i}}
\newcommand{\dif}{\mathrm{d}}
\newcommand{\tabincell}[2]{\begin{tabular}{@{}#1@{}}#2\end{tabular}}

\def\kpc{{\rm kpc}}
\def\km{{\rm km}}
\def\cm{{\rm cm}}
\def\TeV{{\rm TeV}}
\def\GeV{{\rm GeV}}
\def\MeV{{\rm MeV}}
\def\GV{{\rm GV}}
\def\MV{{\rm MV}}
\def\yr{{\rm yr}}
\def\s{{\rm s}}
\def\ns{{\rm ns}}
\def\GHz{{\rm GHz}}
\def\muGs{{\rm \mu Gs}}
\def\arcsec{{\rm arcsec}}
\def\K{{\rm K}}
\def\microK{\mu{\rm K}}
\def\sr{{\rm sr}}
\newcolumntype{p}{D{,}{\pm}{-1}}

\renewcommand{\figurename}{Fig.}
\renewcommand{\tablename}{Tab.}

\renewcommand{\arraystretch}{1.5}

\setlength{\parindent}{0pt}  %取消每段开头的空格

\title{Atomic Physics}
\author{}
\date{\today}
\begin{document}

\maketitle


\subsection{Bohr Model}
1. \textcolor{red}{定态条件}:氢原子中的一个电子绕原子核作圆周运动,电子只能处于一些分立的轨道上,它只能在这些轨道上绕核转动,且不产生辐射。

2. \textcolor{red}{频率条件}:当电子从一个定态轨道跃迁到另一个定态轨道时,会以电磁波的形式 放出(或吸收)能量$h\nu$,其值由能级差决定:
\begin{equation}
h\nu = E_{n^\prime} - E_{n} ~,
\end{equation}

3. 角动量量子化
\begin{equation}
L = n \hbar ~, n = 1, 2, 3, ⋯
\end{equation}

4.对应原理



推导
\begin{align*}
F &= m_e \dfrac{v^2}{r} ~, \\
\dfrac{1}{4\pi \epsilon_0} \dfrac{e^2}{r^2} &= \dfrac{m_e v^2}{r} ~, \\
v^2 &= \dfrac{1}{4\pi \epsilon_0} \dfrac{e^2}{m_e r} ~.
\end{align*}
\begin{align*}
m_e vr &= n\hbar ~, \\
v &= \dfrac{n \hbar}{m_e r} ~.
\end{align*}
\begin{align}
\nonumber  & \dfrac{1}{4\pi \epsilon_0} \dfrac{e^2}{m_e r} = \dfrac{n^2 \hbar^2}{m_e^2 r^2} ~, \\
& \color{red} r = \dfrac{4\pi \epsilon_0 \hbar^2}{m_e e^2} n^2 ~, \\
& \color{red} r_n = \dfrac{1}{4\pi \epsilon_0} \dfrac{e^2}{2Rhc} n^2 ~.
\end{align}
\begin{align}
\nonumber E = T +V &= \dfrac{1}{2} m_e v^2 -\dfrac{e^2}{4\pi \epsilon_0 r} ~, \\
\nonumber &= \dfrac{1}{2} \dfrac{e^2}{4\pi \epsilon_0 r} - \dfrac{e^2}{4\pi \epsilon_0 r} ~, \\
\nonumber &= -\dfrac{1}{2}  \dfrac{e^2}{4\pi \epsilon_0 r} ~, \\
\nonumber &= -\dfrac{1}{2}  \dfrac{e^2}{4\pi \epsilon_0 } \times \dfrac{m_e e^2 }{4\pi \epsilon_0 n^2\hbar^2} ~, \\
\color{red} E_n & \color{red}= - \dfrac{m_e e^4}{(4\pi \epsilon_0)^2 \cdot 2 n^2\hbar^2} ~. \\
\end{align}
电子作圆周运动的频率
\begin{equation}
f = \dfrac{v}{2\pi r} = \dfrac{1}{2\pi r} \sqrt{\dfrac{e^2}{4\pi \epsilon_0 m_e r}} = \dfrac{e}{2\pi} \sqrt{\dfrac{1}{4\pi \epsilon_0 m_e r^3}}
\end{equation}

\begin{align}
& \hbar c = 197 ~\text{fm} \cdot \text{MeV} = 197 ~\text{nm} \cdot \text{eV} ~, \\
& \dfrac{e^2}{4\pi \epsilon_0} = 1.44 ~\text{fm} \cdot \text{MeV} = 1.44 ~\text{nm} \cdot \text{eV} ~, \\
& m_e c^2 = 0.511 ~\text{MeV} = 511 ~\text{keV} ~,
\end{align}
$1$ fm $= 10^{-6}$ nm $= 10^{-15}$ m。精细结构常数
\begin{equation}
\alpha \equiv \dfrac{e^2}{4\pi \epsilon_0 \hbar c} \approx \dfrac{1}{137} ~.
\end{equation}

\begin{equation}
r_1 \equiv a_1 = \dfrac{4\pi \epsilon_0 \hbar^2}{m_e e^2} = \dfrac{(\hbar c)^2}{m_e c^2 e^2 /4\pi \epsilon_0} = \dfrac{(197)^2}{0.511\times 10^6 \times 1.44} ~\text{nm} \approx \dfrac{0.039 \times 10^6}{0.73 \times 10^6} ~\text{nm} \approx 0.053 ~\text{nm} ~.
\end{equation}



























%%%%%%%%%%%%%%%%%%%%%%%%%%%%%%%%%%%%%%%%%%%%%%%%%%%%%%%%%%%%%%%%%%%%%%
\bibliographystyle{unsrt_update}
\bibliography{ref}
%%%%%%%%%%%%%%%%%%%%%%%%%%%%%%%%%%%%%%%%%%%%%%%%%%%%%%%%%%%%%%%%%%%%%%

\end{document}