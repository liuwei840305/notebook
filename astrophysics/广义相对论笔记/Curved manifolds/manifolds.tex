\documentclass[12pt,a4paper]{article}
%\usepackage{fontspec, xunicode, xltxtra}  
%\setmainfont{Hiragino Sans GB}  
%\usepackage{xeCJK}
%\setCJKmainfont[BoldFont=STZhongsong, ItalicFont=STKaiti]{STSong}
%\setCJKsansfont[BoldFont=STHeiti]{STXihei}
%\setCJKmonofont{STFangsong}

%使用Xelatex编译

% 设置页面
%==================================================
\linespread{2} %行距
% \usepackage[top=1in,bottom=1in,left=1.25in,right=1.25in]{geometry}
% \headsep=2cm
% \textwidth=16cm \textheight=24.2cm
%==================================================

% 其它需要使用的宏包
%==================================================
\usepackage[colorlinks,linkcolor=blue,anchorcolor=red,citecolor=green,urlcolor=blue]{hyperref} 
\usepackage{tabularx}
\usepackage{authblk}         % 作者信息
\usepackage{algorithm}     % 算法排版
\usepackage{amsmath}     % 数学符号与公式
\usepackage{amsfonts}     % 数学符号与字体
\usepackage{mathrsfs}      % 花体
\usepackage[framemethod=TikZ]{mdframed}

\usepackage{graphicx} 
\usepackage{graphics}
\usepackage{color}
\usepackage{xcolor}
\usepackage{tcolorbox}
\usepackage{lipsum}
\usepackage{empheq}

\usepackage{fancyhdr}       % 设置页眉页脚
\usepackage{fancyvrb}       % 抄录环境
\usepackage{float}              % 管理浮动体
\usepackage{geometry}     % 定制页面格式
\usepackage{hyperref}       % 为PDF文档创建超链接
\usepackage{lineno}          % 生成行号
\usepackage{listings}        % 插入程序源代码
\usepackage{multicol}       % 多栏排版
%\usepackage{natbib}         % 管理文献引用
\usepackage{rotating}       % 旋转文字,图形,表格
\usepackage{subfigure}    % 排版子图形
\usepackage{titlesec}       % 改变章节标题格式
\usepackage{moresize}   % 更多字体大小
\usepackage{anysize}
\usepackage{indentfirst}  % 首段缩进
\usepackage{booktabs}   % 使用\multicolumn
\usepackage{multirow}    % 使用\multirow

\usepackage{wrapfig}
\usepackage{titlesec}     % 改变标题样式
\usepackage{enumitem}
\usepackage{aas_macros}

\newcommand{\myvec}[1]%
   {\stackrel{\raisebox{-2pt}[0pt][0pt]{\small$\rightharpoonup$}}{#1}}  %矢量符号
\renewcommand{\vec}[1]{\boldsymbol{#1}}
\newcommand{\me}{\mathrm{e}}
\newcommand{\mi}{\mathrm{i}}
\newcommand{\dif}{\mathrm{d}}
\newcommand{\tabincell}[2]{\begin{tabular}{@{}#1@{}}#2\end{tabular}}

\def\kpc{{\rm kpc}}
\def\km{{\rm km}}
\def\cm{{\rm cm}}
\def\TeV{{\rm TeV}}
\def\GeV{{\rm GeV}}
\def\MeV{{\rm MeV}}
\def\GV{{\rm GV}}
\def\MV{{\rm MV}}
\def\yr{{\rm yr}}
\def\s{{\rm s}}
\def\ns{{\rm ns}}
\def\GHz{{\rm GHz}}
\def\muGs{{\rm \mu Gs}}
\def\arcsec{{\rm arcsec}}
\def\K{{\rm K}}
\def\microK{\mu{\rm K}}
\def\sr{{\rm sr}}
\newcolumntype{p}{D{,}{\pm}{-1}}

\renewcommand{\figurename}{Fig.}
\renewcommand{\tablename}{Tab.}

\renewcommand{\arraystretch}{1.5}

\setlength{\parindent}{0pt}  %取消每段开头的空格

\newcounter{theo}[section]\setcounter{theo}{0}
\renewcommand{\thetheo}{\arabic{section}.\arabic{theo}}
\newenvironment{theo}[2][]{%
\refstepcounter{theo}%
\ifstrempty{#1}%
{\mdfsetup{%
frametitle={%
\tikz[baseline=(current bounding box.east),outer sep=0pt]
\node[anchor=east,rectangle,fill=blue!20]
{\strut Theorem~\thetheo};}}
}%
{\mdfsetup{%
frametitle={%
\tikz[baseline=(current bounding box.east),outer sep=0pt]
\node[anchor=east,rectangle,fill=blue!20]
{\strut Theorem~\thetheo:~#1};}}%
}%
\mdfsetup{innertopmargin=10pt,linecolor=blue!20,%
linewidth=2pt,topline=true,%
frametitleaboveskip=\dimexpr-\ht\strutbox\relax
}
\begin{mdframed}[]\relax%
\label{#2}}{\end{mdframed}}

\newcommand*\widefbox[1]{\fbox{\hspace{2em}#1\hspace{2em}}}


\title{Curved manifolds}
\author{}
\date{\today}
\begin{document}

\maketitle
\section{On the relation of gravitation to curvature}
The existence of inertial frames that fill all of spacetime: all of spacetime can be described by a single frame, all of whose coordinate points are always at rest relative to the origin, and all of whose clocks run at the same rate relative to the origin’s clock.

The metric of SR is defined physically by lengths of rods and readings of clocks. 

In a nonuniform gravitational field, it is impossible to construct a frame in which the clocks all run at the same rate. The gravitational fields are incompatible with global SR: the ability to construct a global inertial frame. However in small regions of spacetime - regions small enough that nonuniformities of the gravitational forces are too small to measure - we can always construct a `local' SR frame. 

\textcolor{red}{The clocks don't all run at the same rate in a gravitational field}.



\subsection{The gravitational redshift experiment}





















\subsection{Nonexistence of a Lorentz frame at rest on Earth}




















\subsection{The principle of equivalence}
















\subsection{The redshift experiment again}













\subsection{Tidal forces}
Nonuniformities in gravitational fields are called \textcolor{red}{tidal forces}, since they are the ones that raise tides. These tidal forces prevent the construction of global inertial frames.

\subsection{The role of curvature}







\section{Tensor algebra in polar coordinates}






\section{Tensor calculus in polar coordinates}




\subsection{The Christoffel symbols}




\subsection{The covariant derivative}





\subsection{Divergence and Laplacian}





\subsection{Derivatives of one-forms and tensors of higher types}






\section{Christoffel symbols and the metric}










\section{Noncoordinate bases}








\section{Differentiable manifolds and tensors}
A manifold is essentially a continuous space which looks locally like Euclidean space.






\section{Riemannian manifolds}







\subsection{Lengths and volumes}






\subsection{Proof of the local-flatness theorem}







\section{Covariant differentiation}







\subsection{Divergence formula}





\section{Parallel-transport, geodesics, and curvature}





\section{The curvature tensor}







\section{Bianchi identities: Ricci and Einstein tensors}

\begin{equation}
R_{\alpha\beta\mu\nu, \lambda} = \frac{1}{2} (g_{\alpha\nu, \beta\mu\lambda} -g_{\alpha\mu, \beta\nu\lambda} +g_{\beta\mu, \alpha\nu\lambda} -g_{\beta\nu, \alpha\mu\lambda} = 0 ~.
\end{equation}

\begin{equation}
R_{\alpha\beta\mu\nu, \lambda} +R_{\alpha\beta\lambda\mu, \nu} +R_{\alpha\beta\nu\lambda, \mu} = 0 ~.
\end{equation}


Since in our coordinates $\Gamma^\mu{}_{\alpha\beta} = 0$ at this point, this is equivalent to
\begin{equation}
R_{\alpha\beta\mu\nu; \lambda} +R_{\alpha\beta\lambda\mu; \nu} +R_{\alpha\beta\nu\lambda; \mu} = 0 ~.
\end{equation}


\subsection{The Ricci tensor}
The Ricci tensor is defined as
\begin{equation}
R_{\alpha\beta} := R^\mu{}_{\alpha \mu\beta} = R_{\beta\alpha} ~.
\end{equation}
It is the contraction of $R^\mu{}_{\alpha \nu\beta}$ on the first and third indices. Other contractions would in principle also be possible: on the first and second, the first and fourth, etc. Because $R_{\alpha\beta\mu\nu}$ is antisymmetric on $\alpha$ and $\beta$ and on $\mu$ and $\nu$, all these contractions either vanish identically or reduce to $\pm R_{\alpha\beta}$. The Ricci tensor is essentially the only contraction of the Riemann tensor. It is a symmetric tensor. The Ricci scalar is defined as
\begin{equation}
R := g^{\mu\nu} R_{\mu\nu} = g^{\mu\nu} g^{\alpha\beta} R_{\alpha\mu\beta\nu}  ~.
\end{equation}





































































%%%%%%%%%%%%%%%%%%%%%%%%%%%%%%%%%%%%%%%%%%%%%%%%%%%%%%%%%%%%%%%%%%%%%%
\bibliographystyle{unsrt_update}
\bibliography{ref}
%%%%%%%%%%%%%%%%%%%%%%%%%%%%%%%%%%%%%%%%%%%%%%%%%%%%%%%%%%%%%%%%%%%%%%

\end{document}