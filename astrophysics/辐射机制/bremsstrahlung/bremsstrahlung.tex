\documentclass[12pt,a4paper]{article}
%\usepackage{fontspec, xunicode, xltxtra}  
%\setmainfont{Hiragino Sans GB}  
%\usepackage{xeCJK}
%\setCJKmainfont[BoldFont=STZhongsong, ItalicFont=STKaiti]{STSong}
%\setCJKsansfont[BoldFont=STHeiti]{STXihei}
%\setCJKmonofont{STFangsong}

%使用Xelatex编译

% 设置页面
%==================================================
\linespread{2} %行距
% \usepackage[top=1in,bottom=1in,left=1.25in,right=1.25in]{geometry}
% \headsep=2cm
% \textwidth=16cm \textheight=24.2cm
%==================================================

% 其它需要使用的宏包
%==================================================
\usepackage[colorlinks,linkcolor=blue,anchorcolor=red,citecolor=green,urlcolor=blue]{hyperref} 
\usepackage{tabularx}
\usepackage{authblk}         % 作者信息
\usepackage{algorithm}     % 算法排版
\usepackage{amsmath}     % 数学符号与公式
\usepackage{amsfonts}     % 数学符号与字体
\usepackage{mathrsfs}      % 花体
\usepackage{graphics}
\usepackage{color}
\usepackage{fancyhdr}       % 设置页眉页脚
\usepackage{fancyvrb}       % 抄录环境
\usepackage{float}              % 管理浮动体
\usepackage{geometry}     % 定制页面格式
\usepackage{hyperref}       % 为PDF文档创建超链接
\usepackage{lineno}          % 生成行号
\usepackage{listings}        % 插入程序源代码
\usepackage{multicol}       % 多栏排版
%\usepackage{natbib}         % 管理文献引用
\usepackage{rotating}       % 旋转文字,图形,表格
\usepackage{subfigure}    % 排版子图形
\usepackage{titlesec}       % 改变章节标题格式
\usepackage{moresize}   % 更多字体大小
\usepackage{anysize}
\usepackage{indentfirst}  % 首段缩进
\usepackage{booktabs}   % 使用\multicolumn
\usepackage{multirow}    % 使用\multirow
\usepackage{graphicx} 
\usepackage{wrapfig}
\usepackage{xcolor}
\usepackage{titlesec}     % 改变标题样式
\usepackage{enumitem}
\usepackage{aas_macros}

\newcommand{\myvec}[1]%
   {\stackrel{\raisebox{-2pt}[0pt][0pt]{\small$\rightharpoonup$}}{#1}}  %矢量符号
\renewcommand{\vec}[1]{\boldsymbol{#1}}
\newcommand{\me}{\mathrm{e}}
\newcommand{\mi}{\mathrm{i}}
\newcommand{\dif}{\mathrm{d}}
\newcommand{\tabincell}[2]{\begin{tabular}{@{}#1@{}}#2\end{tabular}}

\def\kpc{{\rm kpc}}
\def\km{{\rm km}}
\def\cm{{\rm cm}}
\def\TeV{{\rm TeV}}
\def\GeV{{\rm GeV}}
\def\MeV{{\rm MeV}}
\def\GV{{\rm GV}}
\def\MV{{\rm MV}}
\def\yr{{\rm yr}}
\def\s{{\rm s}}
\def\ns{{\rm ns}}
\def\GHz{{\rm GHz}}
\def\muGs{{\rm \mu Gs}}
\def\arcsec{{\rm arcsec}}
\def\K{{\rm K}}
\def\microK{\mu{\rm K}}
\def\sr{{\rm sr}}
\newcolumntype{p}{D{,}{\pm}{-1}}

\renewcommand{\figurename}{Fig.}
\renewcommand{\tablename}{Tab.}

\renewcommand{\arraystretch}{1.5}

\setlength{\parindent}{0pt}  %取消每段开头的空格

\title{Bremsstrahlung}
\author{}
\date{\today}
\begin{document}

\maketitle

\subsection{relativistic invariants}
transformation of the energy loss rate by electromagnetic radiation as observed in different inertial frames of reference, i.e. how $\dif E/\dif t$ changes from one inertial frame of reference to another.

\textcolor{red}{$\dif E/\dif t$} is a \textcolor{red}{Lorentz invariant} between inertial frames of reference.

In the moving instantaneous rest frame of an accelerated charged particle, the total energy loss $\dif E^{\prime}$ has dipole symmetry and so is \textcolor{red}{emitted with zero net momentum}. Its four-momentum can be written $[\dif E^{\prime}/c, 0]$. This radiation is emitted in the interval of time $\dif t^{\prime}$, which is the zeroth component of the displacement four-vector $[c ~\dif t^{\prime}, 0]$. Using the inverse Lorentz transforms to relate $\dif E^{\prime}$ and $c ~\dif t^{\prime}$ to $\dif E$ and $c~ \dif t$,
\begin{equation}
\dif E = \gamma \dif E^{\prime} ; ~~ \dif t = \gamma \dif t^{\prime}
\end{equation}
hence
\begin{equation}
\dif E/\dif t  = \dif E^{\prime}/\dif t^{\prime}
\end{equation}


\subsection{The radiation of an accelerated charged particle – J. J. Thomson’s treatment}


\subsection{The radiation of an accelerated charged particle – from Maxwell’s equations}
Maxwell’s equations in free space are 
\begin{eqnarray}
\nabla \times \vec{E} &=& -\frac{\partial \vec{B}}{\partial t} ~,\\
\nabla \times \vec{B} &=& \mu_0 \vec{J} +\frac{1}{c^2}\frac{\partial \vec{E}}{\partial t} ~,\\
\nabla \cdot \vec{B} &=& 0 ~,\\
\nabla \cdot \vec{E} &=& \frac{\rho_{\rm e}}{\epsilon_0}
\end{eqnarray}
Introduce the \textcolor{red}{scalar} and \textcolor{red}{vector potentials}, $\phi$ and $\vec{A}$ respectively,
\begin{eqnarray}
\vec{B} &=& \nabla \times \vec{A} ~, \\
\vec{E} &=& -\frac{\partial \vec{A}}{\partial t} -\nabla \phi
\end{eqnarray}
four-vector potential $[\phi/c, \vec{A}]$ 
\begin{equation}
\nabla \times (\nabla \times \vec{A}) = \mu_0 \vec{J} -\frac{1}{c^2} \frac{\partial}{\partial t} \left(  \frac{\partial \vec{A}}{\partial t} +\nabla \phi \right)
\end{equation}
recall 
\begin{equation*}
\nabla \times (\nabla \times \vec{A})  = \nabla(\nabla\cdot \vec{A}) -\nabla^2 \vec{A}
\end{equation*}

\begin{eqnarray*}
\nabla^2 \vec{A} -\frac{1}{c^2} \frac{\partial^2 \vec{A} }{\partial t^2} &=& -\mu_0 \vec{J} + \nabla\left[\nabla \cdot \vec{A} +\frac{1}{c^2} \frac{\partial \phi }{\partial t} \right] ~,\\
\nabla^2 \phi -\frac{1}{c^2} \frac{\partial^2 \phi }{\partial t^2} &=& -\frac{\rho_{\rm e}}{\epsilon_0} - \frac{\partial}{\partial t}\left[\nabla \cdot \vec{A} +\frac{1}{c^2} \frac{\partial \phi }{\partial t} \right] 
\end{eqnarray*}
We can always add to $\vec{A}$ the gradient of any scalar quantity and it will be guaranteed to disappear upon curling. Write $\vec{A} = \vec{A} + \text{grad} \chi$, the value of $\vec{B}$ will be unchanged. 
\begin{equation}
\vec{E} = -\frac{\partial \vec{A}^{\prime} }{\partial t} -\nabla(\phi -\dot{\chi})
\end{equation}
we need to replace $\phi$ by $\phi^{\prime} = \phi − \dot{\chi}$. Express the condition that
$\nabla \cdot \vec{A} +1/c^2 (\partial \phi/\partial t)$ should vanish as follows
\begin{eqnarray}
\nonumber &&\nabla \cdot (\vec{A}^{\prime} -\nabla \chi) + \frac{1}{c^2}\frac{\partial}{\partial t} (\phi^{\prime} +\dot{\chi}) = 0 ~, \\
 &&\nabla \cdot \vec{A}^{\prime} +\frac{1}{c^2}\frac{\partial \phi^{\prime} }{\partial t} = \nabla^2 \chi -\frac{1}{c^2} \frac{\partial^2 \chi }{\partial t^2} \label{gauge}
\end{eqnarray}
Provided find a suitable function $\chi$, which satisfies (\ref{gauge}), we obtain the pair of equations separately for $\vec{A}$ and $\phi$:
\begin{eqnarray}
\nabla^2 \vec{A} -\frac{1}{c^2} \frac{\partial^2 \vec{A} }{\partial t^2} &=& -\mu_0 \vec{J} ~,\\
\nabla^2 \phi -\frac{1}{c^2} \frac{\partial^2 \phi }{\partial t^2} &=& -\frac{\rho_{\rm e}}{\epsilon_0} \end{eqnarray}
\begin{equation}
\nabla^2 \chi -\frac{1}{c^2} \frac{\partial^2 \chi }{\partial t^2} \label{gauge} = 0
\end{equation}
This procedure is known as \emph{selecting the gauge}. This particular choice is known as \textcolor{red}{Lorentz gauge}.



these results are correct provided the velocities of the charges are small. 

\textcolor{red}{Li\'{e}nard–Wiechert potentials}
\begin{eqnarray}
\vec{A}(\vec{r}, t)  &=& \frac{\mu_0}{4\pi r} \left[\frac{q\vec{v}}{1-(\vec{v}\cdot \vec{n})/c}  \right]_{\rm ret} ~, \\
\phi(\vec{r}, t) &=& \frac{1}{4\pi \epsilon_0 r} \left[\frac{q}{1-(\vec{v}\cdot \vec{n})/c}  \right]_{\rm ret} 
\end{eqnarray}
where \textcolor{red}{$\vec{n}$} is the \textcolor{red}{unit vector in the direction of the point of observation from the moving charge}. The potentials are evaluated at \textcolor{red}{retarded times relative to the location of the observer}.


\subsection{The radiation losses of accelerated charged particles moving at relativistic velocities}
It is assumed that, in the \textcolor{red}{particle’s instantaneous rest frame}, the acceleration of the particle is small. The norm of the acceleration four-vector is an invariant in any inertial frame of reference. The \textcolor{red}{acceleration four-vector of the particle}, \textcolor{red}{$\vec{A}$}, not to be confused with the vector potential $\vec{A}$, can be written
\begin{eqnarray}
\textcolor{blue}{ \vec{A} = \gamma \left[c\frac{\partial \gamma}{\partial t} ,  \frac{\partial \gamma \vec{v}}{\partial t}  \right]  = \left[\left(\frac{\vec{v}\cdot \vec{a}}{c^2}\right) \gamma^4 c,  \gamma^2 \vec{a} +\left(\frac{\vec{v}\cdot \vec{a}}{c^2}\right) \gamma^4 \vec{v}  \right] }
\end{eqnarray}
The acceleration $\vec{a} = \vec{\ddot{r}}$ and the velocity of the particle $\vec{v} = \vec{\dot{r}}$ are measured in the observer’s frame of reference $S$. In the \textcolor{red}{instantaneous rest frame of the particle}, \textcolor{red}{$S^{\prime}$}, the acceleration four-vector is $[0, \vec{a}_0]$, where $\vec{a}_0 = (\vec{\ddot{r}})_0$ is the \textcolor{red}{proper acceleration} of the particle. Equate the norms of the four-vectors in the reference frames $S$ and $S^{\prime}$:
\begin{equation}
-\vec{a}^2_0 = \gamma^8 c^2 \left(\frac{\vec{v}\cdot \vec{a}}{c^2}\right)^2 - \left[\gamma^2 \vec{a} +\left(\frac{\vec{v}\cdot \vec{a}}{c^2}\right) \gamma^4 \vec{v} \right]^2
\end{equation}
\begin{equation}
\vec{a}^2_0 = \gamma^4 \left[ \vec{a}^2 +\gamma^2 \left(\frac{\vec{v}\cdot \vec{a}}{c}\right)^2 \right]
\end{equation}
The radiation rate ($\dif E/\dif t$) is a Lorentz invariant, then 
\begin{eqnarray}
\left(\frac{\dif E}{\dif t}\right)_{S} = \left(\frac{\dif E^{\prime}}{\dif t^{\prime}}\right)_{S^{\prime}} = \frac{q^2 |\vec{a}_0|^2}{6\pi \epsilon_0 c^3} = \frac{q^2 \gamma^4}{6\pi \epsilon_0 c^3} \left[  \vec{a}^2 +\gamma^2 \left(\frac{\vec{v}\cdot \vec{a}}{c}\right)^2 \right]
\end{eqnarray}
Resolve the acceleration of the particle into components parallel $a_{\parallel}$ and perpendicular $a_{\perp}$ to the velocity vector $\vec{v}$, that is,
\begin{eqnarray*}
\vec{a} &=& a_{\parallel} \vec{i}_{\parallel} +a_{\perp} \vec{i}_{\perp} ~,\\
|\vec{a}|^2 &=& |a_{\parallel}|^2 +|a_{\perp}|^2
\end{eqnarray*}
then
\begin{eqnarray}
\nonumber \vec{a}^2 +\gamma^2 \left(\frac{\vec{v}\cdot \vec{a}}{c}\right)^2  &=& |a_{\parallel}|^2 +|a_{\perp}|^2 +\gamma^2\left(\frac{va_{\parallel}}{c} \right)^2 ~,\\
\nonumber &=& |a_{\perp}|^2 + |a_{\parallel}|^2 (1+\frac{\gamma^2 v^2}{c^2}) ~,\\
&=& |a_{\perp}|^2 + |a_{\parallel}|^2 \gamma^2
\end{eqnarray}
The loss rate can be written
\begin{equation}
\textcolor{red}{ \left(\frac{\dif E}{\dif t} \right)_{S} = \frac{q^2 \gamma^4}{6\pi \epsilon_0 c^3} (|a_{\perp}|^2 +\gamma^2 |a_{\parallel}|^2) }
\end{equation}

\subsection{Parseval’s theorem and the spectral distribution of the radiation of an accelerated electron}
decomposition of the radiation field of the electron into its spectral components.

Parseval’s theorem provides an procedure for relating the kinematic history of the particle to its radiation spectrum.

Introduce the Fourier transform of the acceleration of the particle through the Fourier transform pair:
\begin{eqnarray}
\vec{\dot{v}}(t) &=& \frac{1}{(2\pi)^{1/2}} \int_{-\infty}^{\infty} \vec{\dot{v}}(\omega) \exp(-{\rm i} \omega t) \dif \omega ~,\\
\vec{\dot{v}}(\omega) &=& \frac{1}{(2\pi)^{1/2}} \int_{-\infty}^{\infty} \vec{\dot{v}}(t) \exp({\rm i} \omega t) \dif t
\end{eqnarray}
According to \textcolor{red}{Parseval’s theorem}, $\vec{\dot{v}}(\omega)$ and $\vec{\dot{v}}(t)$ are related by the following integral:
\begin{equation}
\int_{-\infty}^{\infty} |\vec{\dot{v}}(t)|^2 \dif t = \int_{-\infty}^{\infty} |\vec{\dot{v}}(\omega)|^2 \dif \omega
\end{equation}

The energy radiated by a particle which has an acceleration history $\vec{\dot{v}}(t)$:
\begin{eqnarray}
\int_{-\infty}^{\infty} \frac{\dif E}{\dif t} \dif t = \int_{-\infty}^{\infty} \frac{e^2}{6\pi \epsilon_0 c^3} |\vec{\dot{v}}(t)|^2 \dif t = \int_{-\infty}^{\infty} \frac{e^2}{6\pi \epsilon_0 c^3} |\vec{\dot{v}}(\omega)|^2 \dif \omega
\end{eqnarray}
Since the acceleration is a real function,
\begin{equation*}
\int_{0}^{\infty} |\vec{\dot{v}}(\omega)|^2 \dif \omega = \int_{-\infty}^{0} |\vec{\dot{v}}(\omega)|^2 \dif \omega
\end{equation*}
The total emitted radiation is
\begin{equation}
\int_0^{\infty} I(\omega) \dif \omega = \int_{0}^{\infty} \frac{e^2}{3\pi \epsilon_0 c^3} |\vec{\dot{v}}(\omega)|^2 \dif \omega
\end{equation}
The total energy per unit bandwidth emitted throughout the period during which the particle is accelerated :
\begin{equation}
I(\omega) = \frac{e^2}{3\pi \epsilon_0 c^3} |\vec{\dot{v}}(\omega)|^2
\end{equation}
For a distribution of particles, this result must be integrated over all the particles contributing to the radiation at frequency $\omega$.




\section{Bremsstrahlung}
\textcolor{red}{free-free emission}: the radiation corresponds to \textcolor{red}{transitions between unbound states of the electron} in the field of the nucleus.

Applications includes the radio emission of compact regions of ionised hydrogen at temperature $T \approx 10^4$ K, the X-ray emission of binary X-ray sources at $T \approx 10^7$ K and the diffuse X-ray emission of intergalactic gas in clusters of galaxies, which may be as hot as $T \approx 10^8$ K. It is also an important loss mechanism for relativistic cosmic ray electrons.

\section{thermal bremsstrahlung}
\cite{2012agn..book.....B} As both particles are not bound before and after the scattering process, this is also referred to as \textcolor{red}{free–free emission}. The interaction leads to a change in the momentum of the particles involved, and through this acceleration of charges to radiation. This type of scattering takes place for example in \textcolor{red}{hot, but nonrelativistic plasmas}. The most prominent source of extragalactic bremsstrahlung is the \textcolor{red}{intracluster gas} which is \textcolor{red}{bound in the gravitational wells of galaxy clusters} with temperatures of \textcolor{red}{$T \simeq 10^7 - 10^8$ K}. Interaction of two particles with the \textcolor{red}{same mass}, like electron-electron or positron-positron scattering does \textcolor{red}{not produce dipole radiation,and the quadrupole term is comparably low}. Only in the \textcolor{yellow}{relativistic} case can this produce a significant amount of radiation.


%%%%%%%%%%%%%%%%%%%%%%%%%%%%%%%%%%%%%%%%%%%%%%%%%%%%%%%%%%%%%%%%%%%%%%
\bibliographystyle{unsrt_update}
\bibliography{ref}
%%%%%%%%%%%%%%%%%%%%%%%%%%%%%%%%%%%%%%%%%%%%%%%%%%%%%%%%%%%%%%%%%%%%%%

\end{document}