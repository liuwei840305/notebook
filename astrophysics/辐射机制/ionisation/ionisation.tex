\documentclass[12pt,a4paper]{article}
%\usepackage{fontspec, xunicode, xltxtra}  
%\setmainfont{Hiragino Sans GB}  
\usepackage{xeCJK}
%\setCJKmainfont[BoldFont=STZhongsong, ItalicFont=STKaiti]{STSong}
%\setCJKsansfont[BoldFont=STHeiti]{STXihei}
%\setCJKmonofont{STFangsong}

%使用Xelatex编译

% 设置页面
%==================================================
\linespread{2} %行距
% \usepackage[top=1in,bottom=1in,left=1.25in,right=1.25in]{geometry}
% \headsep=2cm
% \textwidth=16cm \textheight=24.2cm
%==================================================

% 其它需要使用的宏包
%==================================================
\usepackage[colorlinks,linkcolor=blue,anchorcolor=red,citecolor=green,urlcolor=blue]{hyperref} 
\usepackage{tabularx}
\usepackage{authblk}         % 作者信息
\usepackage{algorithm}     % 算法排版
\usepackage{amsmath}     % 数学符号与公式
\usepackage{amsfonts}     % 数学符号与字体
\usepackage{mathrsfs}      % 花体
\usepackage{graphics}
\usepackage{color}
\usepackage{fancyhdr}       % 设置页眉页脚
\usepackage{fancyvrb}       % 抄录环境
\usepackage{float}              % 管理浮动体
\usepackage{geometry}     % 定制页面格式
\usepackage{hyperref}       % 为PDF文档创建超链接
\usepackage{lineno}          % 生成行号
\usepackage{listings}        % 插入程序源代码
\usepackage{multicol}       % 多栏排版
%\usepackage{natbib}         % 管理文献引用
\usepackage{rotating}       % 旋转文字,图形,表格
\usepackage{subfigure}    % 排版子图形
\usepackage{titlesec}       % 改变章节标题格式
\usepackage{moresize}   % 更多字体大小
\usepackage{anysize}
\usepackage{indentfirst}  % 首段缩进
\usepackage{booktabs}   % 使用\multicolumn
\usepackage{multirow}    % 使用\multirow
\usepackage{graphicx} 
\usepackage{wrapfig}
\usepackage{xcolor}
\usepackage{titlesec}     % 改变标题样式
\usepackage{enumitem}
\usepackage{aas_macros}
\usepackage{enumerate}

\newcommand{\myvec}[1]%
   {\stackrel{\raisebox{-2pt}[0pt][0pt]{\small$\rightharpoonup$}}{#1}}  %矢量符号
\renewcommand{\vec}[1]{\boldsymbol{#1}}
\newcommand{\me}{\mathrm{e}}
\newcommand{\mi}{\mathrm{i}}
\newcommand{\dif}{\mathrm{d}}
\newcommand{\tabincell}[2]{\begin{tabular}{@{}#1@{}}#2\end{tabular}}

\def\kpc{{\rm kpc}}
\def\km{{\rm km}}
\def\cm{{\rm cm}}
\def\TeV{{\rm TeV}}
\def\GeV{{\rm GeV}}
\def\MeV{{\rm MeV}}
\def\GV{{\rm GV}}
\def\MV{{\rm MV}}
\def\yr{{\rm yr}}
\def\s{{\rm s}}
\def\ns{{\rm ns}}
\def\GHz{{\rm GHz}}
\def\muGs{{\rm \mu Gs}}
\def\arcsec{{\rm arcsec}}
\def\K{{\rm K}}
\def\microK{\mu{\rm K}}
\def\sr{{\rm sr}}
\newcolumntype{p}{D{,}{\pm}{-1}}

\renewcommand{\figurename}{Fig.}
\renewcommand{\tablename}{Tab.}

\renewcommand{\arraystretch}{1.5}

\setlength{\parindent}{0pt}  %取消每段开头的空格

\title{Ionisation losses}
\author{}
\date{\today}
\begin{document}

\maketitle

When high energy particles pass through a solid, liquid or gas, they can cause considerable wreckage to the constituent atoms, molecules and nuclei, including:
\begin{itemize}
\item[1)] the \textcolor{red}{ionisation} and \textcolor{red}{excitation} of the \textcolor{red}{atoms and molecules} of the material. In the process of ionisation, electrons are torn off atoms by the electrostatic forces between the charged high energy particle and the electrons. This is not only a source of \textcolor{red}{ionisation} but also a source of \textcolor{red}{heating of the material} because of the \textcolor{red}{transfer of kinetic energy to the electrons};
\item[2)] the \textcolor{red}{destruction of crystal structures and molecular chains};
\item[3)] \textcolor{red}{nuclear interactions} between the high energy particles and the nuclei of the atoms of the material.
\end{itemize}

\section{non-relativistic treatment}
Consider the collision of a \textcolor{red}{high energy proton or nucleus} with a \textcolor{red}{stationary electron}. Only a very small fraction of the kinetic energy of the high energy particle is transferred to the electron as can be appreciated from the case of a head-on collision of a high energy particle of mass $M$ and velocity $v$ with an electron of mass $m_{\rm e}$. Taking the particles to
be solid spheres, the maximum velocity acquired by the electron in a non-relativistic collision is $[2M/(M + m_{\rm e})]v$. $m_{\rm e}  \ll M$, this is approximately $2v$. The loss of kinetic energy of the high energy particle is less than $m_{\rm e} (2v)^2/2 = 2m_{\rm e}v^2$ and its fractional kinetic energy loss is less than $m_{\rm e} (2v)^2/(Mv^2) = 4m_{\rm e}/M$. $M \gg m_{\rm e}$, the fractional loss of energy per collision is very small. 

The electrons of the medium receive a small momentum impulse through the electrostatic attraction or repulsion of the high energy particle.

High energy particle is assumed to move so fast that its trajectory is undeviated and the electron remains stationary during the interaction.

The charge of the high energy particle is $ze$ and its mass $M$; \textcolor{red}{collision parameter $b$}, is the distance of closest approach of the particle to the electron. The \emph{total momentum impulse} given to the electron in this encounter is $\int F \dif t$. By symmetry, the forces parallel to the line of flight of the high energy particle cancel out and only work out the component of force perpendicular to the line of flight.
\begin{eqnarray}
\nonumber F_{\perp} &=& \frac{ze^2}{4\pi \epsilon_0 r^2} \sin \theta ~,\\
\nonumber \dif t &=& \frac{\dif x}{v}
\end{eqnarray}
The \textcolor{red}{momentum impulse} is 
\begin{eqnarray*}
\int_{-\infty}^{\infty} F \dif t &=& -\int_{0}^{\pi} \frac{ze^2}{4\pi \epsilon_0 b^2} \sin^2 \theta \frac{b\sin \theta}{v\sin^2 \theta} \dif \theta ~, \\
&=& -\frac{ze^2}{4\pi \epsilon_0 b v} \int_{0}^{\pi} \sin \theta \dif \theta ~, \\
&=& \frac{ze^2}{2\pi \epsilon_0 b v}
\end{eqnarray*}
The \textcolor{red}{kinetic energy transferred to the electron} is
\begin{equation}
\frac{p^2}{2m_{\rm e}} = \textcolor{red}{\frac{z^2 e^4}{8\pi^2 \epsilon^2_0 b^2 v^2 m_{\rm e}} }
\end{equation}
i.e. \textcolor{red}{energy loss by high energy particle}.

\subsection{average energy loss per unit path length}
The number of encounters with collision parameters in the range $b$ to $b + \dif b$ and integrate over collision parameters. 

The total energy loss of the high energy particle, $-\dif E$, in length $\dif x$ is number of electrons in volume $2\pi b ~\dif b ~\dif  x~ \times$ energy loss per interaction:
\begin{eqnarray}
\nonumber -\left(\frac{\dif E}{\dif x} \right) &=& \frac{z^2 e^4 N_{\rm e} }{8\pi^2 \epsilon^2_0 v^2 m_{\rm e}} \times \int_{b_{\rm min}}^{b_{\rm max}} \frac{2\pi b}{b^2} \dif b ~,\\
&=& \frac{z^2 e^4 N_{\rm e} }{4\pi \epsilon^2_0 v^2 m_{\rm e}} \ln \left(\frac{b_{\rm max}}{b_{\rm min}} \right)
\end{eqnarray}
$N_{\rm e}$ is the number density, or concentration, of electrons. The closer the encounter, the greater the momentum impulse, $p \propto b^{-2}$. However, there are more electrons at large distances ($\propto b \dif b$).  



\section{The relativistic case}

\subsection{The relativistic transformation of an inverse square law Coulomb field}


\subsection{Relativistic ionisation losses}


\subsection{Relativistic collision between a high energy particle and a stationary electron}
The momentum four-vectors of the high energy particle and the electron in the laboratory frame of reference are
\begin{eqnarray*}
&&{\rm high ~energy ~particle} ~~ [\gamma M, \gamma M \vec{v}] = [\gamma M, \gamma M v, 0, 0] \\
&&{\rm electron} ~~ [m_{\rm e}, 0, 0, 0]
\end{eqnarray*}
Transform both four-vectors into a frame of reference moving at velocity $V_{\rm F}$, for which the Lorentz factor is $\gamma_{\rm F} = (1 − V_{\rm F}^2/c^2)^{-1/2}$ and $V_{\rm F} \parallel \vec{v}$. The relativistic three-momenta are:
\begin{eqnarray*}
&&{\rm high ~energy ~particle} ~~ (\gamma M v)^{\prime} = \gamma_{\rm F}\gamma M (v- V_{\rm F}) \\
&&{\rm electron} ~~ p^{\prime}_{\rm e} = \gamma_{\rm F}(0-V_{\rm F} m_{\rm e})
\end{eqnarray*}
In the centre of momentum frame $(\gamma Mv)^{\prime} + p^{\prime}_{\rm e}  = 0$, 
\begin{equation}
V_{\rm F} = \frac{\gamma M v}{m_{\rm e} +\gamma M}
\end{equation}
In this frame of reference, the relativistic three-momentum of the electron is $-\gamma_{\rm F}V_{\rm F} m_{\rm e}$, that is, the particle is travelling in the negative $x^{\prime}$-direction. The \textcolor{red}{maximum energy exchange} is obtained if the electron is sent back along the positive $x^{\prime}$-direction following the collision. Since the collision is elastic, its three-momentum is $+\gamma_{\rm F}V_{\rm F} m_{\rm e}$ and the zeroth component of the four-vector, the \textcolor{red}{total energy, is unchanged in the centre of momentum frame of reference}. Next transform the four-momentum $[\gamma_{\rm F} m_{\rm e}, \gamma_{\rm F} V_{\rm F} m_{\rm e}, 0, 0]$ back into the laboratory frame of reference. Transforming the zeroth component of the momentum four-vector using the inverse Lorentz transformation,
\begin{equation*}
(\gamma m_{\rm e})_{\rm in ~S} = \gamma_{\rm F}^2 m_{\rm e} \left(1 +\frac{V_{\rm F}^2}{c^2} \right)
\end{equation*}
The total energy in S is $\gamma_{\rm F}^2 m_{\rm e} c^2(1 + V_{\rm F}^2/c^2)$ and maximum kinetic energy of the electron is
\begin{equation}
\gamma_{\rm F}^2 m_{\rm e} c^2 \left(1 +\frac{V_{\rm F}^2}{c^2} \right) -m_{\rm e} c^2 = 2\frac{V_{\rm F}^2}{c^2} \gamma_{\rm F}^2 m_{\rm e} c^2
\end{equation}
$m_{\rm e} \ll \gamma M$, hence $V_{\rm F} \approx v, \gamma_{\rm F} \approx \gamma$. In the \textcolor{red}{ultra-relativistic limit}, the \textcolor{red}{maximum energy transfer to the electron} is
\begin{equation*}
E_{\rm max} = 2\gamma^2 m_{\rm e} v^2
\end{equation*}





%%%%%%%%%%%%%%%%%%%%%%%%%%%%%%%%%%%%%%%%%%%%%%%%%%%%%%%%%%%%%%%%%%%%%%
\bibliographystyle{unsrt_update}
\bibliography{ref}
%%%%%%%%%%%%%%%%%%%%%%%%%%%%%%%%%%%%%%%%%%%%%%%%%%%%%%%%%%%%%%%%%%%%%%

\end{document}