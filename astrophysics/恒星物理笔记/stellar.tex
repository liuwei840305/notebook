\documentclass[12pt,a4paper]{article}
%\usepackage{fontspec, xunicode, xltxtra}  
%\setmainfont{Hiragino Sans GB}  
\usepackage{xeCJK}
%\setCJKmainfont[BoldFont=STZhongsong, ItalicFont=STKaiti]{STSong}
%\setCJKsansfont[BoldFont=STHeiti]{STXihei}
%\setCJKmonofont{STFangsong}

%使用Xelatex编译

% 设置页面
%==================================================
\linespread{2} %行距
% \usepackage[top=1in,bottom=1in,left=1.25in,right=1.25in]{geometry}
% \headsep=2cm
% \textwidth=16cm \textheight=24.2cm
%==================================================

% 其它需要使用的宏包
%==================================================
\usepackage[colorlinks,linkcolor=blue,anchorcolor=red,citecolor=green,urlcolor=blue]{hyperref} 
\usepackage{tabularx}
\usepackage{authblk}         % 作者信息
\usepackage{algorithm}     % 算法排版
\usepackage{amsmath}     % 数学符号与公式
\usepackage{amsfonts}     % 数学符号与字体
\usepackage{mathrsfs}      % 花体
\usepackage{graphics}
\usepackage{color}
\usepackage{fancyhdr}       % 设置页眉页脚
\usepackage{fancyvrb}       % 抄录环境
\usepackage{float}              % 管理浮动体
\usepackage{geometry}     % 定制页面格式
\usepackage{hyperref}       % 为PDF文档创建超链接
\usepackage{lineno}          % 生成行号
\usepackage{listings}        % 插入程序源代码
\usepackage{multicol}       % 多栏排版
\usepackage{natbib}         % 管理文献引用
\usepackage{rotating}       % 旋转文字,图形,表格
\usepackage{subfigure}    % 排版子图形
\usepackage{titlesec}       % 改变章节标题格式
\usepackage{moresize}   % 更多字体大小
\usepackage{anysize}
\usepackage{indentfirst}  % 首段缩进
\usepackage{booktabs}   % 使用\multicolumn
\usepackage{multirow}    % 使用\multirow
\usepackage{graphicx} 
\usepackage{wrapfig}
\usepackage{xcolor}
\usepackage{titlesec}     % 改变标题样式
\usepackage{enumitem}

\newcommand{\myvec}[1]%
   {\stackrel{\raisebox{-2pt}[0pt][0pt]{\small$\rightharpoonup$}}{#1}}  %矢量符号
\renewcommand{\vec}[1]{\boldsymbol{#1}}
\newcommand{\me}{\mathrm{e}}
\newcommand{\mi}{\mathrm{i}}
\newcommand{\dif}{\mathrm{d}}
\newcommand{\tabincell}[2]{\begin{tabular}{@{}#1@{}}#2\end{tabular}}

\def\kpc{{\rm kpc}}
\def\km{{\rm km}}
\def\cm{{\rm cm}}
\def\TeV{{\rm TeV}}
\def\GeV{{\rm GeV}}
\def\MeV{{\rm MeV}}
\def\GV{{\rm GV}}
\def\MV{{\rm MV}}
\def\yr{{\rm yr}}
\def\s{{\rm s}}
\def\ns{{\rm ns}}
\def\GHz{{\rm GHz}}
\def\muGs{{\rm \mu Gs}}
\def\arcsec{{\rm arcsec}}
\def\K{{\rm K}}
\def\microK{\mu{\rm K}}
\def\sr{{\rm sr}}
\newcolumntype{p}{D{,}{\pm}{-1}}

\renewcommand{\figurename}{Fig.}
\renewcommand{\tablename}{Tab.}

\renewcommand{\arraystretch}{1.5}

\setlength{\parindent}{0pt}  %取消每段开头的空格

\title{恒星物理}
\author{}
\date{\today}
\begin{document}

\maketitle

\section{光度与颜色}
恒星每秒钟辐射的能量叫做\textcolor{red}{光度}。恒星的亮度用\textcolor{red}{星等}表示。在全波段的光度$L$和\textcolor{red}{绝对星等$M_{\rm bol}$}之间的关系定义为
\begin{equation}
M_{\rm bol} = 4.72 -2.5 \log \left(\frac{L}{L_{\odot}} \right) ~.
\end{equation}


联系两个天体的亮度与星等的关系由Pogson公式给出
\begin{equation}
m_2 -m_1 = -2.5 \lg \frac{E_2}{E_1}
\end{equation}
或者
\begin{equation}
\frac{E_2}{E_1} = 2.512^{m_1-m_2}
\end{equation}
现测定$1$标准烛光在$1$m处的照度($1$lx),其视星等为$-13^{\rm m}.98$,或者说零等星的照度为$2.54\times 10^{-6}$ lx。太阳的视星等$m_{\rm V} = -26^{\rm m}.74$,满月的$m_{\rm V} = -12^{\rm m}.74$。

天体的照度$E$,发光强度$I$和到天体距离$r$的关系为
\begin{equation}
E = \frac{I}{r^2}
\end{equation}
故视星等表示为
\begin{equation}
m = -2.5\lg I + 5\lg r
\end{equation}

天体的\textcolor{red}{绝对星等$M$}定义为\textcolor{cyan}{位于$10$ pc处天体的视星等}。天体的视星等$m$和绝对星等$M$与距离$r$ (以pc为单位) 或周年视差$\pi$ (以角秒为单位) 的关系为
\begin{eqnarray}
M &=& m +5 -5\lg r -A ~, \\
M &=& m +5 +5\lg \pi -A ~.
\end{eqnarray}
$A$为星际消光。视星等与绝对星等之差就是天体距离的度量,称为\textcolor{red}{距离模数},即
\begin{equation}
m - M= 5\lg r - 5 ~.
\end{equation}


1951年,H.L. Johnson和W. W. Morgan提出\textcolor{red}{$UBV$三色测光系统}作为星等和颜色的标准。$U$、$B$、$V$分别表示紫外、蓝色和目视的意思。上述三种星等$U$、$B$、$V$规定对于$A0$型星 ($A0$型主序星) 有
\begin{equation}
U = B = V ~.
\end{equation}
表面温度比$A0$型星高的恒星,$U-B$和$B-V$取较小的负值,低温星具有较大的正值。\textcolor{red}{$B-V$成为衡量恒星表面温度的尺度},称为\textcolor{red}{色指数}。

研究全波段辐射问题时,须考虑$V$和辐射星等的差,这个差叫做\textcolor{red}{热改正 B.C.},即
\begin{equation}
{\rm B.C.} = m_V - m_{\rm bol} = M_V - M_{\rm bol} ~.
\end{equation}
对于地球大气和星际气体吸收而观测不到的紫外和红外波段,可用黑体辐射分布的假定而求出各个光谱型星的B.C.的值。规定F5型星的B.C.$=0$。B.C.通常取正值,随着温度的升高或降低而热改正变得不准确。

遥远的星光,在到达地面之前,由于星际气体中的粒子吸收而散射。波长越短,吸收越大,叫做\textcolor{red}{星际红花}。如果测定两个独立的色指数B-V和U-B,同时又知道星际红花值,可以求出实际的表面温度。








\section{质量}
只有在双星情况下才能直接测定恒星的质量。

\section{The Hertzsprung–Russell and Color-Magnitude Diagrams}
When the \textcolor{red}{absolute brightness of a set of stars} is plotted against \textcolor{red}{surface temperature}, as measured either from an analysis of the stellar spectrum or by the star’s color, only certain portions of such \textcolor{red}{Hertzsprung-Russell} and \textcolor{red}{color-magnitude} diagrams are populated appreciably.







\section{The Formation of Stars}

\section{静力学平衡}



\section{多方过程}




\section{Lane-Emden方程}




















\end{document}