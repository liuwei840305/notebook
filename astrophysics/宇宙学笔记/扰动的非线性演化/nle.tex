\documentclass[12pt,a4paper]{article}
%\usepackage{fontspec, xunicode, xltxtra}  
%\setmainfont{Hiragino Sans GB}  
\usepackage{xeCJK}
%\setCJKmainfont[BoldFont=STZhongsong, ItalicFont=STKaiti]{STSong}
%\setCJKsansfont[BoldFont=STHeiti]{STXihei}
%\setCJKmonofont{STFangsong}

%使用Xelatex编译

% 设置页面
%==================================================
\linespread{2} %行距
% \usepackage[top=1in,bottom=1in,left=1.25in,right=1.25in]{geometry}
% \headsep=2cm
% \textwidth=16cm \textheight=24.2cm
%==================================================

% 其它需要使用的宏包
%==================================================
\usepackage[colorlinks,linkcolor=blue,anchorcolor=red,citecolor=green,urlcolor=blue]{hyperref} 
\usepackage{tabularx}
\usepackage{authblk}         % 作者信息
\usepackage{algorithm}     % 算法排版
\usepackage{amsmath}     % 数学符号与公式
\usepackage{amsfonts}     % 数学符号与字体
\usepackage{graphics}
\usepackage{color}
\usepackage{fancyhdr}       % 设置页眉页脚
\usepackage{fancyvrb}       % 抄录环境
\usepackage{float}              % 管理浮动体
\usepackage{geometry}     % 定制页面格式
\usepackage{hyperref}       % 为PDF文档创建超链接
\usepackage{lineno}          % 生成行号
\usepackage{listings}        % 插入程序源代码
\usepackage{multicol}       % 多栏排版
\usepackage{natbib}         % 管理文献引用
\usepackage{rotating}       % 旋转文字,图形,表格
\usepackage{subfigure}    % 排版子图形
\usepackage{titlesec}       % 改变章节标题格式
\usepackage{moresize}   % 更多字体大小
\usepackage{anysize}
\usepackage{indentfirst}  % 首段缩进
\usepackage{booktabs}   % 使用\multicolumn
\usepackage{multirow}    % 使用\multirow
\usepackage{graphicx} 
\usepackage{wrapfig}
\usepackage{xcolor}
\usepackage{titlesec}     % 改变标题样式
\usepackage{enumitem}

\renewcommand{\vec}[1]{\boldsymbol{#1}}
\newcommand{\me}{\mathrm{e}}
\newcommand{\mi}{\mathrm{i}}
\newcommand{\dif}{\mathrm{d}}
\newcommand{\tabincell}[2]{\begin{tabular}{@{}#1@{}}#2\end{tabular}}

\def\kpc{{\rm kpc}}
\def\km{{\rm km}}
\def\cm{{\rm cm}}
\def\TeV{{\rm TeV}}
\def\GeV{{\rm GeV}}
\def\MeV{{\rm MeV}}
\def\GV{{\rm GV}}
\def\MV{{\rm MV}}
\def\yr{{\rm yr}}
\def\s{{\rm s}}
\def\ns{{\rm ns}}
\def\GHz{{\rm GHz}}
\def\muGs{{\rm \mu Gs}}
\def\arcsec{{\rm arcsec}}
\def\K{{\rm K}}
\def\microK{\mu{\rm K}}
\def\sr{{\rm sr}}
\newcolumntype{p}{D{,}{\pm}{-1}}

\renewcommand{\figurename}{Fig.}
\renewcommand{\tablename}{Tab.}

\renewcommand{\arraystretch}{1.5}

\setlength{\parindent}{0pt}  %取消每段开头的空格

\title{密度扰动的线性演化}
\author{}
\date{\today}
\begin{document}

\maketitle

$10 < z < 1000$:黑暗时代(Dark Ages)

\section{球对称塌缩模型}




\section{Press-Schechter质量函数}


\section{Zel'dovich近似}

对于一个三轴椭球状的扰动,坍缩过程不会终结到一个点,而是终结到一个准二维的平展结构,通常称为“薄饼”模型。假设压力可以忽略,流体可以看成是由大量尘埃质点组成的,并假设在焦散线(caustics)处,密度可以达到无穷大,但这些区域引起的引力加速度保持为有限。在这些近似下,粒子的位置坐标为
\begin{equation}
\vec{r} (t, \vec{q}) = a(t) [\vec{q} -b(t) f(\vec{q})] 
\end{equation}
$\vec{r}$:Eular位置坐标,即固有坐标;
$a(t)$:宇宙尺度因子,$a(t) \propto t^{2/3}$;
$\vec{q}$:Lagrange位置坐标,相当于无扰动时粒子的初始共动坐标;
在有扰动情况下粒子的共动坐标:$\vec{x} = \vec{q} -b(t) f(\vec{q})$,$f(\vec{q})$:扰动产生的与时间无关的位移场的函数,$b(t)$:位移的线性演化,且在扰动开始时刻$t_i$,$b(t_i) = 0$,并满足演化方程
\begin{equation}
\ddot{b} - 2\frac{\dot{a}}{a} \dot{b} - 4\pi G \rho b = 0
\end{equation}
$b(t)$随时间增长的规律是$b(t) \propto t^{2/3}$;
再设扰动位移场是无旋的,可以把它写成某个势函数(速度势函数)的梯度:
\begin{equation}
f(\vec{q}) = \nabla_{q} \Psi(\vec{q})
\end{equation}
在共动坐标下,粒子的本动速度
\begin{equation}
\vec{u} \equiv \dot{\vec{x}} = \frac{1}{a} \left( \frac{\dif \vec{r}}{\dif t} -H\vec{r}\right) = -\dot{b} f(\vec{q})
\end{equation}
速度场也是无旋的;

在扰动位移为线性时,通过$\vec{r}$与$\vec{q}$之间坐标变换的Jacobi行列式$|J(\vec{r}, t)| = |\partial \vec{r}/\partial \vec{q}|$,扰动密度和平均背景密度之间的守恒关系$\rho(\vec{r}, t) \dif^3 \vec{r} = \bar{\rho}(t) \dif^3 \vec{q}$可表示为
\begin{equation}
\rho(\vec{r}, t) = \frac{\bar{\rho}(t)}{|J(\vec{r}, t)|}
\end{equation}
或
\begin{equation}
\frac{\rho}{\bar{\rho}} = \frac{1}{[1+b(t)\alpha_1][1+b(t)\alpha_2][1+b(t)\alpha_3]}
\end{equation}
$1+b(t)\alpha_i$:矩阵$J$的本征值,$\alpha_i$:应变张量(形变张量)$\partial f_i/\partial q_j$的本征值;

Zel'dovich近似是对粒子的位移取一阶线性近似,而不是直接对密度扰动取线性近似;位移被认为是线性变化的,而密度的变化可能是线性的,也可能是非线性的;
当$|b(t) \alpha_i| \ll 1$时,密度扰动为
\begin{equation}
\delta \approx -(\alpha_1+\alpha_2+\alpha_3) b(t) = b \nabla\cdot f = b \nabla^2_{\vec{q}} \Psi
\end{equation}
若$\alpha_i$为负值,则当扰动增大到使得$b(t) = -\frac{1}{\alpha_i}$时,上式密度为无穷大,即形成奇点,称为\textcolor{red}{壳层交叉}(shell crossing);意味着Lagrange坐标不同的两个粒子(或多个粒子)碰到一起了,具有相同的Eular坐标,粒子的轨道交叉了,坐标变换不是一对一的了;发生壳层交叉的地方称为\textcolor{red}{焦散线}(\textcolor{red}{焦散面});
发生坍缩的地方,至少有一个$\alpha_i$为负值,若不止一个$\alpha_i$为负,则坍缩首先发生在最负的$\alpha_i$相应的轴方向,从而形成“薄饼”结构;在很少的情况下,当两个或三个$\alpha_i$相同时,也可能在两个或三个方向同时坍缩,从而形成“纤维”状或“点”状结构。Zel'dovich近似的结果是,薄饼状结构是引力坍缩的最一般形式。
























%%%%%%%%%%%%%%%%%%%%%%%%%%%%%%%%%%%%%%%%%%%%%%%%%%%%%%%%%%%%%%%%%%%%%%
\bibliographystyle{unsrt_update}
\bibliography{ref}
%%%%%%%%%%%%%%%%%%%%%%%%%%%%%%%%%%%%%%%%%%%%%%%%%%%%%%%%%%%%%%%%%%%%%%



\end{document}