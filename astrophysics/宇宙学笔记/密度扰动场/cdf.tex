\documentclass[12pt,a4paper]{article}
%\usepackage{fontspec, xunicode, xltxtra}  
%\setmainfont{Hiragino Sans GB}  
\usepackage{xeCJK}
%\setCJKmainfont[BoldFont=STZhongsong, ItalicFont=STKaiti]{STSong}
%\setCJKsansfont[BoldFont=STHeiti]{STXihei}
%\setCJKmonofont{STFangsong}

%使用Xelatex编译

% 设置页面
%==================================================
\linespread{2} %行距
% \usepackage[top=1in,bottom=1in,left=1.25in,right=1.25in]{geometry}
% \headsep=2cm
% \textwidth=16cm \textheight=24.2cm
%==================================================

% 其它需要使用的宏包
%==================================================
\usepackage[colorlinks,linkcolor=blue,anchorcolor=red,citecolor=green,urlcolor=blue]{hyperref} 
\usepackage{tabularx}
\usepackage{authblk}         % 作者信息
\usepackage{algorithm}     % 算法排版
\usepackage{amsmath}     % 数学符号与公式
\usepackage{amsfonts}     % 数学符号与字体
\usepackage{graphics}
\usepackage{color}
\usepackage{fancyhdr}       % 设置页眉页脚
\usepackage{fancyvrb}       % 抄录环境
\usepackage{float}              % 管理浮动体
\usepackage{geometry}     % 定制页面格式
\usepackage{hyperref}       % 为PDF文档创建超链接
\usepackage{lineno}          % 生成行号
\usepackage{listings}        % 插入程序源代码
\usepackage{multicol}       % 多栏排版
\usepackage{natbib}         % 管理文献引用
\usepackage{rotating}       % 旋转文字,图形,表格
\usepackage{subfigure}    % 排版子图形
\usepackage{titlesec}       % 改变章节标题格式
\usepackage{moresize}   % 更多字体大小
\usepackage{anysize}
\usepackage{indentfirst}  % 首段缩进
\usepackage{booktabs}   % 使用\multicolumn
\usepackage{multirow}    % 使用\multirow
\usepackage{graphicx} 
\usepackage{wrapfig}
\usepackage{xcolor}
\usepackage{titlesec}     % 改变标题样式
\usepackage{enumitem}

\renewcommand{\vec}[1]{\boldsymbol{#1}}
\newcommand{\me}{\mathrm{e}}
\newcommand{\mi}{\mathrm{i}}
\newcommand{\dif}{\mathrm{d}}
\newcommand{\tabincell}[2]{\begin{tabular}{@{}#1@{}}#2\end{tabular}}

\def\kpc{{\rm kpc}}
\def\km{{\rm km}}
\def\cm{{\rm cm}}
\def\TeV{{\rm TeV}}
\def\GeV{{\rm GeV}}
\def\MeV{{\rm MeV}}
\def\GV{{\rm GV}}
\def\MV{{\rm MV}}
\def\yr{{\rm yr}}
\def\s{{\rm s}}
\def\ns{{\rm ns}}
\def\GHz{{\rm GHz}}
\def\muGs{{\rm \mu Gs}}
\def\arcsec{{\rm arcsec}}
\def\K{{\rm K}}
\def\microK{\mu{\rm K}}
\def\sr{{\rm sr}}
\newcolumntype{p}{D{,}{\pm}{-1}}

\renewcommand{\figurename}{Fig.}
\renewcommand{\tablename}{Tab.}

\renewcommand{\arraystretch}{1.5}

\setlength{\parindent}{0pt}  %取消每段开头的空格

\title{密度扰动场}
\author{}
\date{\today}
\begin{document}

\maketitle

宇宙学的密度扰动通常假设为Gauss随机场;非线性结构形成于该场中的局域极大处;

\section{Gauss随机场}
宇宙密度涨落
\begin{equation}
\delta(\vec{x}, t) \equiv \frac{\rho(\vec{x}, t) -\left\langle \rho \right\rangle}{\left\langle \rho \right\rangle}
\end{equation}

原初密度涨落$\delta(\vec{x}, t_i)$的空间分布定义了一个三维随机场,$t_i$相应于暴涨结束的时刻;

根据宇宙学原理,这个随机场应当是均匀各项同性的;

由中心极限定理,大量独立的随机变量(事件)相加的结果趋于正态分布或Gauss分布;故密度扰动场应当是Gauss场,其平均值为0。

一个Gauss随机场的统计性质可以用\textcolor{red}{功率谱$P(k)$}来表征,或功率谱的Fourier变换,\textcolor{red}{自相关函数$\xi(x)$}

一个Gauss随机场若进行空间Fourier分解,则它的各谐频分量$\delta_k$的位相是相互独立的,即不同分量的位相在$0$到$2\pi$之间随机分布;

设Gauss随机场$F(\vec{x})$,它是三维空间位置坐标$\vec{x}$的函数,若该场在尺度为$L$的方盒子内是周期函数,则利用Fourier分析,可以把它展开为一系列平面波的迭加
\begin{equation}
F(\vec{x}) = \sum F_k e^{-i\vec{k}\cdot \vec{x}}
\end{equation}
其中波数满足谐和边界条件
\begin{equation}
k_x = n\frac{2\pi}{L}, n = 1, 2, \cdots
\end{equation}
当盒子的大小变为无穷大,即$L\rightarrow \infty$,
\begin{eqnarray}
\nonumber F(\vec{x}) &=& \left(\frac{L}{2\pi} \right)^3  \int F_k(\vec{k}) \exp(-i\vec{k}\cdot \vec{x}) \dif^3 k, \\
F_k(\vec{k}) &=& \frac{1}{L^3}  \int F(\vec{x}) \exp(i\vec{k}\cdot \vec{x}) \dif^3 x,
\end{eqnarray}

一个$n$维随机场$F(\vec{x})$是一组随机变量的集合,其中每个变量都是$n$维空间位置矢量的$\vec{x}$的函数。设有$m$个随机变量$y_i$,它们的联合Gauss几率分布是(BBKS)
\begin{equation}
P(y_1, y_2, \cdots, y_m) \dif y_1 \cdots \dif y_m = \frac{e^{-Q}}{[(2\pi)^m |M|]^{1/2}} \dif y_1 \cdots \dif y_m 
\end{equation}
\begin{eqnarray}
Q \equiv \frac{1}{2} \sum \Delta y_i (M^{-1})_{ij} \Delta y_j, \\ 
M_{ij} = \left\langle \Delta y_i \Delta y_j \right\rangle, ~ \Delta y_i \equiv y_i -\left\langle y_i \right\rangle
\end{eqnarray}
其中$i,j = 1, \cdots, m$,且$|M|$:协方差矩阵$M_{ij}$的行列式,

根据Gauss场的统计性质,若$F(\vec{x})$是一个Gauss场,则由该场及其导数、积分以及线性函数构成的联合分布也是Gauss的



\section{傅里叶变换的性质}

\subsection{$\delta$函数的Fourier变换}
\begin{equation}
\delta(\vec{x}) = \left(\frac{L}{2\pi} \right)^3 \int \dif^3 k e^{-i\vec{k}\cdot \vec{x}} 
\end{equation}


\subsection{Parseval定理}
\begin{eqnarray}
\nonumber \int \dif^3 x |F(\vec{x})|^2 &=& \frac{1}{(2\pi)^6} \int \dif^3 x\int \dif^3 k F_k(\vec{k}) e^{-i\vec{k} \cdot \vec{x}} \int \dif^3 k' F^*_k(\vec{k}') e^{i\vec{k}' \cdot \vec{x}}, \\
\nonumber &=& \frac{1}{(2\pi)^6} \int \dif^3 k \int \dif^3 k'  F_k(\vec{k}) F^*_k(\vec{k}') \int \dif^3 x e^{-i(\vec{k}-\vec{k}') \cdot \vec{x}}, \\
\nonumber &=& \frac{1}{(2\pi)^3} \int \dif^3 k \int \dif^3 k'  F_k(\vec{k}) F^*_k(\vec{k}') \delta(\vec{k}-\vec{k}'), \\
&=& \frac{1}{(2\pi)^3} \int \dif^3 k |F_k(\vec{k})|^2
\end{eqnarray}
$\int \dif^3 x |F(\vec{x})|^2$:场$F(\vec{x})$的总功率,

$|F_k(\vec{k})|^2$:相应的功率谱,

各项同性的功率谱,$|F_k(\vec{k})|^2 = |F_k(k)|^2 \equiv P(k)$

\subsection{卷积定理}
两个连续函数$f(\vec{x})$、$g(\vec{x})$的卷积
\begin{equation}
c(\vec{x}) = \int \dif^3 x' f(\vec{x}') g(\vec{x} -\vec{x}')
\end{equation}
它的Fourier变换
\begin{eqnarray}
\nonumber c_k(\vec{k}) &=& \int \dif^3 x c(\vec{x}) e^{i\vec{k} \cdot \vec{x}}, \\
\nonumber &=& \int \dif^3 x \int \dif^3 x' f(\vec{x}') g(\vec{x}-\vec{x}') e^{i\vec{k} \cdot \vec{x}}, \\
\nonumber &=& \int \dif^3 x \int \dif^3 x' \frac{1}{(2\pi)^3} \int \dif^3 k' \frac{1}{(2\pi)^3} \int \dif^3 k'' f_k(\vec{k}') g_k(\vec{k}'') e^{i[\vec{k}\cdot \vec{x} -\vec{k}'\cdot \vec{x}' -\vec{k}''\cdot (\vec{x}-\vec{x}') ]}, \\
\nonumber &=& \int \dif^3 k' \int \dif^3 k'' f_k(\vec{k}') g_k(\vec{k}'') \delta(\vec{k}-\vec{k}'') \delta(\vec{k}''-\vec{k}'), \\
&=& f_k(\vec{k}) g_k(\vec{k})
\end{eqnarray}
两个函数卷积的Fourier变换等于两函数各自Fourier变换的乘积


\subsection{相关函数和功率谱(Wiener-Khinchin定理)}

函数$F(\vec{x})$的\textcolor{red}{自相关函数},简称\textcolor{red}{相关函数},定义为
\begin{equation}
\xi(\vec{r}) =  \left\langle F(\vec{x}) F(\vec{x}+\vec{r}) \right\rangle 
\end{equation}
$\left\langle \cdot \right\rangle$:对函数$F(\vec{x})$的定义空间进行平均,即
\begin{eqnarray}
\nonumber \xi(\vec{r}) &=& \frac{1}{L^3} \int \dif^3 x F(\vec{x}) F(\vec{x}+\vec{r}), \\
\nonumber &=& \frac{1}{L^3} \int \dif^3 x \int \left(\frac{L}{2\pi} \right)^3 \dif^3 k F_k(\vec{k}) e^{-i\vec{k}\cdot \vec{x}} \int \left(\frac{L}{2\pi} \right)^3 \dif k' F_k(\vec{k}') e^{-i\vec{k}'\cdot (\vec{x}+\vec{r})}, \\
\nonumber &=& \left(\frac{L}{2\pi} \right)^3 \int \dif^3 k \int \dif^3 k' \frac{1}{(2\pi)^3} F_k(\vec{k}) F_k(\vec{k}') e^{-i\vec{k}'\cdot \vec{r}} \int \dif^3 x e^{-i(\vec{k}' + \vec{k})\cdot \vec{x}}, \\
\nonumber &=& \left(\frac{L}{2\pi} \right)^3 \int \dif^3 k F_k(-\vec{k}) F_k(\vec{k}) e^{-i\vec{k}'\cdot \vec{r}}, \\
&=& \left(\frac{L}{2\pi} \right)^3 \int \dif^3 k |F_k(\vec{k})|^2 e^{-i\vec{k}'\cdot \vec{r}}
\end{eqnarray}
相关函数和功率谱是Fourier变换对。

在场为各项同性时,$|F_k(\vec{k})|^2 = P(k)$,且$\xi$具有转动不变性,
\begin{eqnarray}
\nonumber \xi(r) &=& \frac{L^3}{(2\pi)^3} \int P(k) \cos(kr \cos \theta) 2\pi k^2 \sin \theta \dif \theta \dif k, \\
&=& \frac{L^3}{2\pi^2} \int P(k) \frac{\sin kr}{kr} k^2 \dif k
\end{eqnarray}

\subsection{导数和积分的Fourier变换}




\subsection{Fourier延迟定理}


\subsection{Fourier位移定理}











































































































%%%%%%%%%%%%%%%%%%%%%%%%%%%%%%%%%%%%%%%%%%%%%%%%%%%%%%%%%%%%%%%%%%%%%%
\bibliographystyle{unsrt_update}
\bibliography{ref}
%%%%%%%%%%%%%%%%%%%%%%%%%%%%%%%%%%%%%%%%%%%%%%%%%%%%%%%%%%%%%%%%%%%%%%



\end{document}