\documentclass[12pt,a4paper]{article}
%\usepackage{fontspec, xunicode, xltxtra}  
%\setmainfont{Hiragino Sans GB}  
%\usepackage{xeCJK}
%\setCJKmainfont[BoldFont=STZhongsong, ItalicFont=STKaiti]{STSong}
%\setCJKsansfont[BoldFont=STHeiti]{STXihei}
%\setCJKmonofont{STFangsong}

%使用Xelatex编译

% 设置页面
%==================================================
\linespread{2} %行距
% \usepackage[top=1in,bottom=1in,left=1.25in,right=1.25in]{geometry}
% \headsep=2cm
% \textwidth=16cm \textheight=24.2cm
%==================================================

% 其它需要使用的宏包
%==================================================
\usepackage[colorlinks,linkcolor=blue,anchorcolor=red,citecolor=green,urlcolor=blue]{hyperref} 
\usepackage{tabularx}
\usepackage{authblk}         % 作者信息
\usepackage{algorithm}     % 算法排版
\usepackage{amsmath}     % 数学符号与公式
\usepackage{amssymb}
\usepackage{amsfonts}     % 数学符号与字体
\usepackage{mathrsfs}      % 花体
\usepackage{graphics}
\usepackage{color}
\usepackage{fancyhdr}       % 设置页眉页脚
\usepackage{fancyvrb}       % 抄录环境
\usepackage{float}              % 管理浮动体
\usepackage{geometry}     % 定制页面格式
\usepackage{hyperref}       % 为PDF文档创建超链接
\usepackage{lineno}          % 生成行号
\usepackage{listings}        % 插入程序源代码
\usepackage{multicol}       % 多栏排版
\usepackage{rotating}       % 旋转文字,图形,表格
\usepackage{subfigure}    % 排版子图形
\usepackage{titlesec}       % 改变章节标题格式
\usepackage{moresize}   % 更多字体大小
\usepackage{anysize}
\usepackage{indentfirst}  % 首段缩进
\usepackage{booktabs}   % 使用\multicolumn
\usepackage{multirow}    % 使用\multirow
\usepackage{graphicx} 
\usepackage{wrapfig}
\usepackage{xcolor}
\usepackage{titlesec}     % 改变标题样式
\usepackage{enumitem}
\usepackage{harpoon}   %矢量符号
%\usepackage{natbib}         % 管理文献引用
\usepackage{aas_macros}

\newcommand{\myvec}[1]%
   {\stackrel{\raisebox{-2pt}[0pt][0pt]{\small$\rightharpoonup$}}{#1}}  %矢量符号
\renewcommand{\vec}[1]{\boldsymbol{#1}}
\newcommand{\me}{\mathrm{e}}
\newcommand{\mi}{\mathrm{i}}
\newcommand{\dif}{\mathrm{d}}
\newcommand{\tabincell}[2]{\begin{tabular}{@{}#1@{}}#2\end{tabular}}

\def\kpc{{\rm kpc}}
\def\km{{\rm km}}
\def\cm{{\rm cm}}
\def\TeV{{\rm TeV}}
\def\GeV{{\rm GeV}}
\def\MeV{{\rm MeV}}
\def\GV{{\rm GV}}
\def\MV{{\rm MV}}
\def\yr{{\rm yr}}
\def\s{{\rm s}}
\def\ns{{\rm ns}}
\def\GHz{{\rm GHz}}
\def\muGs{{\rm \mu Gs}}
\def\arcsec{{\rm arcsec}}
\def\K{{\rm K}}
\def\microK{\mu{\rm K}}
\def\sr{{\rm sr}}
\newcolumntype{p}{D{,}{\pm}{-1}}

\renewcommand{\figurename}{Fig.}
\renewcommand{\tablename}{Tab.}

\renewcommand{\arraystretch}{1.5}

\setlength{\parindent}{0pt}  %取消每段开头的空格

\title{Interstellar gas and magnetic  fields}
\author{}
\date{\today}
\begin{document}

\maketitle

\section{The interstellar medium in the life cycle of stars}
The mass of the interstellar gas amounts to about $5\%$ of the visible mass of our Galaxy. In the Galactic plane close to the Sun, the overall gas density is to about $10^6$ particles m$^{-3}$, but there are very wide variations in density and temperature from place to place throughout the interstellar medium.

\section{neutral interstellar gas}

\subsection{Neutral hydrogen : 21-cm line emission and absorption}
Neutral hydrogen emits line radiation at a frequency \textcolor{red}{$\nu_0 = 1420.4058$ MHz} (\textcolor{red}{$\lambda_0 = 21.1$ cm}) through an almost totally forbidden hyperfine transition in which the \textcolor{cyan}{spins of the electron and proton change from being parallel to antiparallel}. The spontaneous transition probability is \textcolor{cyan}{$A_{21} = 2.85 \times 10^{-15}$ s$^{-1}$} for the \textcolor{cyan}{ground state of hydrogen}, that is, about once every \textcolor{cyan}{$10^7$ years}. Because there are two possible orientations of the spins of both the electron and the proton, there are \textcolor{cyan}{four stationary states}, \textcolor{cyan}{three degenerate in the upper state} and \textcolor{cyan}{one in the lower state}. \textcolor{orange}{Because of the very small transition probability, collisions and other processes have time to establish an equilibrium distribution of hydrogen atoms in the upper and lower states}, labelled $2$ and $1$, respectively, and so the ratio of the number of atoms in these states is given by the Boltzmann distribution $N_2/N_1 = (g_2/g_1) \exp(−h\nu_0/kT)$. $T$ is the excitation temperature, i.e. \textcolor{red}{spin temperature $T_s$}. $g_2$ and $g_1$ are the statistical weights of the upper and lower levels, $g_2/g_1 = 3$. Under all cosmic conditions $h\nu_0/k = 7 \times 10^{-2}$ K $\ll T_s$ and therefore $N_2/N_1 = 3$.

If the emitting region is optically thin, only spontaneous emission need be considered and so the emissivity $\kappa_{21}$ of the gas is
\begin{equation}
\kappa = \frac{g_2}{g_2+g_1} N_{\rm H} A_{21} h \nu_0 = \frac{3}{4} N_{\rm H} A_{21} h \nu_0
\end{equation}
where NH is the number density of neutral hydrogen atoms.

If the neutral hydrogen is distributed along the line of sight from the observer, the flux
density received within solid angle $\Omega$, say, the solid angle subtended by the beam of the radio telescope, is
\begin{eqnarray}
\nonumber S &=& \int \frac{\kappa_{21}(r)}{4\pi r^2} \Omega r^2 \dif r ~, \\
I &=& \frac{S}{\Omega} = \frac{3}{16} A_{21} h \nu_0 \int N_{\rm H} \dif r ~.
\end{eqnarray}

where $r$ is distance along the line of sight. $I = S/\Omega$  is the intensity of radiation in that direction and is a measure of the total column density of neutral hydrogen along the line of sight $\int N_{\rm H} \dif r$. In this calculation I is measured in W m$^{-2}$ and is equal to the integral of the intensity of radiation per unit bandwidth $I_\nu$ over the line profile $I = \int I_\nu \dif \nu$.
 
Because of its very small transition probability, the natural linewidth of the 21-cm line is very narrow. If the neutral hydrogen is in motion relative to the observer, Doppler shifts of the 21-cm line emission can be readily measured by making observations with a multi-channel 21-cm line receiver. This provides a very powerful tool for investigating the dynamics of neutral hydrogen in our own and in other galaxies.


\subsection{Molecular radio lines}


\subsection{Optical and ultra violet absorption lines}


\subsection{X-ray absorption}


\section{Ionised interstellar gas}


\subsection{Thermal bremsstrahlung}


\subsection{Permitted and forbidden transitions in gaseous nebulae}


\subsection{The dispersion measure of pulsars}



\subsection{Faraday rotation of linearly polarised radio signals}



\section{Interstellar dust}



\section{An overall picture of the interstellar gas}


\subsection{Large scaled ynamics}


\subsection{Heating mechanisms}


\subsection{Cooling mechanisms}


\subsection{The overall state of the interstellar gas}


\section{Star formation}



\subsection{The initial mass function and the Schmidt–Kennicutt law}


\subsection{Regions of star formation}



\subsection{Issues in the theory of star formation}


\section{The Galactic magnetic field}




\subsection{Faraday rotation in the interstellar medium}



\subsection{Optical polarisation of starlight}



\subsection{Radio emission of spinning dust grains}



\subsection{Zeeman splitting of 21-cm line radiation}


\subsection{The radio emission from the Galaxy}

\subsection{Summary of the information on the Galactic magnetic field}

%%%%%%%%%%%%%%%%%%%%%%%%%%%%%%%%%%%%%%%%%%%%%%%%%%%%%%%%%%%%%%%%%%%%%%
\bibliographystyle{unsrt_update}
\bibliography{ref}
%%%%%%%%%%%%%%%%%%%%%%%%%%%%%%%%%%%%%%%%%%%%%%%%%%%%%%%%%%%%%%%%%%%%%%

\end{document}