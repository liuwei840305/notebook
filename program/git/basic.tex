\documentclass[12pt,a4paper]{article}
%\usepackage{fontspec, xunicode, xltxtra}  
%\setmainfont{Hiragino Sans GB}  
\usepackage{xeCJK}
%\setCJKmainfont[BoldFont=STZhongsong, ItalicFont=STKaiti]{STSong}
%\setCJKsansfont[BoldFont=STHeiti]{STXihei}
%\setCJKmonofont{STFangsong}

%使用Xelatex编译

% 设置页面
%==================================================
\linespread{2} %行距
% \usepackage[top=1in,bottom=1in,left=1.25in,right=1.25in]{geometry}
% \headsep=2cm
% \textwidth=16cm \textheight=24.2cm
%==================================================

% 其它需要使用的宏包
%==================================================
\usepackage[colorlinks,linkcolor=blue,anchorcolor=red,citecolor=green,urlcolor=blue]{hyperref} 
\usepackage{tabularx}
\usepackage{authblk}         % 作者信息
\usepackage{algorithm}     % 算法排版
\usepackage{amsmath}     % 数学符号与公式
\usepackage{amsfonts}     % 数学符号与字体
\usepackage{mathrsfs}      % 花体
\usepackage{amssymb}

\usepackage{graphicx} 
\usepackage{graphics}
\usepackage{color}
\usepackage{xcolor}

\usepackage{fancyhdr}       % 设置页眉页脚
\usepackage{fancyvrb}       % 抄录环境
\usepackage{float}              % 管理浮动体
\usepackage{geometry}     % 定制页面格式
\usepackage{hyperref}       % 为PDF文档创建超链接
\usepackage{lineno}          % 生成行号
\usepackage{listings}        % 插入程序源代码
\usepackage{multicol}       % 多栏排版
%\usepackage{natbib}         % 管理文献引用
\usepackage{rotating}       % 旋转文字,图形,表格
\usepackage{subfigure}    % 排版子图形
\usepackage{titlesec}       % 改变章节标题格式
\usepackage{moresize}   % 更多字体大小
\usepackage{anysize}
\usepackage{indentfirst}  % 首段缩进
\usepackage{booktabs}   % 使用\multicolumn
\usepackage{multirow}    % 使用\multirow

\usepackage{wrapfig}
\usepackage{titlesec}     % 改变标题样式
\usepackage{enumitem}
\usepackage{aas_macros}

\newcommand{\myvec}[1]%
   {\stackrel{\raisebox{-2pt}[0pt][0pt]{\small$\rightharpoonup$}}{#1}}  %矢量符号
\renewcommand{\vec}[1]{\boldsymbol{#1}}
\newcommand{\me}{\mathrm{e}}
\newcommand{\mi}{\mathrm{i}}
\newcommand{\dif}{\mathrm{d}}
\newcommand{\tabincell}[2]{\begin{tabular}{@{}#1@{}}#2\end{tabular}}

\def\kpc{{\rm kpc}}
\def\km{{\rm km}}
\def\cm{{\rm cm}}
\def\TeV{{\rm TeV}}
\def\GeV{{\rm GeV}}
\def\MeV{{\rm MeV}}
\def\GV{{\rm GV}}
\def\MV{{\rm MV}}
\def\yr{{\rm yr}}
\def\s{{\rm s}}
\def\ns{{\rm ns}}
\def\GHz{{\rm GHz}}
\def\muGs{{\rm \mu Gs}}
\def\arcsec{{\rm arcsec}}
\def\K{{\rm K}}
\def\microK{\mu{\rm K}}
\def\sr{{\rm sr}}
\newcolumntype{p}{D{,}{\pm}{-1}}

\renewcommand{\figurename}{Fig.}
\renewcommand{\tablename}{Tab.}

\renewcommand{\arraystretch}{1.5}

\setlength{\parindent}{0pt}  %取消每段开头的空格

\title{Basic}
\author{}
\date{\today}
\begin{document}

\maketitle

\section{Preparation}

\subsection{初始设置}
对本地计算机里安装的Git进行设置

设置使用 Git 时的姓名和邮箱地址

$\$$ git config $--$global user.name ``Firstname Lastname" \\
$\$$ git config $--$global user.email ``$\rm your\_email@example.com$"

这个命令,会在``$\rm \sim/.gitconfig$”中以如下形式输出设置文件

[user] \\
name = Firstname Lastname \\
email = $\rm your\_email@example.com$

想更改这些信息时,可以直接编辑这个设置文件。这里设置的姓名和邮箱地址会用在 Git 的提交日志中。由于在 GitHub 上公开仓库时,这里的姓名和邮箱地址也会随着提交日志一同被公开,所以请不要使用不便公开的隐私信息。

将 color.ui 设置为 auto 可以让命令的输出拥有更高的可读性

$\$$ git config $--$global color.ui auto \\
``$\rm \sim/.gitconfig$”中会增加下面一行。

[color] \\
ui = auto

\subsection{设置 SSH Key}
GitHub 上连接已有仓库时的认证,是通过使用了 SSH 的公开密钥认证方式进行的。创建公开密钥认证所需的 SSH Key,并将其添加至 GitHub。运行下面的命令创建 SSH Key

$\$$ ssh-keygen -t rsa -C ``$\rm your\_email@example.com$" \\
Generating public/private rsa key pair. \\
Enter file in which to save the key \\
$\rm (/Users/your\_user\_directory/.ssh/id\_rsa)$: 按回车键 \\
Enter passphrase (empty for no passphrase): 输入密码 \\
Enter same passphrase again: 再次输入密码

``$\rm your\_email@example.com$”的部分请改成您在创建账户时用的邮箱地址。密码需要在认证时输入,请选择复杂度高并且容易记忆的组合。输入密码后会出现以下结果。

Your identification has been saved in $\rm /Users/your\_user\_directory/.ssh/id\_rsa$. \\
Your public key has been saved in $\rm /Users/your\_user\_directory/.ssh/id\_rsa.pub$. \\
The key fingerprint is: \\
fingerprint值 $\rm your\_email@example.com$ \\
The key's randomart image is:

$\rm id\_rsa$ 文件是私有密钥,$\rm id\_rsa.pub$ 是公开密钥。

在 GitHub 中添加公开密钥,今后就可以用私有密钥进行认证了。点击右上角的账户设定按钮( Account Settings),选择 SSH Keys 菜单。点击 Add SSH Key 之后,会出现输入框。在 Title 中输入适当的密钥名称。 Key 部分请粘贴 $\rm id\_rsa.pub$ 文件里的内容。 $\rm id\_rsa.pub$的内容可以用如下方法查看。

$\$$ cat $\rm \sim/.ssh/id\_rsa.pub$ \\
ssh-rsa 公开密钥的内容 $\rm your\_email@example.com$

添加成功之后,创建账户时所用的邮箱会接到一封提示“公共密钥添加完成”的邮件。完成以上设置后,就可以用手中的私人密钥与 GitHub 进行认证和通信了。

$\$$ ssh -T git@github.com \\
The authenticity of host 'github.com (207.97.227.239)' can't be established. \\
RSA key fingerprint is fingerprint值 . \\
Are you sure you want to continue connecting (yes/no)? 输入yes \\

出现如下结果即为成功。\\
Hi hirocastest! You've successfully authenticated, but GitHub does not provide shell access.

创建一个公开的仓库。点击右上角工具栏里的 New repository 图标,创建新的仓库。

在 Initialize this repository with a README 选项上打钩,随后 GitHub 会自动初始化仓库并设置 README 文件,让用户可以立刻clone 这个仓库。如果想向 GitHub 添加手中已有的 Git 仓库,建议不要勾选,直接手动 push。

下方左侧的下拉菜单非常方便,通过它可以在\textcolor{yellow}{初始化时自动生成 .gitignore 文件\footnote{该文件用来描述 Git 仓库中不需管理的文件与目录。}}。这个设定会帮我们把\textcolor{yellow}{不需要在 Git 仓库中进行版本管理的文件记录在 .gitignore 文件中},省去了每次根据框架进行设置的麻烦。下拉菜单中包含了主要的语言及框架,选择今后将要使用的即可。

右侧的下拉菜单可以选择要添加的许可协议文件。如果这个仓库中包含的代码已经确定了许可协议,那么请在这里进行选择。随后将自动生成包含许可协议内容的 LICENSE 文件,用来表明该仓库内容的许可协议。

输入选择都完成后,点击 Create repository 按钮,完成仓库的创建。

下面这个 URL 便是刚刚创建的仓库的页面。\\
https://github.com/用户名/Hello-Word

README.md 在初始化时已经生成好了。README.md 文件的内容会自动显示在仓库的首页当中。因此,人们一般会在这个文件中标明本仓库所包含的软件的概要、使用流程、许可协议等信息。如果使用Markdown 语法进行描述,还可以添加标记,提高可读性。

在 GitHub 上进行交流时用到的 Issue、评论、 Wiki,都可以用Markdown 语法表述,从而进行标记。准确地说应该是 GitHub Flavored Markdown( GFM)语法。该语法虽然是 GitHub 在 Markdown 语法基础上扩充而来的,但一般情况下只要按照原本的 Markdown 语法进行描述就可以。使用 GitHub 后,很多文档都需要用 Markdown 来书写。

将已有仓库 clone 到身边的开发环境中。

$\$$ git clone git@github.com:hirocastest/Hello-World.git \\
Cloning into 'Hello-World'... \\
remote: Counting objects: 3, done. \\
remote: Total 3 (delta 0), reused 0 (delta 0) \\
Receiving objects: $100\% (3/3)$, done.

这里会要求输入 GitHub 上设置的公开密钥的密码。认证成功后,仓库便会被 clone 至仓库名后的目录中。将想要公开的代码提交至这个仓库再 push 到 GitHub 的仓库中,代码便会被公开。

添加至 Git 仓库的文件显示为 Untracked files。通过 \textcolor{red}{git add} 命令将文件加入暂存区 A,再通过 \textcolor{red}{git commit} 命令提交。添加成功后,可以通过 \textcolor{red}{git log} 命令查看提交日志。之后只要执行 \textcolor{red}{git push},GitHub 上的仓库就会被更新。



\section{基本操作}
要使用 Git 进行版本管理,必须先初始化仓库。 Git 是使用 \textcolor{red}{\bf git init} 命令\textcolor{blue}{进行初始化}的。建立一个目录并初始化仓库。如果初始化成功,执行了 git init命令的目录下就会生成 \textcolor{red}{.git 目录}。这个 .git 目录里\textcolor{blue}{存储着管理当前目录内容所需的仓库数据}。

在 Git 中,将这个目录的内容称为``\textcolor{blue}{附属于该仓库的工作树}”。文件的编辑等操作在工作树中进行,然后记录到仓库中,以此管理文件的历史快照。如果想将文件恢复到原先的状态,可以从仓库中调取之前的快照,在工作树中打开。开发者可以通过这种方式获取以往的文件。

 \textcolor{red}{\bf git status} 命令用于显示 Git 仓库的状态。工作树和仓库在被操作的过程中,状态会不断发生变化。在 Git 操
作过程中时常用 git status 命令查看当前状态。

$\$$ git status \\
$\#$ On branch master \\
$\#$  \\
$\#$  Initial commit \\
$\#$  \\
nothing to commit (create/copy files and use ``git add" to track) 

结果显示了当前正处于 master 分支下。接着还显示了没有可提交的内容。所谓\textcolor{red}{提交(Commit)},是指``\textcolor{orange}{记录工作树中所有文件的当前状态}"。尚没有可提交的内容,就是说当前建立的这个仓库中还没有记录任何文件的任何状态。只要对 Git 的工作树或仓库进行操作, git status 命令的显示结果就会发生变化。

如果只是用 Git 仓库的工作树创建了文件,那么该文件并不会被记入 Git 仓库的版本管理对象当中。要想让文件成为 Git 仓库的管理对象,就需要用 \textcolor{red}{\bf git add} 命令将其\textcolor{orange}{加入暂存区}(Stage 或者 Index)中。\textcolor{blue}{暂存区是提交之前的一个临时区域}。

$\$$ git add README.md \\
$\$$ git status \\
$\#$ On branch master \\
$\#$ \\
$\#$ Initial commit \\
$\#$ \\
$\#$ Changes to be committed: \\
$\#$ (use ``git rm $--$cached <file>..." to unstage) \\
$\#$ \\
$\#$ new file: README.md \\

README.md 文件显示在 Changes to be committed 中了。

\textcolor{red}{\bf git commit }命令可以\textcolor{blue}{将当前暂存区中的文件实际保存到仓库的历史记录中。通过这些记录,可以在工作树中复原文件。}

$\$$ git commit $\rm -m$ ``First commit" \\
$\rm [master (root-commit) 9f129ba]$ First commit \\
1 file changed, 0 insertions(+), 0 deletions(-) \\
create mode 100644 README.md 

\textcolor{orange}{-m 参数后}的 ``First commit"称作\textcolor{red}{提交信息},\textcolor{orange}{是对这个提交的概述}。想要记述得更加详细,\textcolor{orange}{不加 -m},\textcolor{red}{直接执行 git commit命令}。执行后编辑器就会启动,并显示如下结果。

$\#$ Please enter the commit message for your changes. Lines starting \\
$\#$ with '$\#$' will be ignored, and an empty message aborts the commit. \\
$\#$ On branch master \\
$\#$ \\
$\#$ Initial commit \\
$\#$ \\
$\#$ Changes to be committed: \\
$\#$ (use ``git rm $--$cached <file>..." to unstage) \\
$\#$ \\
$\#$ new file: README.md 

在编辑器中\textcolor{orange}{记述提交信息的格式}如下: \\
第一行:用一行文字简述提交的更改内容 \\
第二行:空行 \\
第三行以后:记述更改的原因和详细内容 

只要按照上面的格式输入,今后可以通过确认日志的命令或工具看到这些记录。在\textcolor{orange}{以 $\#$(井号)标为注释的 Changes to be committed(要提交的更改)栏中,可以查看本次提交中包含的文件}。将提交信息按格式记述完毕后,请保存并关闭编辑器,以$\#$(井号)标为注释的行不必删除。随后,刚才记述的提交信息就会被提交。

如果在编辑器启动后想中止提交,请将提交信息留空并直接关闭编辑器,随后提交就会被中止。

当前工作树处于刚刚完成提交的最新状态,所以结果显示没有更改。

\textcolor{red}{\bf git log} 命令可以\textcolor{blue}{查看以往仓库中提交的日志}。包括可以查看什么人在什么时候进行了提交或合并,以及操作前后有怎样的差别。

$\$$ git log \\
commit 9f129bae19b2c82fb4e98cde5890e52a6c546922 \\
Author: hirocaster <hohtsuka@gmail.com> \\
Date: Sun May 5 16:06:49 2013 +0900 \\
First commit 

commit 栏旁边显示的``9f129b……"是\textcolor{blue}{指向这个提交的哈希值}。 Git 的其他命令中,在指向提交时会用到这个哈希值。Author 栏中显示 Git 设置的用户名和邮箱地址。 Date 栏中显示提交执行的日期和时间。再往下就是该提交的提交信息。

如果只想让程序显示第一行简述信息,可以在 git log命令后加上 \textcolor{violet}{$--$pretty=short}。

只要在 git log命令后加上目录名,便会只显示该目录下的日志。果加的是文件名,就会只显示与该文件相关的日志。






Git commands :

\textcolor{blue}{(master) $\$:$ git commandname parameter1 parameter2 $--$option}

The command name (commandname in the example) is one of over $100$ individual functions that Git can perform. Behind the scenes, each of these commands is a separate program responsible for its own specific job. 

Options are special parameters that are denoted by at least one leading dash character. Many options have both a \textcolor{orange}{long form}, like \textcolor{blue}{$--$global}, and a \textcolor{orange}{shortcut form}, like \textcolor{blue}{$-$g}. There are also options that take values, like \textcolor{blue}{git commit $--$message=``hello world"}.

There are two that it absolutely needs in order to function: \textcolor{red}{your name} and \textcolor{red}{email address}. \textcolor{orange}{Git adds an Author attribute to every commit you make} that includes both your name and email address, so that your collaborators on a project can know who made a given change. The name you enter will be used to identify you in change logs and any other place where Git shows who made a particular change, while your email address
not only tells people how to reach you, but also tells a hosted service like GitHub who you are on their service.

Use the \textcolor{blue}{git config} command to tell Git who you are. Unlike most Git commands, which \textcolor{orange}{only work inside of a Git project}, these can be \textcolor{orange}{run from any directory}. 

%%%%%%%%%%%%%%%%%%%%%%%%%%%%%%%%%%%%%%%%%%%%%%%%%%%%%%%%%%%%%%%%%%%%%%
\bibliographystyle{unsrt_update}
\bibliography{ref}
%%%%%%%%%%%%%%%%%%%%%%%%%%%%%%%%%%%%%%%%%%%%%%%%%%%%%%%%%%%%%%%%%%%%%%

\end{document}